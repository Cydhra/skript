\chapter{Glossar}
\section{Begriffserklärungen}
\begin{description}
	\item[Bildraum] Bei einer Funktion $f\colon A \rightarrow B$ bezeichnet $B$ den Bildraum.
	\item[Diskreter Logarithmus] Bezeichne $\mathbbm{G} = \langle g \rangle$ eine endliche zyklische Gruppe mit Ordnung $N$. Dann gibt es für
	$\forall h \in \mathbbm{G} : \exists x \in \mathbbm{Z}_N :  g^x \equiv h$ und es bezeichnet $x = \log_g h$ den diskreten Logarithmus von $h$
	bezüglich $g$.
	\item[Gleichverteilung] Gilt für eine Verteilung $U$ über der Menge $M$, dass
	\begin{align*}
		\forall x \in M : \Pr [x \leftarrow U] = p\, ,
	\end{align*}
	heißt $U$ Gleichverteilung.
	\item[Gruppe] Es sei $M$ eine Menge und $\ast$ eine abgeschlossene Verknüpfung auf $M$. Dann heißt $(M, \ast)$ eine Gruppe, falls
	\begin{enumerate}
		\item das Assoziativgesetz gilt,
		\item ein neutrales Element $e_M \in M$ und
		\item $\forall x \in M : x^{-1} \in M$.
	\end{enumerate}
	\item[Gruppenordnung] Bezeichne $\mathbbm{G} = (M, \ast)$, dann heißt $\vert M \vert$ Gruppenordnung von $\mathbbm{G}$. Umgangssprachlich
	schreibt man auch $\vert \mathbbm{G} \vert$.
	\item[Heuristik] Eine Heuristik ist eine plausible, aber nicht bewiesene, Annahme über ein System.
	\item[Homomorphismus] Ein Homomorphismus bezeichnet eine strukturerhaltende Abbildung. Für ein homomorphes Verschlüsselungsverfahren $\enc$
	und zwei Nachrichten $M_1, M_2$ (die Elemente einer additiven Gruppe sind) sähe das \emph{beispielsweise} folgendermaßen aus:
	\begin{align*}
		\enc(M_1 + M_2) = \enc(M_1) \cdot \enc(M_2)
	\end{align*}
	\item[Kollision] Falls für eine (Hash-)Funktion $H\colon A \rightarrow B : \exists x, x' \in A : x \neq x' \land H(x) = H(x')$ spricht man von einer Kollision in $H$.
	\item[Kryptosystem] Ein System bestehend aus Verschlüsselungs- und dazugehörigem Entschlüsselungsalgorithmus.
	\item[Padding] Ein Mechanismus, um eine gewisse Menge an Daten auf eine vorgeschriebene (Block-)Länge aufzufüllen. %Hierzu werden
	%beispielsweise 0en oder 1en an das Ende des Bitstroms geschrieben. Im kryptographischen Kontext kann es allerdings auch sinnvoll sein,
	%die Daten mit Pseudozufall aufzufüllen.
	\item[Permutation] Bezeichne $\{L_n\}$ die Menge geordneter Listen der Elemente $\{l_1, \dots, l_n\}$. Dann heißt $\phi\colon \{L_n\} \rightarrow \{L_n\}$ eine Permutation.
	\item[Prüfsumme] Ein Mechanismus zur (approximativen) Gewährleistung der Datenintegrität bei Datenübertragung und Datensicherung.
	\item[Semantik] Die ursprüngliche Wortbezeichnung beschreibt ein Teilgebiet der Linguistik, dass sich mit der Bedeutung von Zeichen oder Zeichenfolgen
	auseinandersetzt. Im informationstheoretisch-kryptographischen Kontext wird es gelegentlich auch synonym zu Information verwendet (Vgl. 
	\ref{def:semsec}).
	\item[Untergruppe] Bezeichne $\mathbbm{G} = (M, \ast)$ eine Gruppe. Dann bezeichnet $\mathbbm{H} = (M', \ast)$ eine Untergruppe von $\mathbbm{G}$,
	falls
	\begin{enumerate}
		\item $M' \subseteq M$,
		\item die Verknüpfung $\ast$ in $\mathbbm{H}$ abgeschlossen ist,
		\item das neutrale Element $e_M \in \mathbbm{H}$ und
		\item für alle $x \in \mathbbm{H} : x^{-1} \in \mathbbm{H}$.
	\end{enumerate}
	Umgangssprachlich schreibt man $\mathbbm{H} \subseteq \mathbbm{G}$.
	\item[Urbildraum] Bei einer Funktion $f\colon A \rightarrow B$ bezeichnet $A$ den Urbildraum.
\end{description}

\section{Mathematische Bezeichnungen}
\begin{description}
	\item[$\mathbbm{Z}^{\ast}_p$] Zyklische multiplikative Gruppe ganzer Zahlen, die kleiner $p$ und koprim zu $p$ sind, das heißt $\{x : \ggT(x, p) = 1\}$
	\item[$\mathbbm{Z}_N$] Zyklische additive Gruppe ganzer Zahlen modulo $N$, das heißt $\{0, \dots, N-1\}$
	\item[$\mathbbm{F}^{\ast}_q$] Multiplikative Gruppe des dazugehörigen Galois-Körpers $\mathbbm{F}_q$
\end{description}

\section{Notationsformalismus}
%\begin{tabular}{l|l}
%	$A \vert B$ & kdsaldjskladjalksj\\
%	$\A^{\B}$ &dsa\\
%	$Adv^{cr}_{H,\A}(k)$ & dsa \\
%	$Adv^{ow}_{H,\A}(k)$ & dsa \\
%	$Adv^{tcr}_{H,\A}(k)$ & dsa
%\end{tabular}

\section{Komplexitätsklassen}
%\begin{tabular}{l|l}
%	$P$ & dsa \\
%	$NP$ & dsa
%	%NP-vollständig
%\end{tabular}

%Teta / Groß-O, Omega
% a <- M, probabilistische Zuweisung einer Variablen 
%Unterscheider
%Schlüsselzentrale
%Replay-Attacke
%Sitzungsschlüssel
%Client / Server 
%forward-secrecy
%Meet-in-the-Middle
%Man-in-the-Middle
%kryptografische Hashfunktion
%binäre Suche
%Reduktionsfunktion