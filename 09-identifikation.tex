\chapter{Identifikationsprotokolle}
Nachdem wir jetzt Authentifikation von Nachrichten und den authentifizierten Austausch von Schlüsseln betrachtet haben, befasst sich dieses Kapitel mit der
asymmetrischen Identifikation von Kommunikationsteilnehmern. Das bedeutet, Alice ist im Besitz eines geheimen Schlüssels $\skey$ und Bob, der den dazugehörigen
öffentlichen Schlüssel $\pkey$ kennt, möchte sicher sein, dass er mit einer Instanz redet, die in Besitz von $\skey$ ist. Üblicherweise geht es bei dieser
Prüfung um den Nachweis einer Identität, der an bestimmte (Zugangs-)Rechte gekoppelt ist.

Da Alice im Folgenden \emph{beweisen} muss, dass sie den geheimen Schlüssel besitzt, und Bob ihre Identität \emph{überprüft}, heißen die beiden für den Rest
dieses Kapitels \texttt{Prover} \indexProver und \texttt{Verifier}\indexVerifier.

Der einfachste Weg, dem Verifier zu beweisen, dass der Prover das Geheimnis $\skey$ kennt, ist es, ihm den Schlüssel einfach direkt zu schicken. Der Verifier
kann dann die Zugehörigkeit zu $\pkey$ feststellen und sicher sein, dass der Prover das Geheimnis kennt. Allerdings wird bei diesem Vorgehen $\skey$ allgemein
bekannt und garantiert nach der ersten Verwendung keine Zuordnung mehr zu einer bestimmten Identität.

Die Protokollanforderungen steigen also darauf, dass der Verifier sicher sein kann, dass der Prover das Geheimnis kennt, der Verifier selbst jedoch $\skey$
nicht lernt.

Ein zweiter Versuch umfasst die bereits entwickelten Signaturschemata. Der Prover schickt $\sigma := \sig(\skey_A, \text{"`ich bin's, P"'})$ an den Verifier.
$\ver(\pkey_A,\text{"`ich bin's, P"'}, \sigma)$ liefert dem Verifier die Gültigkeit der entsprechenden Signatur und damit die Identität des Absenders. Um die
Signatur zu fälschen, müsste ein Angreifer also das dahinterstehende Signaturverfahren brechen. Allerdings kann er die Signatur $\sigma$ mit dieser trivialen
Nachricht einfach wiederverwenden und sich so entweder als Man-in-the-Middle oder mithilfe eine Replay-Attacke Ps Identität zunutze machen.

Aus den ersten beiden Versuchen geht hervor, dass wir ein interaktives Protokoll wie in Abbildung \ref{fig:id:interaktiv} benötigen, um den geheimen Schlüssel
gleichzeitig zu verbergen und den Besitz dieses Geheimnisses zu beweisen.%
\footnote{In der Praxis mag es sinnvoll sein, nicht nur die Zufallszahl $R$ zu signieren, sondern dieser noch das aktuelle Datum und die aktuelle Uhrzeit hinzuzufügen. So kann, selbst wenn der Verifier irgendwann zum zweiten Mal die selbe Zufallszahl ausgibt, eine gerade erzeugte von einer alten Signatur unterschieden werden.}

\begin{figure}[h]
\begin{center}
\unitlength=1mm
\linethickness{0.4pt}
\hspace{-3 cm}
\begin{picture}(30,20)
    \put(10,15){\makebox(0,0)[cb]{$\mathtt{Prover}_{\skey_A}$}}
    \put(50,15){\makebox(0,0)[cb]{$\mathtt{Verifier}_{\pkey_A}$}}
    
    \put(10,0){\line(0,1){13}}
    \put(50,0){\line(0,1){13}}

    \put(50,10){\vector(-1,0){40}}
    \put(30,11){\makebox(0,0)[cb]{$R$}}
    
    \put(10,3){\vector(1,0){40}}
    \put(30,4){\makebox(0,0)[cb]{$\sigma := \sig(\skey_A, R)$}}
\end{picture}
\end{center}
\caption{Interaktives Protokoll, in dem der Verifier dem Prover eine Zufallszahl $R$ gibt, um dessen Identität durch eine Signatur sicherzustellen.}
\label{fig:id:interaktiv}
\end{figure}

\section{Sicherheitsmodell}
Ein Public-Key-Identifikationsprotokoll \indexPKIdentificationProtocoll ist definiert durch das Tupel $(\gen, \mathrm{P}, \mathrm{V})$ von PPT-Algorithmen. Dabei gibt
$\gen$ wie gewohnt bei Eingabe eines Sicherheitsparameters $1^k$ das Schlüsselpaar $(\pkey, \skey)$ aus. Der Prover P und der Verifier V
sind zustandsbehaftet und interagieren während des Identitätsnachweises miteinander.
\begin{enumerate}
  \item V erhält den öffentlichen Schlüssel $\pkey_P$ als Eingabe und gibt $\mathrm{out}_V$ aus
  \item P erhält Vs Ausgabe $\mathrm{out}_V$ und den privaten Schlüssel $\skey_P$ und gibt $\mathrm{out}_P$ aus
  \item V erhält Ps Ausgabe $\mathrm{out}_P$ und gibt $\mathrm{out}_V$ aus
  \item ist $\mathrm{out}_V \in \{0,1\}$ beende die Interaktion, ansonsten springe zurück zu Schritt 2
\end{enumerate}
Der Verifier erzeugt also eine Ausgabe, mit deren Hilfe P beweisen muss, dass er das Geheimis $\skey$ kennt. P liefert auf Basis des Geheimnisses und der
Ausgabe von V seinerseits eine Ausgabe und gibt diese an V weiter. V prüft das Ergebnis und entscheidet, ob die Prüfung erfolgreich abgeschlossen wurde. Falls ja, gibt
er 1 aus, falls nein 0.

Das Verfahren muss \emph{korrekt} sein, also muss schließlich gelten:
\begin{align*}
\forall (\pkey, \skey) \leftarrow \gen(1^k): V(\mathrm{out}_P) \rightarrow 1
\end{align*}
$\langle \mathrm{P}(\skey), \mathrm{V}(\pkey) \rangle$ bezeichnet im Folgenden das Transkript der Interaktion zwischen Prover und Verifier.

Einem Angreifer $\A$ darf es nun intuitiv nicht möglich sein, gegenüber einem Verifier die Identität eines anderen anzunehmen. Um das
überprüfen zu können, führen wir ein neues Spiel ein. Zunächst erzeugt das Spiel i $(\pkey, \skey)$-Paare und ordnet die privaten
Schlüssel i Provern zu.
\begin{enumerate}
  \item $\A$ darf nun mit beliebig vielen dieser gültigen Prover interagieren. Dabei nimmt er die Rolle des Verifiers ein und hat
  demnach Zugriff auf die passenden öffentlichen Schlüssel $\pkey_i$, während die gültigen Prover seine Anfragen mit ihren
  privaten Schlüsseln $\skey_i$ beantworten.
  \item $\A$ wählt sich nun einen der $\pkey_{i^*}$ aus und stellt sich damit als Prover dem \emph{echten} Verifier mit der Eingabe
  $\pkey_{i^*}$.
  \item $\A$ gewinnt, wenn der Verifier als Ergebnis schließlich 1 ausgibt.
\end{enumerate}
Wir nennen ein Public-Key-Identifikationsprotokoll $(\gen, \mathrm{P}, \mathrm{V})$ sicher, wenn kein PPT-Angreifer $\A$ das oben
genannte Spiel häufiger als vernachlässigbar oft gewinnt.

Allerdings verhindert das oben genannte Spiel keinen Man-in-the-Middle-Angriff, in dem $\A$ die Ausgaben einfach weiterreicht.


\section{Protokolle}
Unter diesem Aspekt können wir nun unseren Vorschlag aus Abbildung \ref{fig:id:interaktiv} aufgreifen und untersuchen. Dieser Ansatz basiert auf einem
Signaturverfahren. Seine Sicherheit ist demnach von der Sicherheit des verwendeten Signaturalgorithmus abhängig.~\\

\begin{theorem}
Ist das verwendete Signaturverfahren EUF-CMA-sicher, so ist das in Abbildung \ref{fig:id:interaktiv} gezeigte PK-Identifikationsprotokoll $(\gen,
\mathrm{P}, \mathrm{V})$ \indexPKIdentificationProtocoll sicher.~\\
\end{theorem}
\begin{beweisidee}
Angenommen, es gibt einen Angreifer $A$, der das PK-Identifikationsprotokoll bricht. Dann ist er in der Lage, nicht-vernachlässigbar oft aus
dem öffentlichen Schlüssel $\pkey_{i^*}$ und einer vom Verifier ausgewählten Zufallszahl $R$ eine Signatur $\sigma := \sig(\skey_{i^*}, R)$
zu berechnen.

Aus $A$ kann nun ein Angreifer $B$ konstruiert werden, der die Ergebnisse von $A$ nutzt, um das EUF-CMA-sichere Signaturverfahren zu
brechen.~\\
\end{beweisidee}

Ein weiterer Ansatz für ein funktionierendes Identifikationsprotokoll auf Public-Key-Basis ist in Abbildung \ref{fig:id:protokoll2}
dargestellt.
Hier wird $R$ vor der Übertragung über die Leitung vom Verifier mit $\pkey_{i^*}$ verschlüsselt, sodass die Kenntnis von $\skey_{i^*}$ durch
einen Entschlüsselungsvorgang überprüft wird.

Es ist hierbei darauf zu achten, dass das Schlüsselpaar, das für dieses Identifikationsprotokoll verwendet wird, nicht auch zum
Verschlüsseln gebraucht werden sollte. Ansonsten kann ein Angreifer in der Rolle des Verifiers die Entschlüsselung von ihm bekannten Chiffraten herbeiführen und somit jedes beliebige Chiffrat entschlüsseln lassen.

\begin{figure}[h]
\begin{center}
\unitlength=1mm
\linethickness{0.4pt}
\hspace{-3 cm}
    \begin{picture}(50,30)(0,0)
        \put(10,22){\makebox(0,0)[cb]{\texttt{Prover}$_{\skey_{i^*}}$}}
        \put(10,0){\line(0,1){20}}
    
        \put(70,22){\makebox(0,0)[cb]{\texttt{Verifier}$_{\pkey_{i^*}}$}}
        \put(70,0){\line(0,1){20}}
        
        \put(40,16){\makebox(0,0)[cb]{$\ciphert \leftarrow \enc(\pkey_{i^*}, R)$}}
        \put(70,15){\vector(-1,0){60}}
        
        \put(40,6){\makebox(0,0)[cb]{$R = \dec(\skey_{i^*}, \ciphert)$}}
        \put(10,5){\vector(1,0){60}}    
    \end{picture}
\end{center}
\caption{Dieses Identifikationsprotokoll profitiert von der Sicherheit des verwendeten Public-Key-Verschlüsselungsverfahrens.}
\label{fig:id:protokoll2}
\end{figure}

\begin{theorem}
Ist das in Abbildung \ref{fig:id:protokoll2} verwendete Verschlüsselungsverfahren IND-CCA-sicher, so ist das darauf basierende
PK-Identifikationsprotokoll \indexPKIdentificationProtocoll $(\gen, \mathrm{P}, \mathrm{V})$ sicher.
\end{theorem}

\begin{beweisidee}
Der Beweis dafür läuft analog zum obigen. Aus einem Angreifer $A$, der das Identifikationsprotokoll nicht vernachlässigbar oft bricht, wird
ein Angreifer $B$ konstruiert, der das IND-CCA-sichere Verschlüsselungsverfahren bricht.
\end{beweisidee}

Identifikationsprotokolle \indexPKIdentificationProtocoll wie die in Abbildungen \ref{fig:id:interaktiv} und \ref{fig:id:protokoll2} gezeigten heißen auch "`Challenge-Response-Verfahren"'\indexChallengeResponce, denn der Verifier stellt dem Prover eine Aufgabe (oder Herausforderung, die "`Challenge"'), die nur der echte Prover lösen kann. In dem Protokoll aus Abbildung \ref{fig:id:interaktiv} ist diese Aufgabe die Erstellung einer Signatur für einen Zufallsstring~$R$; in Abbildung \ref{fig:id:protokoll2} ist diese Aufgabe die Entschlüsselung eines zufälligen Chiffrats $C = \enc(\pkey_{i^*}, R)$. Die Lösung des Provers wird daher auch als die Antwort, oder "`Response"' bezeichnet.
