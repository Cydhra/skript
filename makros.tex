%%%%%%%%%%%%%%%%%%%%%%%%%%%%%%%%%%%%%%%%
%%% Makros für einheitliche Notation %%%
%%%%%%%%%%%%%%%%%%%%%%%%%%%%%%%%%%%%%%%%

\newcommand{\changefont}[3]{\fontfamily{#1} \fontseries{#2} \fontshape{#3} \selectfont}

%%% Notation für Gruppen, Ringe und Körper
\newcommand{\N}{\ensuremath{\mathbbm{N}}}
\newcommand{\No}{\ensuremath{\mathbbm{N}_0}}
\newcommand{\Z}[1]{\ensuremath{\mathbbm{Z}_{#1}}}
\newcommand{\Zx}[1]{\ensuremath{\mathbbm{Z}_{#1}^{\times}}}
\newcommand{\Q}{\ensuremath{\mathbbm{Q}}}
\newcommand{\R}{\ensuremath{\mathbbm{R}}}
\renewcommand{\P}{\ensuremath{\mathbbm{P}}}
\newcommand{\F}[1]{\ensuremath{\mathbbm{F}_{#1}}}
\newcommand{\Fx}[1]{\ensuremath{\mathbbm{F}_{#1}^{\times}}}
\newcommand{\K}{\ensuremath{\mathbbm{K}}}
\newcommand{\G}{\ensuremath{\mathbbm{G}}}
\newcommand{\Gx}{\ensuremath{\mathbbm{G}^{\times}}}

% For compatibility.
\newcommand{\calA}{\ensuremath{\mathcal{A}}}
\newcommand{\calC}{\ensuremath{\mathcal{C}}}
\newcommand{\calL}{\ensuremath{\mathcal{L}}}
\newcommand{\calO}{\ensuremath{\mathcal{O}}}
\newcommand{\calP}{\ensuremath{\mathcal{P}}}
\newcommand{\calR}{\ensuremath{\mathcal{R}}}
\newcommand{\calS}{\ensuremath{\mathcal{S}}}

%%% Notation für Kryptographie und Sonstiges
\newcommand{\plaint}{\ensuremath{M}}
\newcommand{\ciphert}{\ensuremath{C}}
\newcommand{\key}{\ensuremath{K}}
\newcommand{\skey}{\ensuremath{sk}}
\newcommand{\pkey}{\ensuremath{pk}}
\newcommand{\secpara}{\ensuremath{k}}
\newcommand{\enc}{\textsc{Enc}}
\newcommand{\dec}{\textsc{Dec}}
\newcommand{\sig}{\textsc{Sig}}
\newcommand{\ver}{\textsc{Ver}}
\newcommand{\keygen}{\textsc{KeyGen}}
\newcommand{\gen}{\textsc{Gen}}
\newcommand{\hash}{\ensuremath{h}}
\newcommand{\Com}{\textsc{Com}} %Commitment-Algorithmus
\newcommand{\A}{\ensuremath{\mathcal{A}}}
\newcommand{\advA}{\ensuremath{\mathcal{A}}}
\newcommand{\B}{\ensuremath{\mathcal{B}}}
\newcommand{\C}{\ensuremath{\mathcal{C}}}
\newcommand{\Sim}{\ensuremath{\mathcal{S}}} % Simulator
\newcommand{\ext}{\ensuremath{\mathcal{E}}} % Extraktor
\newcommand{\pw}{\ensuremath{\texttt{pw}}}

\newcommand\adv[2]{\mathbf{Adv}^{#1}_{#2}}
\DeclareMathOperator{\ggT}{ggT}
\DeclareMathOperator{\ggt}{ggT}
\newcommand*{\rArrow}{\ensuremath{\rightarrow}}
\newcommand*{\concat}{\ensuremath{\mathbin{\|}}} % Konkatenationssymbol %

\newcommand{\randUnif}{\xleftarrow{\textdollar}}

%Index-Macros
\newcommand{\indexCaesarROT}{\index{Stromchiffre!ROT-13}}
\newcommand{\indexVignere}{\index{Stromchiffre!Vigenère-Chiffre}}
\newcommand{\indexCaesar}{\index{Stromchiffre!Caesar-Chiffre}}
\newcommand{\indexOTP}{\index{Stromchiffre!One-Time-Pad}}
\newcommand{\indexLFSR}{\index{Stromchiffre!Linear Feedback Shift Register (LFSR)}}

\newcommand{\indexSeed}{\index{Seed|see {Pseudozufallszahlengenerator}}}

\newcommand{\indexPNS}{\index{Pseudozufallsfolge}}
\newcommand{\indexPRNG}{\index{Pseudozufallszahlengenerator}}

\newcommand{\indexECB}{\index{Betriebsmodus!Electronic Codebook Mode (ECB-Modus)}}
\newcommand{\indexCBC}{\index{Betriebsmodus!Cipher Block Chaining Mode (CBC-Modus)}}
\newcommand{\indexCTR}{\index{Betriebsmodus!Counter Mode (CTR-Modus)}}


\newcommand{\indexFeistel}{\index{Blockchiffren!Feistel networks}}
\newcommand{\indexDES}{\index{Blockchiffren!Data Encryption Standard (DES)}}
\newcommand{\indexTwoDES}{\index{2DES}}
\newcommand{\indexThreeDES}{\index{Triple Data Encryption Standard (3DES)}}
\newcommand{\indexAES}{\index{Blockchiffren!Advanced Encryption Standard (AES)}}


\newcommand{\indexSecParam}{\index{Sicherheitsparameter}}

\newcommand{\indexNegl}{\index{Vernachlässigbarkeit}}

\newcommand{\indexINDCPA}{\index{Indistinguishability under chosen-plaintext attacks}}
\newcommand{\indexINDCCA}{\index{Indistinguishability under chosen-ciphertext attacks}}

\newcommand{\indexKeyExchange}{\index{Schlüsselaustausch}}

\newcommand{\indexOracle}{\index{Orakel}}

\newcommand{\indexAdv}{Angreifer}
\newcommand{\indexEfficientAdv}{\index{Angreifer!Effizient}}
\newcommand{\indexPPTAdv}{\index{Angreifer!Probabilistic Polynomial Time (PPT)}}

\def\dh{d.\,h.\ }

%%% Umgebungen

\theoremstyle{plain}

\newtheorem{theorem}{Theorem}[chapter]
\newtheorem*{beweis}{Beweis}
\newtheorem*{beweisidee}{Beweisidee}
\newtheorem{beispiel}[theorem]{Beispiel}

\theoremstyle{definition}

\newtheorem{definition}[theorem]{Definition}