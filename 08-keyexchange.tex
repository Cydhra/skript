\chapter{Schlüsselaustauschprotokolle}
\label{cha:keyexchange}

In diesem Kapitel widmen wir uns der offenen Frage nach dem Schlüsselaustauschproblem, das insbesondere bei der Besprechung von symmetrischen Verschlüsselungs-
und Signaturverfahren einige Male aufgekommen ist. Zwei Kommunikationspartner Alice und Bob können ohne vorherigen Schlüsselaustausch keine sichere Verbindung
einrichten. Allerdings werden sie nicht jedes Mal die Möglichkeit haben, sich vor ihrer eigentlichen Kommunikation privat zu treffen, um einen gemeinsamen
Sitzungsschlüssel auszuhandeln. Vielleicht kennen sie einander nicht einmal persönlich, auf jeden Fall aber wäre ein solches Vorgehen sichtlich nicht
praktikabel.

Alice und Bob müssen also die unsichere Leitung zum Schlüsselaustausch verwenden. Den Schlüssel im Klartext darüber zu senden, würde einen Mithörer
trivial in die Situation bringen, auch den verschlüsselten Teil der darauf folgenden Kommunikation mitzulesen. Der neue Sitzungsschlüssel
$\key$ von Alice und Bob muss also bereits so über die Leitung gesendet werden, dass ein Lauscher nicht in der Lage ist, den Schlüssel zu rekonstruieren. Dabei
sind folgende grundlegende Szenarien denkbar:\indexSecretKeyInfrastructure \indexPublicKeyInfrastructure
\begin{itemize}
  \item Alice und Bob besitzen bereits einen alten Schlüssel $\key'$ aus einem früheren Austausch und möchten ein frisches $\key$ erzeugen
  \item es existierte eine Secret-Key-Infrastruktur mit einer Schlüsselzentrale (Alice besitzt einen Schlüssel $\key_A$, Bob $\key_B$ und
  die Schlüsselzentrale beide)
  \item es existiert eine Public-Key-Infrastruktur ($\pkey_A, \pkey_B$ sind öffentlich, Alice besitzt $\skey_A$, Bob besitzt $\skey_B$)
  \item Alice und Bob besitzen ein gemeinsames Passwort
  \item Alice und Bob besitzen keine gemeinsamen Informationen
\end{itemize}


\section{Symmetrische Verfahren}
Als Grundszenario für symmetrische Verfahren wird hier ein System mit einer Secret-Key-Infrastruktur \indexSecretKeyInfrastructure gewählt. Das bedeutet, dass jeder
Teilnehmer einen geheimen, symmetrischen Schlüssel mit der Schlüsselzentrale hat. Jeder Verbindungsaufbau mit einem
anderen Teilnehmer beginnt deshalb mit einer Anfrage an die Zentrale. Da die Zentrale die Anlaufstelle für viele Teilnehmer ist, sollte die Kommunikation mit
dieser Stelle möglichst minimiert werden, was die vollständige Kommunikation der beiden Teilnehmer Alice und Bob über die Zentrale ausschließt. Gleichzeitig sind jedoch die Leitungen
nicht vertrauenswürdig, sodass die Kommunikation über große Strecken verschlüsselt stattfinden sollte.

\subsection{Kerberos}
Eine Lösung für dieses Szenario bietet das Protokoll \emph{Kerberos} \indexKerberos an, das in Abbildung \ref{fig:keyex:kerberos} in seiner ursprünglichen
Form dargestellt ist. Alice sendet dabei der Schlüsselzentrale eine Anfrage, die ihren Namen und den ihres gewünschten Gesprächspartners
erhält und bekommt dafür von der Zentrale zwei Pakete zurück, von denen eines mit ihrem und eins mit Bobs Schlüssel verschlüsselt ist. Beide
Pakete erhalten den gemeinsamen Sitzungsschlüssel $K$, sowie die Lebensdauer $L$ des Schlüssels und einen Zeitstempel $T_{KC}$ der Schlüsselzentrale, der
Replay-Attacken erschwert.
Alice entpackt das an sie adressierte Paket, erhält den Sitzungsschlüssel und leitet nach Prüfung von $L$ und $T$ das für Bob vorbereitete Paket weiter. Sie
fügt außerdem eine mit $K$ verschlüsselte Nachricht bei, in der sie ihre Identität und einen von ihr erstellten Zeitstempel $T_A$ einfügt.

Bob überprüft seinerseits den Zeitstempel der Zentrale und die Lebensdauer des Sitzungsschlüssels und dechiffriert dann Alices Nachricht mit dem neuen
Sitzungsschlüssel. Er kann nun sowohl den Zeitstempel überprüfen als auch, ob die Anfrage an die Schlüsselzentrale vom selben Teilnehmer stammt wie die mit dem
Sitzungsschlüssel chiffrierte Nachricht. Außerdem kann er bei erfolgreicher Entschlüsselung sicher sein, dass Alice $K$ besitzt. Er sendet nun seinerseits eine
mit $K$ verschlüsselte Nachricht an Alice, mit der er nachweist, dass er den Sitzungsschlüssel besitzt. Mit der Erhöhung des Zeitstempels kann er außerdem
beweisen, dass er die korrekte Nachricht erhalten und dechiffriert hat.

\begin{figure}[h]
\begin{center}
\unitlength=1mm
\linethickness{0.4pt}
\hspace{-3 cm}
	\begin{picture}(120,60)(-10,0)
		\put(0,53){\makebox(0,0)[cb]{\texttt{Alice}$_{\key_A}$}}
		\put(100,53){\makebox(0,0)[cb]{\texttt{Schlüsselzentrale (KC)$_{\key_A, \key_B}$}}}
		\put(120,28){\makebox(0,0)[cb]{\texttt{Bob$_{\key_B}$}}}
	
		\put(0,2){\line(0,1){50}}
		\put(100,30){\line(0,1){22}}
		\put(120,2){\line(0,1){25}}
		
		\put(50,46){\makebox(0,0)[cb]{(Alice, Bob)}}
		\put(0,45){\vector(1,0){100}}
	
		\put(50,36){\makebox(0,0)[cb]{$\enc(\key_A,(T_{KC}, L, \key, \text{Bob}))$, $\enc(\key_B, (T_{KC}, L, \key, \text{Alice}))$}}
		\put(100,35){\vector(-1,0){100}}
		
		\put(60,20){\makebox(0,0)[cb]{$\enc(\key, (\text{Alice}, T_A))$, $\enc(\key_B, (T_{KC}, L, \key, \text{Alice}))$}}
		\put(0,19){\vector(1,0){120}}
		
		\put(60,10){\makebox(0,0)[cb]{$\enc(K, T_A+1)$}}
		\put(120,9){\vector(-1,0){120}}
	
	\end{picture}
\end{center}
\caption{Ursprüngliches Schlüsselaustauschprotokoll Kerberos. $T_X$ bezeichnet einen von $X$ ausgestellten Zeitstempel, $\key$ den
erzeugten Sitzungsschlüssel für Alice und Bob und $L$ seine Lebensdauer.}
\label{fig:keyex:kerberos}
\end{figure}

Die verschachtelte Konstruktion von Kerberos verhindert Man-in-the-Middle-Angriffe. Die Kodierung der Absender- und Empfängernamen durch die
Schlüsselzentrale ermöglicht eine Authentifizierung der Kommunikationsteilnehmer und der Einsatz von Zeitstempeln sowie die Zuordnung
einer Lebensdauer zu einem Schlüssel erschwert zudem Replay-Attacken.
Nichtsdestotrotz ist für das Protokoll ein aktiv sicheres Verschlüsselungsverfahren nötig. Über die Sicherheit von Kerberos lässt sich also
formal keine Aussage treffen.

\section{Asymmetrische Verfahren}
Als Grundlage für die folgenden Schlüsselaustauschprotokolle nutzen wir eine Public-Key Infrastruktur. Die Schlüssel werden wie in Kapitel
\ref{ch:asymmenc} von den Teilnehmern selbst erzeugt. Jeder hält also seinen privaten Schlüssel geheim. Die öffentlichen Schlüssel
hinterliegen an einem allgemein zugänglichen Ort und sind von einer vertrauenswürdigen Stelle zertifiziert.

\subsection{Public-Key Transport}
Das einfachste Verfahren, das sich zum Schlüsselaustausch in Public-Key-Infrastruktur \indexPublicKeyInfrastructure anbietet, nennt sich \emph{Public-Key Transport}\indexPublicKeyTransport. Alice erzeugt einen
Sitzungsschlüssel, den sie für die Kommunikation mit Bob verwenden will. Die bereits bestehende Infrastruktur wird nun dafür
genutzt, den Sitzungsschlüssel mit Bobs öffentlichem Schlüssel zu chiffrieren und an Bob zu senden (siehe Abb.
\ref{fig:keyex:publickeytransport}).

\begin{figure}[h]
\begin{center}
\unitlength=1mm
\linethickness{0.4pt}
\hspace{-3 cm}
\begin{picture}(30,10)
\put(0,2){\makebox(0,0)[cb]{$\text{Alice}_{\skey_A}$}}
\put(10,3){\vector(1,0){40}}
\put(30,4){\makebox(0,0)[cb]{$C := \enc(\pkey_B, \key)$}}
\put(55,0.5){\makebox(10,5){$\text{Bob}_{\skey_B}$}}
\end{picture}
\end{center}
\caption{Während des Protokolls Public-Key Transport wählt Alice einen Sitzungsschlüssel $\key$ und sendet ihn unter Ausnutzung der zur
Verfügung stehenden Public-Key-Infrastruktur an Bob.}
\label{fig:keyex:publickeytransport}
\end{figure}

Vorausgesetzt, das verwendete Public-Key-Verfahren ist IND-CPA-sicher, kann der Angreifer $\ciphert$ nicht von Zufall unterscheiden oder den darin
enthaltenen Sitzungsschlüssel extrahieren. Public-Key Transport ermöglicht also passive Sicherheit gegenüber einem Angreifer, der $\ciphert$ auf der Leitung
mithören kann.

Allerdings bietet das Verfahren in dieser Form keine Möglichkeit zur Authentifizierung der Kommunikationsteilnehmer an. Das lässt sich durch
das Hinzufügen von Signaturen wie in Abbildung \ref{fig:keyex:publickeytransportauth} lösen. Trotzdem ist es dann noch immer möglich, einen
Replay-Angriff durchzuführen und $\ciphert$ zu einem späteren Zeitpunkt noch einmal zu senden, ohne dass Bob der Fehler sofort auffällt.

\begin{figure}[h]
\begin{center}
\unitlength=1mm
\linethickness{0.4pt}
\hspace{-3 cm}
\begin{picture}(120,10)(-15,0)
\put(0,0){\makebox(0,0)[cb]{$\text{Alice}_{\skey_{\text{PKE,}A}, \skey_{\sig, A}}$}}
\put(16,3){\vector(1,0){80}}
\put(55,4){\makebox(0,0)[cb]{$(C := \enc(\pkey_{\text{PKE,}B}, \key),$}}
\put(55,-2){\makebox(0,0)[cb]{$\sigma := \sig(\skey_{\sig, A}, C))$}}
\put(110,0){\makebox(10,5){$\text{Bob}_{\skey_{\text{PKE,}B}, \skey_{\sig, B}}$}}
\end{picture}
\end{center}
\caption{Digitale Signaturen ermöglichen den Ausbau des Protokolls Public-Key Transport auf die Authentifikation der Teilnehmer.}
\label{fig:keyex:publickeytransportauth}
\end{figure}

\subsection{Diffie-Hellman-Schlüsselaustausch}
Der Diffie-Hellman-Schlüsselaustausch \indexDiffieHellmanKeyExchange (1976) hat auf den ersten Blick Ähnlichkeit mit dem asymmetrischen Verschlüsselungsverfahren von
ElGamal (1985). Auch hier benötigen wir eine ausreichend große, zyklische Gruppe $\G = \langle g \rangle$ mit Ordnung $q$. Alice und Bob wählen sich
jeweils eine Zufallszahl $x, y \in \mathbbm{Z}_q$ und schicken $g^x$ bzw. $g^y$ an den jeweils anderen. Jeder von beiden ist nun in der Lage, $g^{xy}$
zu berechnen.

Das Berechnen des gemeinsamen Geheimnisses $g^{xy}$ als Außenstehender bezeichnet man als \emph{computational Diffie-Hellman}-Problem (CDH-Problem)\indexComputationalDiffieHellmanProblem.
Dabei hat ein Angreifer Zugriff auf das Generatorelement und die beiden Zahlen $g^{x}$, $g^{y}$. Die Sicherheit des Verfahrens beruht auf der sogenannten
\emph{computational Diffie-Hellman}-Annahme (CDH-Annahme)\indexComputationalDiffieHellmanAssumption, die besagt, dass das Lösen des CDH-Problems in manchen zyklischen Gruppen schwer ist.
Aktive Angriffe, wie Replay- oder Man-in-the-Middle-Attacken, sind damit allerdings nicht ausgeschlossen.

\begin{figure}[h]
	\begin{center}
		\unitlength=1mm
		\linethickness{0.4pt}
		\hspace{-3 cm}
		\begin{picture}(80,50)(-10,0)
			\put(10,42){\makebox(0,0)[cb]{\texttt{Alice}$_x$}}
			\put(10,10){\line(0,1){30}}
	
			\put(70,42){\makebox(0,0)[cb]{\texttt{Bob}$_y$}}
			\put(70,10){\line(0,1){30}}
		
			\put(40,31){\makebox(0,0)[cb]{$g^x$}}
			\put(10,30){\vector(1,0){60}}
		
			\put(40,21){\makebox(0,0)[cb]{$g^y$}}
			\put(70,20){\vector(-1,0){60}}
	
			\put(40,0){\makebox(0,0)[cb]{$(g^y)^x \qquad = \qquad g^{xy} \qquad = \qquad (g^x)^y$}}
		\end{picture}
	\end{center}
	\caption{Diffie-Hellman-Schlüsselaustausch}
	\label{fig:keyex:dh}
\end{figure}

\section{Transport Layer Security (TLS)}\indexTLS
\label{sec:keyexchange:tls}
Dieses Kapitel befasst sich mit einem Protokoll zum Schlüsselaustausch zweier einander unbekannter Kommunikationspartner.
Die klassische Motivation hierfür sind Einkäufe mit der Kreditkarte. Dabei ist es nicht ausschließlich Alices Sorge, dass die Daten unterwegs
abgefangen und für andere Käufe verwendet werden könnten. Ein Angreifer könnte außerdem die Kaufsumme ihres Auftrags
manipulieren oder sich für den Server ausgeben, mit dem Alice kommunizieren möchte und dem sie ihre Kreditkartendaten überträgt.

Dieses Problem, das gleichzeitig den Schlüsselaustausch, wie auch die Authentifikation der Kommunikationspartner umfasst, beschränkt sich allerdings nicht auf
den Interneteinkauf über \emph{http} sondern auch auf andere Anwendungsprotokolle wie \emph{ftp} zur Übertragung von Dateien und \emph{imap} und \emph{smtp},
denen Alice ihre E-Mail-Passwörter anvertraut.

Kurz gefasst benötigt Alice also ein Protokoll, das die Integrität der übertragenen Daten sowie die Authentifikation des Senders bzw. Empfängers implementiert
und einen sicheren Schlüsselaustausch zur Verfügung stellt. Gleichzeitig sollte es möglichst viele Anwendungsprotokolle abdecken, damit nicht jedes einzeln
abgesichert werden muss.

Zu diesem Zweck wurde SSL (\emph{Secure Socket Layer}) entwickelt und in 1999 mit einigen Änderungen als TLS (\emph{Transport Layer Security}) standardisiert.
TLS ist ein hybrides Protokoll\indexEncryptionHybrid zum Aufbau und Betrieb sicherer Kanäle über ein eigentlich unsicheres Medium, einschließlich eines Schlüsselaustauschs.
Dafür wird erst ein authentifizierter asymmetrischer Schlüsselaustausch durchgeführt und danach mit diesem ausgehandelten Schlüssel symmetrisch
verschlüsselt kommuniziert. Es ist sogar möglich, einen Schlüssel neu auszuhandeln, falls der Verdacht besteht, dass er kompromittiert
ist. Außerdem bietet TLS eine ganze Reihe an Schlüsselaustausch- und Verschlüsselungsalgorithmen an, auf die die beiden Parteien sich
einigen können.

% Transportschicht sollte kurz erklärt werden
Dadurch, dass TLS auf der Transportschicht\footnote{Die Transportschicht \indexTransportSchicht ist die 4. Schicht des OSI-Modells, eine in Schichten gegliederte Architektur für Netzwerkprotokolle. Auf der 4. Schicht sind die bekannten Transportprotokolle TCP und UDP angesiedelt.} verschlüsselt, ist es vergleichsweise einfach, Anwendungsprotokolle wie \emph{http}, \emph{smtp} oder \emph{ftp}
darauf anzupassen.

\subsection{TLS-Handshake}\indexTLSHandshake
Der für das Schlüsselaustauschproblem interessante Teil von TLS besteht aus einem Handshake, der vereinfacht in Abbildung \ref{fig:keyex:tls-handshake}
dargestellt ist.
Dafür signalisiert der Client dem Server, dass er den Aufbau eines verschlüsselten Kanals wünscht (\emph{client\_hello}). Er liefert
dem Server eine Zufallszahl $R_C$ sowie eine nach seiner Präferenz sortierte Liste von Algorithmen (Hashfunktionen, symmetrische
Verschlüsselungsverfahren und Schlüsselaustauschprotokolle). Der Server generiert seinerseits eine Zufallszahl $R_S$, wählt einen Satz
Algorithmen aus der Liste des Clients aus und schickt diese zurück (\emph{server\_hello}). Im Folgenden werden die vom Server ausgewählten
Verfahren verwendet.

Im nächsten Schritt schickt der Server dem Client seinen öffentlichen Schlüssel $pk_S$, sowie das dazugehörige Zertifikat, damit der Client die Identität seines
Gesprächspartners überprüfen kann. Haben sich Client und Server auf beidseitige Authentifikation geeinigt, fordert der Server außerdem das Zertifikat des
Clients an.
Wie genau diese Authentifizierung abläuft, wurde im vorigen Schritt durch die Auswahl der entsprechenden Algorithmen festgelegt. Der Client antwortet mit
seinem Zertifikat und seinem öffentlichen Schlüssel $pk_C$. Um die Integrität der bisherigen Kommunikation sicherzustellen, berechnet der Client außerdem den Hashwert $H$
der bisher ausgetauschten Nachrichten und signiert diesen mit seinem privaten Schlüssel. Der Server prüft das Zertifikat, die Signatur und den Hashwert.

Nun berechnet der Client eine weitere Zufallszahl, das so genannte \emph{premaster secret} (PMS)\indexTLSPreMasterSecret, und schickt es verschlüsselt mit dem
zertifizierten öffentlichen Schlüssel an den Server. Beide Teilnehmer besitzen nun einen selbstgewählten Zufallswert sowie einen des
Kommunikationspartners und das premaster secret. Aus diesen drei Zufallszahlen berechnen Client und Server nun mithilfe eines öffentlich bekannten
Algorithmus den \emph{master key} (MS)\indexTLSMasterKey, aus dem wiederum die für die Kommunikation verwendeten session keys abgeleitet werden.
Für die Berechnung der jeweiligen Schlüssel werden Funktionen verwendet, die pseudozufällige Ergebnisse liefern.

Im letzten Teil des Handshakes signalisiert der Client, dass er nun verschlüsselt kommunizieren wird (\emph{ChangeCipherSpec}) und dass damit der
Handshake beendet ist (\emph{Finished}). Der Server antwortet analog. Beide verwenden für die fortlaufende Kommunikation den vereinbarten
Verschlüsselungsalgorithmus und den gemeinsamen session key.
\begin{figure}[h]
\begin{center}
\unitlength=1mm
\linethickness{0.4pt}
\hspace{-3 cm}
	\begin{picture}(120,150)(-10,0)
		\put(20,143){\makebox(0,0)[cb]{\texttt{Client}$_{\skey_C, \pkey_C}$}}
		\put(100,143){\makebox(0,0)[cb]{\texttt{Server}$_{\skey_S, \pkey_S}$}}
	
		\put(20,2){\line(0,1){140}}
		\put(100,2){\line(0,1){140}}
		
		\put(5,131){\makebox(0,0)[cb]{berechne}}
		\put(5,127){\makebox(0,0)[cb]{Zufallszahl $R_C$}}
		\put(60,123){\makebox(0,0)[cb]{\emph{client\_hello}(Liste$_{Algorithmen}$, $R_C$)}}
		\put(20,122){\vector(1,0){80}}
	
		\put(117,118){\makebox(0,0)[cb]{berechne}}
		\put(117,114){\makebox(0,0)[cb]{Zufallszahl $R_S$}}
		\put(60,110){\makebox(0,0)[cb]{\emph{server\_hello}(Auswahl$_{Algorithmen}$, $R_S$)}}
		\put(100,109){\vector(-1,0){80}}
		
		\put(60,100){\makebox(0,0)[cb]{Server-Zertifikat (inkl. $\pkey_S$)}}
		\put(100,99){\vector(-1,0){80}}
		\put(60,94){\makebox(0,0)[cb]{Anfrage Client-Zertifikat}}
		\put(100,93){\vector(-1,0){80}}
		
		\put(3,87){\makebox(0,0)[cb]{überprüfe}}
		\put(3,84){\makebox(0,0)[cb]{Server-Zertifikat}}
		
		\put(60,80){\makebox(0,0)[cb]{Client-Zertifikat (inkl. $\pkey_C$)}}
		\put(20,79){\vector(1,0){80}}
	
		\put(3,73){\makebox(0,0)[cb]{berechne Hash $H$}}
		\put(3,69){\makebox(0,0)[cb]{aller bisherigen}}
		\put(3,66){\makebox(0,0)[cb]{Nachrichten}}
		\put(60,62){\makebox(0,0)[cb]{$\sig(\skey_C, H)$}}
		\put(20,61){\vector(1,0){80}}
		
		\put(117,55){\makebox(0,0)[cb]{überprüfe $H$}}
		\put(117,51){\makebox(0,0)[cb]{und Signatur}}
		
		\put(3,55){\makebox(0,0)[cb]{berechne}}
		\put(3,51){\makebox(0,0)[cb]{Zufallszahl \emph{PMS}}}
		\put(60,47){\makebox(0,0)[cb]{$\enc(\pkey_S, \textit{PMS})$}}
		\put(20,46){\vector(1,0){80}}
		
		\put(3,40){\makebox(0,0)[cb]{berechne \emph{MS}}}
		\put(3,36){\makebox(0,0)[cb]{aus $R_C$, $R_S$, \emph{PMS}}}
		
		\put(117,40){\makebox(0,0)[cb]{berechne \emph{MS}}}
		\put(117,36){\makebox(0,0)[cb]{aus $R_C$, $R_S$, \emph{PMS}}}
		
		\put(60,32){\makebox(0,0)[cb]{\emph{ChangeCipherSpec}}}
		\put(20,31){\vector(1,0){80}}
		
		\put(60,26){\makebox(0,0)[cb]{\emph{Finished}}}
		\put(20,25){\vector(1,0){80}}
	
		\put(60,16){\makebox(0,0)[cb]{\emph{ChangeCipherSpec}}}
		\put(100,15){\vector(-1,0){80}}
		
		\put(60,10){\makebox(0,0)[cb]{\emph{Finished}}}
		\put(100,9){\vector(-1,0){80}}
	
	\end{picture}
\end{center}
\caption{Vereinfachter Ablauf eines SSL/TLS-Handshakes mit beidseitiger Authentifikation.}
\label{fig:keyex:tls-handshake}
\end{figure}


\subsection{Angriffe auf TLS}
Unter Verwendung einer idealen Verschlüsselung, also im idealen Modell, ist TLS sicher. Auch in der Praxis gilt die Sicherheit von TLS
in der neuesten Version und Verwendung der richtigen Parameter und Algorithmen als etabliert. Allerdings mussten konkrete Implementierungen
als Reaktion auf veröffentlichte Angriffe immer wieder gepatcht werden und es existieren einige Angriffe auf bestimmte Varianten und
Kombinationen von eingesetzten Algorithmen, von denen im Folgenden einige erklärt werden.

\subsubsection{\texttt{ChangeCipherSpec} Drop}
Dieser Angriff entstammt dem Jahr 1996 und richtet sich gegen SSL unter Version 3.0, also gegen das Protokoll \emph{vor} seiner Standardisierung als TLS.
\begin{description}
	\item[Beobachtung:] Server und Client tauschen zu Beginn ihrer Kommunikation eine Reihe unverschlüsselter Nachrichten aus (öffentliche Schlüssel,
	Präferenzen für verwendete Algorithmen, Details der Authentifikation \ldots), die es einem Angreifer erlauben, den Status des
	Schlüsselaustauschs zu erkennen. Kurz vor Ende des Handshakes sendet der Client, ebenfalls im Klartext, \emph{ChangeCipherSpec}, um auf
	verschlüsselte Kommunikation umzuschalten.
	\item[Angriff:] Ein aktiver Angreifer unterdrückt den \emph{ChangeCipherSpec}-Hinweis des Clients.
	\item[Konsequenz:] Falls der Server sofort danach Nutzdaten sendet, werden diese nicht verschlüsselt und können vom Angreifer von der Leitung
	gelesen werden.
	\item[Gegenmaßnahme:] Bevor die Nutzdaten gesendet werden, muss der Server auf die Bestätigung des Clients warten.
\end{description}

\subsubsection{Beispielangriff auf RSA-Padding}
1998 wurde ein Angriff auf das RSA-Padding bekannt, der bei entsprechender Algorithmenwahl in SSL ausgenutzt werden kann, um Einblick
in den für die gemeinsame Kommunikation verwendeten Schlüssel zu erlangen.
\begin{description}
	\item[Beobachtung:] Die von SSL eingesetzte Variante von RSA verwendet beim Transport des Master Keys \indexTLSMasterKey "`naives"' Padding:
		\begin{align*}
		C = \enc(\pkey, \text{pad}(M)) = (\text{pad}(M))^e \mod N
		\end{align*}
		Dabei kann durch homomorphe Veränderungen des Chiffrats $C$ und ständige Überprüfung, ob $C$ noch immer gültig ist, auf die
		Beschaffenheit von $M$ geschlossen werden.
	\item[Angriff:] Eine vereinfachte Darstellung des zu übertragenden Schlüssels $K$ ist:
		\begin{align*}
		C = \text{pad}(K)^e = (0\mathrm{x}0002 \parallel \mathtt{rnd} \parallel 0\mathrm{x}00 \parallel K)^e \mod N
		\end{align*}
		Klar ist, dass $K$ vergleichsweise kurz sein und deshalb mit vielen Nullbits beginnen muss. Ziel ist es nun, möglichst viele gültige Faktoren
		$\alpha_i$ zu finden, sodass 
		\begin{align*}
		M_i := \alpha_i \cdot (0\mathrm{x}0002 \parallel \mathtt{rnd} \parallel 0\mathrm{x}00 \parallel K)^e \mod N = \dec(\alpha_i^e \cdot C \mod
		N)
		\end{align*}
		gültig ist. Die Gültigkeit wird festgestellt, indem die $M_i$ zur Überprüfung an den Server weitergeleitet werden. Der Server gibt in
		älteren SSL-Versionen Hinweise, wenn das Padding fehlerhaft ist.
	\item[Konsequenz:] Viele gültige $M_i$ liefern ein grobes Intervall, in dem $K$ liegt.
	\item[Gegenmaßnahme:] Wähle $K$ zufällig, wenn das Padding ungültig ist. (Zu diesem Zeitpunkt stand eigentlich bereits RSA-OAEP zur Verfügung.)
\end{description}
Aus diesem Angriff geht das Theorem von Håstad und Näslund hervor, das besagt, dass jedes Bit von RSA \emph{hardcore} ist.~\\
\begin{theorem}[Håstad und Näslund]
Sei N, e, d wie bei RSA, $M^* \in \mathbbm{Z}_N$ und $i \in \{1, \ldots , \lfloor \log_2(N)\rfloor \}$ beliebig. Sei $\mathcal{O}$ ein
Orakel, das bei Eingabe $C$ das i-te Bit von $M = C^d \mod N$ ausgibt. Dann existiert ein (von N, e, d unabhängiger)
Polynomialzeit-Algorithmus, der bei Eingabe N, e, i und $C^* := (M^*)^e \mod N$ und mit $\mathcal{O}$-Zugriff $M^*$ berechnet.
\end{theorem}

\subsubsection{CRIME}
\label{sec:keyex:crime}
Dieser Angriff aus 2002 (Aktualisierung in 2012) funktioniert bei eingeschalteter Kompression.
\begin{description}
	\item[Beobachtung:] Bei eingeschalteter Kompression wird nicht mehr $M$ sondern \emph{comp}$(M)$ übertragen. TLS verwendet
	\emph{DEFLATE}-Kompression. Bereits einmal aufgetretene Muster werden also nach dem Prinzip \emph{comp}\texttt{(Fliegen fliegen) = Fliegen
	f(-8,6)} wiederverwendet.
	\item[Angriff:] Ein Angreifer kann über die Länge des Chiffrats feststellen, ob im nachfolgenden (unbekannten) Teil des Klartextes Kompression
	verwendet wurde, indem er einen vorangegangenen Teil manipuliert. Die Länge des Chiffrats sinkt dann und der Angreifer weiß, dass zumindest ein Teil seines
	selbst eingefügten Textes im Rest des Chiffrats vorgekommen sein muss.
	
	Konkret kann sich ein Angreifer, der in der Lage ist, einem Client einen Teil seiner Kommunikation mit dem Server zu diktieren, Stück für
	Stück dem von ihm gesuchten Klartext nähern. Wenn er beispielsweise den Session-Cookie des Clients (mit dem geheimen Inhalt
	\texttt{ABCD}) stehlen möchte, so kann er (z.B. über Schadcode) dem Client eine Eingabe (z.B. \texttt{WXYZ}) diktieren, die dieser vor dem
	Verschlüsseln der Nachricht hinzufügt. Er komprimiert und verschlüsselt also nicht mehr nur \texttt{ABCD} sondern \texttt{WXYZABCD}. Aus
	dem belauschten Chiffrat $C := \enc(K, \mathit{comp}(\mathtt{WXYZABCD}))$ kann der Angreifer die Länge von
	$\mathit{comp}(\mathtt{WXYZABCD})$ extrahieren und so den Abstand seines eingeschleusten Textstücks \texttt{WXYZ} zu dem vom Client geheim
	gehaltenen Cookie bestimmen. 
	\item[Konsequenz:] Mit mehreren Wiederholungen kann der Angreifer den Inhalt des Cookies immer weiter einschränken und ihn schließlich
	rekonstruieren.
	\item[Gegenmaßnahme:] Keine Kompression verwenden.
\end{description}

\subsubsection{Fazit}
TLS \indexTLS ist ein historisch gewachsenes Protokoll mit hoher Relevanz. Allerdings bietet es durch die hohe Anzahl an Versionen und
Einstellungsmöglichkeiten auch eine große Angriffsfläche, die häufiger durch Fixes als durch Einführung sichererer Algorithmen reduziert
wird. Dazu kommt, dass von vielen Browsern ausschließlich der TLS-Standard von 1999 unterstützt wird, was zwar Schwierigkeiten in der
Kompatibilität mit anderen Systemen umgeht, aber auch dazu führt, dass einige bereits bekannte Ansatzpunkte für Angriffe
noch immer flächendeckend bestehen.

\section{Weitere Protokolle}

\subsection{IPsec}\indexIPsec
IPsec (\emph{Internet Protocol Security}) ist eine Sammlung von Standards, die zur Absicherung eines IP-Netzwerks entworfen wurden. Es setzt demnach nicht wie
TLS auf der Transportschicht sondern auf der Internetschicht des TCP/IP-Protokollstapels auf. Es soll die Schutzziele Vertraulichkeit, Integrität und
Authentizität in IP-Netzwerken sicherstellen. Allerdings liegt der Fokus von IPsec dabei nicht auf dem Schlüsselaustausch, der deshalb vorher getrennt
stattfinden muss (aktuell durch IKE). Stattdessen bietet IPsec Maßnahmen zur Integritätssicherung der Daten an (u.A. HMAC), soll die Vortäuschung falscher
IP-Adressen (IP-Spoofing) verhindern und bietet verschiedene Modi zur Verschlüsselung von IP-Paketen an.

Obwohl IPsec nicht sonderlich stark verbreitet und nicht sehr gut untersucht ist, haben sich bereits einige Angriffe herauskristallisiert, auf die hier jedoch
nicht näher eingegangen wird.


\subsection{Password Authentication Key Exchange (PAKE)}\indexPAKE
Dieses Protokoll basiert auf der Annahme, dass Alice und Bob, die miteinander kommunizieren wollen, ein gemeinsames Geheimnis \texttt{passwort} besitzen. Über
dieses Passwort wollen sie einander authentifizieren und einen Schlüssel für ihre Kommunikation errechnen. Natürlich kann ein Angreifer trotz allem noch eine
vollständige Suche über die möglichen Passwörter durchführen, es sollte ihm jedoch nicht möglich sein, schneller ans Ziel zu kommen.

Es handelt sich dabei eher um ein grundlegendes Prinzip als um ein feststehendes Protokoll. Bei der Konstruktion eines PAKE ist darauf zu achten, dass die
simpelste Variante, das Senden von $\enc(\mathtt{passwort}, \key)$ keine forward-secrecy bietet. Das bedeutet, wenn im Nachhinein ein Angreifer das Passwort
eines Teilnehmers knackt, ist er nicht nur zukünftig in der Lage, dessen Identität zu simulieren sondern kann außerdem sämtliche vergangene Kommunikation
nachvollziehen.

Eine funktionierende Konstruktion ist \emph{Encrypted Key Exchange} (EKE), bei dem zunächst $\enc(\mathtt{passwort},\pkey)$ gesendet und infolgedessen
asymmetrisch kommuniziert wird. Bei \emph{Simple Password Exponential Key Exchange} wird ein Diffie-Hellman-Schlüsselaustausch auf der Basis von einem nur den
Teilnehmern bekannten $g = H(\mathtt{passwort})^2$ durchgeführt. Der beweisbare PAKE von Goldreich-Lindell nutzt Zero-Knowledge, um die Teilnehmer zu
authentifizieren, ohne das dafür nötige Geheimnis aufzudecken.

PAKE wird z.B. als Basis für EAP (\emph{Extensible Authentication Protocol}) in WPA verwendet und ist formal modellierbar und seine Sicherheit unter bestimmten
theoretischen Annahmen beweisbar.
