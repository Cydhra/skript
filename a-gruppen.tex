\chapter{Gruppen}
Gruppen sind algebraische Strukturen, die in vielen kryptographischen
Verfahren Anwendung finden. Sie werden in der Grundlagenvorlesung
\glqq Lineare Algebra\grqq~ausführlich behandelt. Hier sollen aber noch einmal
einige grundlegende Dinge wiederholt werden, die für die Kryptographie
und die Vorlesung \glqq Sicherheit\grqq~wichtig sind. 

\textbf{Anmerkung:} Die hier dargestellten Zusammenhänge sollen nur
einen groben Überblick über die Gruppentheorie geben. Insbesondere
werden wir die genannten Sätze nicht beweisen. In der Kryptographie
werden auch noch weitere Eigenschaften von Gruppen ausgenutzt, wir
beschränken uns hier bewusst nur auf die für die Vorlesung \glqq Sicherheit\grqq{}
relevanten Begriffe und Aussagen.

Gruppen sind eine Abstraktion, mit der man unter anderem übliche
Rechenregeln verallgemeinern kann. Eine Gruppe besteht aus einer Menge
(z.B. von Zahlen) und einer Verknüpfung. Diese Verknüpfung bildet zwei
Elemente der Menge auf ein Mengenelement ab. Dass die Verknüpfung nur
auf Gruppenelemente abbildet, wird als \emph{Abgeschlossenheit}
bezeichnet.

Die Verknüpfung hat auf der Menge ein \emph{neutrales
  Element}. Verknüpft man ein Element $x$ aus der Gruppe mit diesem
Element, so erhält man wieder $x$. Bei einer Gruppe, die die Addition
als Verknüpfung hat, ist dies beispielsweise die $0$.

Für jedes Element $x$ enthält eine Gruppe ein \emph{inverses Element}
$x^{-1}$. Verknüpft man $x$ und $x^{-1}$, so erhält man das neutrale
Element. Für eine Gruppe mit Addition ist dies $-x$, bei einer Gruppe
mit Multiplikation $\frac{1}{x}$.

Außerdem ist die Verknüpfung \emph{assoziativ}. Das bedeutet, dass in
verschachtelten Ausdrücken die Reihenfolge, in der man die Verknüpfung
auf die Elemente anwendet, egal ist.

\begin{definition}[Gruppe]
Sei \G~eine Menge und $\cdot$ eine Verknüpfung mit $\cdot: \G\times \G \to
\G$. Dann heißt $(\G, \cdot)$ eine \textit{Gruppe}, wenn die Operation
in der Menge abgeschlossen ist (d.h. dass man durch Ausführen der
Operation auf Elementen der Menge keine Elemente erhält, die nicht in der
Menge liegen), es ein neutrales Element gibt, jedes 
Element ein Inverses hat und die Operation assoziativ ist. Im Folgenden
werden diese Eigenschaften formal definiert: 
\begin{description}
\item[Abgeschlossenheit:] \G~ist abgeschlossen bezüglich $\cdot$, also gilt:
  \[\forall x, y \in \G: (x\cdot y)\in \G\]
\item[neutrales Element:] Es existiert ein \textit{neutrales Element}  $e \in \G$ , sodass für
  alle $x \in \G$ gilt \[x\cdot e=x=e\cdot x.\]
\item[inverses Element:] Jedes Element aus \G~hat ein \textit{inverses Element}, d.h. es gilt \[\forall x \in
  \G~ \exists y \in \G: x\cdot y=e=y\cdot x.\]

\item[Assoziativität:] $\cdot$ ist assoziativ, d.h. es gilt \[\forall x, y, z  \in ~\G:
  (x\cdot y)\cdot z = x\cdot(y \cdot z).\]
\end{description}
\end{definition}

Ein zusätzliche Eigenschaft, die Gruppen erfüllen können, ist die
Kommutativität. Ist eine Gruppe kommutativ, so bedeutet das, dass die
Reihenfolge bei der Verknüpfung von Elementen keine Rolle spielt. In der
Kryptographie verwendet man meistens kommutative Gruppen und in
\glqq Sicherheit\grqq{} beschränken wir uns auch auf diese. 
\begin{definition}[kommuntative Gruppe]
  Eine Gruppe $(\G, \cdot)$ heißt \textit{kommutativ}, wenn gilt:
\[
  \forall x, y \in \G: x \cdot y = y \cdot x
\]
\end{definition}

Als Beispiel wird geprüft, ob die natürlichen Zahlen mit der Addition
als Verknüpfung eine Gruppe bilden:
\begin{beispiel}
  Um zu prüfen, ob \((\Z{}, +)\) eine Gruppe ist, müssen
  wir nur die oben genannten Eigenschaften nachrechnen:
  \begin{itemize}
  \item Für die Abgeschlossenheit muss gelten: 
    \[\forall x, y \in \Z{}: (x+y)\in \Z{}. \] 
     Dies ist erfüllt. Die Summe zweier natürlicher Zahlen ist wieder
     eine natürliche Zahl.
  \item Es muss ein neutrales Element geben. Da gilt 
    \[\forall x \in \Z{}: x+0 = x = 0+x, \] ist $0$ das neutrale Element.
  \item Es muss für alle Elemente ein Inverses geben. Dies ist gegeben
    durch \[\forall x \in \Z{}: x + (-x) = 0.\]
  \item Die Verknüpfung muss assoziativ sein. Es gilt:
    \[\forall x, y, z \in \Z{}: x + (y+z)=(x+y)+z,\] also ist die
    Addition assoziativ.
  \end{itemize}
  Damit ist gezeigt, dass $(\Z{}, +)$ eine Gruppe ist.
\end{beispiel}

Eine Teilmenge von Elementen einer Gruppe, die bzgl. der
Gruppenverknüpfung ebenfalls eine Gruppe ist, nennen wir Untergruppe. In
der Kryptographie rechnet man häufig in Untergruppen. Dies hat
komplexitätstheoretische Gründe, die an dieser Stelle nicht weiter
vertieft werden sollen. 

\begin{definition}[Untergruppe]
  Sei $(\G, \cdot)$ eine Gruppe und $\mathbb{U} \subset \G$. Wenn
  $(\mathbb{U}, \cdot)$ wieder eine Gruppe ist, so nennt man $(\mathbb{U},
  \cdot)$ \textit{Untergruppe} von $(\G, \cdot)$.
\end{definition}

\begin{beispiel}
  Wir haben gesehen, dass $\G := (\Z{}, +)$ eine Gruppe ist. Es sei
  $\Z{g}$ die Menge der geraden Zahlen. Dann ist $\mathbbm{U} := (\Z{g},
  +)$ eine Untergruppe von $\G$:
  \begin{itemize}
  \item Es ist $\mathbb{U} \subset \G$, weil die geraden Zahlen eine
     Teilmenge der natürlichen Zahlen sind.
  \item $\mathbbm{U}$ ist eine Gruppe:
    \begin{description}
    \item[Abgeschlossenheit:] Die Summe zweier gerader Zahlen ist gerade.
    \item[neutrales Element:] Das neutrale Element $0$ verhält sich für
      gerade Zahlen genau wie für ungerade.
    \item[Inverses Element:] Für jede gerade Zahl $x$ ist $-x$ auch
      gerade.
    \item[Assoziativität:] Die Assoziativität gilt für gerade Zahlen
      genauso wie für alle natürlichen Zahlen.
    \end{description}
  \end{itemize}
  Es ist also $\mathbbm{U}$ eine Gruppe und damit Untergruppe von $\G$.
\end{beispiel}

Wird die Verknüpfung einer Gruppe als Multiplikation aufgefasst,
verwendet man häufig eine Exponentenschreibweise, um das mehrfache
Ausführen der Verknüpfung auszudrücken. Diese wird beispielsweise bei den
ElGamal-Verschlüsselungs-Systemen verwendet:

\begin{definition}[Exponentenschreibweise für Gruppen]
 Dabei ist $g\in \G$, $m \in \Z{}$.
\[ g^m = \underbrace{g \cdot \dotsc \cdot g}_{\text{m mal}} \]
Hierbei gelten die gewohnten Rechenregeln für Exponenten: $g^m\cdot g^n =
g^{m+n}$, $(g^m)^n = g^{mn}$
\end{definition}
Ein weiter wichtiger Begriff ist der der Ordnung von Gruppen und
Gruppenelementen. Viele kryptographische Anwendungen in Gruppen machen
direkt oder indirekt Gebrauch von der Ordnung, beispielsweise bei
Korrektheits- oder Sicherheitsbeweisen. 
\begin{definition}[Ordnung einer Gruppe]
  Sein $(\G, \cdot)$ eine Gruppe. Dann heißt die Anzahl $|\G|$ der Elemente in
  \G~die Ordnung der Gruppe.
\end{definition}

\begin{definition}[Ordnung eines Gruppenelements]
  Die Ordung $\operatorname{ord}(g) $ eines Gruppenelements $g$
  bezeichnet die kleinste 
  natürliche Zahl $n>0$, für die $g^n = e$ gilt. Existiert kein solches
  Element, so sagt man, $g$ habe unendliche Ordnung.
 \end{definition}

Für den Korrektheitsbeweiß des RSA-Verschlüsselungsverfahrens wurde der
kleine Satz von Fermat verwendet: 
 \begin{theorem}[Kleiner Satz von Fermat]
   Sei $(\G, \cdot)$ eine Gruppe, $a \in \Z{}$ und $p$ eine
   Primzahl. Dann gilt
   \[a^{p-1} \equiv 1 \mod p\]
 \end{theorem}
\section{Zyklische Gruppen}
Eine zyklische Gruppe hat die Eigenschaft, dass es mindestens einen
Erzeuger (engl. Generator) $g$
gibt. Ein Erzeuger ist ein Gruppenelement, von dem aus man durch
mehrfahches Anwenden der Gruppenoperation auf sich selbst
jedes andere Gruppenelement erreichen kann. Wichtige Beispiele von
zyklischen Gruppen sind Restklassengruppen $(\Z{}/m\Z{}, +)$ oder
elliptische Kurven.

\begin{definition}[Zyklische Gruppe]
  Sei $(\G, \cdot)$ eine Gruppe. Dann heißt $(\G, \cdot)$ zyklisch, wenn
  es ein $g \in \G$ gibt, sodass gilt
  \[
    \forall x \in \G: \exists n \in \Z{}: x=g^n,
  \]
  wobei gilt:
\[
 g^n = 
  \begin{cases} 
   \overbrace{g \cdot \dotsc \cdot g}^{\text{n mal}} & \text{falls } n > 0 \\
   g^e       & \text{falls } n = 0 \\
   \underbrace{g^{-1} \cdot \dotsc \cdot g^{-1}}_{\text{n mal}} &
   \text{falls }n < 0
  \end{cases}
\]

$g$ heißt \textit{Erzeuger} von $(\G, \cdot)$. 
\end{definition}

\begin{theorem}
Jede zyklische Gruppe ist kommutativ.  
\end{theorem}

Eine zyklische Gruppe kann mehrere Erzeuger besitzen. Die Gruppe $(\Z{},
+)$ hatbeispielsweise die zwei Erzeuger $-1$ und $1$. Tatsächlich gibt es auch
Gruppen, bei denen fast jedes Element ein Erzeuger ist: 
\begin{theorem} 
Wenn die Ordung einer zyklischen Gruppe eine Primzahl ist, dann ist
jedes Element, mit Ausnahme des neutralen Elements, ein Erzeuger. 
\end{theorem}

In der Kryptographie hat man es fast ausschließlich mit endlichen
Gruppen zu tun. Es sei aber angemerkt, dass es auch endliche Gruppen
gibt, die nicht zyklisch sind (z.B. die sogenannte \emph{kleinsche
Vierergruppe} oder manche $\Z{}/m\Z{}$, siehe dazu auch Beispiel \ref{bsp:azyklische_gruppe}). Außerdem
gibt es auch unendliche zyklische Gruppen, beispielsweise $(\Z{},+)$.
\section{Die Gruppe $\Z{N}^*$}
Es ist $\Z{N}$ die Menge der Restklassen der Multiplikation modulo $N$
(Umgangssprachlich kann man sich diese als die Menge der Zahlen
${0,\dots , N-1}$ vorstellen). Die Operation $\cdot$ wird als
Multiplikation modulo $N$ aufgefasst, also 
\[
a \cdot b = ab \mod N.
\]
 Offentsichtlich
ist $(\Z{N}, \cdot)$ keine Gruppe, denn $0$ hat kein inverses Element
bezüglich der Multiplikation.
Trotzdem gibt es Gruppen ganzer Zahlen bezüglich der
Multiplikation modulo $N$. Es reicht im Allgemeinen nicht aus, lediglich die $0$ zu
entfernen, damit $(\Z{N}\setminus \{0\}, \cdot)$ eine Gruppe ist. Sei
z.B. $N=6$ und damit $\Z{6}\setminus \{0\} = \{1,2,3,4,5\}$. Damit ist  
\[
2 \cdot 3 = 6 \equiv 0 \notin \Z{6}\setminus \{0\},
\]
also ist $\Z{6}\setminus \{0\}$ nicht abgeschlossen bezüglich $\cdot$. 
Dieses Problem lässt sich lösen, indem man alle weiteren Elemente $x$ aus
der Menge entfernt, für die $\ggt(x, N)\neq 1$ gilt. Wir bezeichnen eine
solche Gruppe mit $\Z{N}^*$. 

Für eine Primzahl $p$ gilt, dass  $|\Z{p}^*|= p-1$, denn für jede Zahl $x\in
{1,..., p-1}$ gilt, dass $\ggt(x, p)=1$. die Gruppe $\Z{p}^*$ enthält
also als Elemente die Äquivalenzklassen der Zahlen von $1$ bis $p-1$.

\begin{beispiel}
\label{bsp:azyklische_gruppe}
Betrachten wir die Gruppe $\G := (\Z{8}, \cdot)$. Zunächst prüfen wir, welche
Elemente die Gruppe hat:
\begin{eqnarray*}
ggT(1, 8) = 1 \rightarrow 1 \in \G\\
ggT(2, 8) = 2 \rightarrow 2 \notin \G\\
ggT(3, 8) = 1 \rightarrow 3 \in \G\\
ggT(4, 8) = 2 \rightarrow 4 \notin \G\\
ggT(5, 8) = 1 \rightarrow 5 \in \G\\
ggT(6, 8) = 2 \rightarrow 6 \notin \G\\
ggT(7, 8) = 1 \rightarrow 7 \in \G\\
\end{eqnarray*}
Wir prüfen nun, ob die Gruppe einen Erzeuger hat. Für einen Erzeuger
$g \in \G$ gilt 
\[
\forall x \in \G \exists n\in\{1,\dots, 7\}: x = g^n.
\]
Es kann $1$ kein Erzeuger sein, denn $1^n=1$. Geprüft werden nun alle
weiteren Elemente:
\begin{eqnarray*}
  3^1 & = & 3\\
  3^2 & = & 9 \stackrel{\text{mod 8}}{\equiv} 1 \Rightarrow 3^n\in\{1, 3\}\\
  5^1 & = & 5\\
  5^2 & = & 25 \stackrel{\text{mod 8}}{\equiv} 1 \Rightarrow 5^n\in\{1, 5\}\\
  7^1 & = & 7\\
  7^2 & = & 49 \stackrel{\text{mod 8}}{\equiv} 1 \Rightarrow 7^n\in\{1, 7\}\\
\end{eqnarray*}
Die Gruppe $\G$ hat also keinen Erzeuger, ist also nicht zyklisch.
\end{beispiel}

\begin{beispiel}
  Analog zu Beispiel \ref{bsp:azyklische_gruppe} prüfen wir, ob $\G :=
  (\Z{7}, \cdot)$ zyklisch ist. $\G$ hat die Elemente $\{1, 2, 3, 4, 5,
  6\}$.

  $1$ ist das neutrale Element und damit kein Erzeuger. Nun werden die
  anderen Elemente daraufhin geprüft, ob sie Erzeuger sind:
\[  \begin{array}[]{ccccl}
    2^1 &= & 2 & &\\
    2^2 &= & 4 & &\\
    2^3 &= & 8 &\stackrel{\text{mod 7}}{\equiv}& 1 \Rightarrow 2^n \in \{1, 2, 4\}\\
\\    
    3^1 &= & 3 & &\\
    3^2 &= & 9  & \stackrel{\text{mod 7}}{\equiv}& 2\\
    3^3 &= & 27 & \stackrel{\text{mod 7}}{\equiv}& 6\\
    3^4 &= & 81 & \stackrel{\text{mod 7}}{\equiv}& 4\\
    3^5 &= & 243& \stackrel{\text{mod 7}}{\equiv}& 5\\
    3^6 &= & 729& \stackrel{\text{mod 7}}{\equiv}& 1 \Rightarrow \langle 3\rangle = \G
   \end{array}\]
  Es ist also $3$ ein Erzeuger. Damit ist die Gruppe zyklisch. Ein
  weiterer Erzeuger ist $5$, während $4$ und $6$ keine Erzeuger sind.

\end{beispiel}