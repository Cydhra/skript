
\chapter{Gruppen}
Gruppen sind algebraische Strukturen, die in vielen kryptographischen
Verfahren Anwendung finden. Sie werden in den Grundlagenvorlesungen zur
Algebra ausführlich behandelt, sollen hier aber noch einmal in
Erinnerung gerufen werden.

\begin{definition}[Gruppe]
Sei \G~eine Menge und $\cdot$ eine Verknüpfung mit $\cdot: \G\times \G \to
\G$. Dann heißt $(G, \cdot)$ eine \textit{Gruppe}, wenn
\begin{itemize}
\item \G~bezüglich $\cdot$ abgeschlossen ist.
\item Ein Element $e \in \G$ existiert, sodass für alle $x \in \G$ gilt
  $x\cdot e=x=e\cdot x$.
\item Jedes Element aus \G~ein Inverses hat, d.h. es gilt $\forall x \in
  \G~ \exists y \in \G: x\cdot y=e=y\cdot x$.
\item $\cdot$ assoziativ ist,  d.h. es gilt $\forall x, y, z  \in ~\G:
  (x\cdot y)\cdot z = x\cdot(y \cdot z)$.
\end{itemize}
\end{definition}
\begin{definition}[Untergruppe]
  Sei $(\G, \cdot)$ eine Gruppe und $\mathbb{U} \subset \G$. Wenn
  $(\mathbb{U}, \cdot)$ wieder eine Gruppe ist, so nennt man $(\mathbb{U},
  \cdot)$ \textit{Untergruppe} von $(\G, \cdot)$.
\end{definition}

\begin{definition}[Exponentenschreibweise für Gruppen]
Wird die Verknüpfung einer Gruppe als Multiplikation aufgefasst,
verwendet man häufig eine Exponentenschreibweise, um das mehrfache
Ausführen der Verknüpfung auszudrücken. Dabei ist $g\in \G$, $m \in \Z{}$.
\[ g^m = \underbrace{g\cdots g}_{\text{m mal}} \]
Hierbei halten die gewohnten Rechenregeln für Exponenten: $g^m\cdot g^n =
g^{m+n}$, $(g^m)^n = g^{mn}$
\end{definition}


\begin{definition}[Ordnung einer Gruppe]
  Sein $(\G, \cdot)$ eine Gruppe. Dann heißt die Anzahl $|\G|$ der Elemente in
  \G~die Ordnung der Gruppe.
\end{definition}

\begin{definition}[Ordnung eines Gruppenelements]
  Die Ordung $\operatorname{ord}(g) $ eines Gruppenelements $g$
  bezeichnet die kleinste 
  natürliche Zahl $n>0$, für die $g^n = e$ gilt. Existiert kein solches
  Element, so sagt man, $g$ habe unendliche Ordnung.
 \end{definition}

 \begin{theorem}[Satz von Lagrange]
   Für jede Untergruppe $(\mathbb{U}, \cdot)$ einer Gruppe $(\G, \cdot)$
   ist $|U|$ ein Teiler von $|G|$.   
 \end{theorem}

 \begin{theorem}[Kleiner fermatscher Satz]
   Sei $(\G, \cdot)$ eine Gruppe, $a \in \Z{}$ und $p$ eine
   Primzahl. Dann gilt
   \[a^p \equiv a \mod p\]
 \end{theorem}
Damit gilt insbesondere auch
\[a^n \equiv a^{(n \mod p)} \mod p\]
\section{zyklische Gruppen}

\subsection{zyklische Gruppen}
\begin{definition}[zyklische Gruppe]
  Sei $(\G, \cdot)$ eine Gruppe. Dann heißt $(\G, \cdot)$ zyklisch, wenn
  es ein $g \in \G$ gibt, sodass gilt
\[
  \forall x \in \G: \exists n \in \Z{}: x=g^n.
\]
$g$ heißt \textit{Erzeuger} von $(\G, \cdot)$.
\end{definition}

\subsection{Die Gruppe $\Z{N}^*$}
Es ist $\Z{N}$ die Menge der Zahlen ${0,\dots , N-1}$. Offentsichtlich
ist $(\Z{N}, \cdot)$ keine Gruppe, denn $0$ hat kein inverses Element
bezüglich der Multiplikation.
Trotzdem gibt Gruppen ganzer Zahlen bezüglich der
Multiplikation. Um Abgeschlossenheit zu erreichen, wird $\cdot$
definiert als Multiplikation modulo $N$, also 
\[a \cdot b = ab \mod N\]. Dies reicht aber im Allgemeinen noch nicht
aus, damit $(\Z{N}\setminus \{0\}, \cdot)$ eine Gruppe ist. Sei z.B. $N=6$ und
damit $\Z{6}\setminus \{0\} = {1,2,3,4,5,}$. Damit ist 
\[
2 \cdot 3 = 6 \mod 6 = 0 \notin \Z{6}\setminus \{0\},
\]
also ist $\Z{6}\setminus \{0\}$ nicht abgeschlossen bezüglich $\cdot$. 
Dieses Problem lässt sich lösen, indem man alle weiteren Elemente $x$ aus
der Menge entfernt, für die gilt $\ggt(x, N)\neq 1$. Wir bezeichnen eine
solche Gruppe mit $\Z{N}^*$. 

Für eine Primzahl $p$ gilt, dass  $|\Z{q}^*|= q-1$, denn jede Zahl $x\in
{1,..., q-1}$ gilt, dass $\ggt(x, q)=1$.