\chapter{Hashfunktionen}\label{cha:hash}
\section{Grundlagen}
Hashfunktionen sind Funktionen, die von einer großen, potentiell unbeschränkten Menge in eine kleinere Menge abbilden, also
\begin{align*}
H_k\colon \{0,1\}^* \rightarrow \{0,1\}^k\, ,
\end{align*}
wobei $k$ den in \ref{sec:secparam} eingeführten Sicherheitsparameter bezeichnet.
Diese Funktionen werden dazu verwendet, größere Datenmengen effizient zu kennzeichnen (ihnen sozusagen einen Fingerabdruck zuzuordnen). Anwendungsgebiete sind
z.B. das Verifizieren der Datenintegrität eines Downloads oder das Signieren von Daten durch einen Abgleich der entsprechenden Hashwerte.

\section{Sicherheitseigenschaften}

Um eine Hashfunktion im kryptographischen Sinne verwenden zu können, reicht eine Funktion, die von einer großen Menge in eine kleine Menge abbildet, nicht aus.
Sie muss zusätzlich einige weitere Anforderungen erfüllen.

\subsection{Kollisionsresistenz}
Die wichtigste Eigenschaft einer Hashfunktion $H$ ist die Kollisionsresistenz (\textit{collision resistance}). Das bedeutet, es soll schwierig sein, zwei
unterschiedliche Urbilder $X, X'$ zu finden, für die gilt: 
\begin{align*}
X \neq X' \text{ und } H(X) = H(X')
\end{align*}

Da wir von einer großen in eine kleine Menge abbilden, kann $H$ nicht injektiv sein. Es ist uns also nicht möglich, Kollisionen komplett zu
verhindern. Trotzdem können wir fordern, dass diese möglichst selten auftreten. Präziser formuliert verlangen wir, dass bei jeder
kollisionsresistenten Hashfunktion ein PPT Algorithmus eine Kollision nur mit vernachlässigbarer Wahrscheinlichkeit findet.
\begin{definition}[Kollisionsresistenz]
Eine Funktion $H_k$ ist kollisionsresistent, wenn jeder PPT-Algorithmus nur mit höchstens vernachlässigbarer Wahrscheinlichkeit eine Kollision findet.
Präziser formuliert ist der Vorteil für jeden PPT-Angreifer $\A$
\begin{align*}
Adv^{cr}_{H,\A}(k):= \Pr \left[ (X, X') \leftarrow \A(1^k) : X \neq X' \land H_k(X) = H_k(X')\right]
\end{align*}
in $k$ vernachlässigbar.
\end{definition}

\subsection{Einwegeigenschaft}
Die zweite kryptographisch wichtige Eigenschaft von Hashfunktionen ist die Einwegeigenschaft (\textit{pre-image resistance}), die sicherstellt,
dass eine Hashfunktion nur in eine Richtung berechenbar ist. Genauer gesagt fordern wir, dass es bei einem gegebenen Wert $H(X)$ \emph{schwierig} ist, ein
passendes $X$ zu finden.

Angewendet wird diese Eigenschaft beispielsweise beim Speichern von Passwörtern auf einem Server. Der Server speichert nur $H(X)$ ab und vergleicht bei
einem Anmeldungsversuch lediglich $H(X)$ mit dem ihm vom Client zugesendeten $H(X')$. Dadurch muss das Passwort nicht im Klartext auf dem Server liegen.
Ebenfalls nützlich ist die Einwegeigenschaft bei der Integritätssicherung von Daten. Wenn die verwendete Hashfunktion die Einwegeigenschaft erfüllt, ist
es schwierig, einen Datensatz so zu verändern, dass der Hashwert des Datensatzes gleich bleibt und die Veränderung sich nicht bemerkbar macht.

Es stellt sich nun die Frage, wie eine Hashfunktion beschaffen sein muss, damit sie die Einwegeigenschaft erfüllen kann. Ist z.B. die Urbildmenge zu klein, kann
durch Raten einfach auf ein passendes $X'$ geschlossen werden. Außerdem sollte es intuitiv keinen Kandidaten $X'$ als Urbild für $H(X)$ geben, der
wahrscheinlicher ist als andere Kandidaten. Um das zu erreichen, wird für die Elemente der Urbildmenge üblicherweise eine Gleichverteilung angestrebt.
\vspace{10pt}

\begin{definition}[Einwegfunktion]
Eine über $k$ parametrisierte Funktion $H$ ist eine Einwegfunktion bezüglich der Urbildverteilung $\chi_k$, wenn jeder PPT-Algorithmus nur mit höchstens
vernachlässigbarer Wahrscheinlichkeit ein Urbild eines gegebenen, aus $\chi_k$ bezogenen Bildes findet. Genauer ist der Vorteil für jeden PPT-Angreifer $\A$
\begin{equation*}
Adv^{ow}_{H,\A}(k):= \Pr \left[ X' \leftarrow \A(H(X), 1^k) : H(X) = H(X')\right]
\end{equation*}
in $k$ vernachlässigbar, wobei $X \leftarrow \chi_k$ gewählt wurde. Dabei muss $\A$ nicht zwingend $X' = X$ zurückgeben.
\end{definition}

~\\
Die Forderungen nach Kollisionsresistenz und Einwegeigenschaft, die wir bisher für eine kryptographische Hashfunktion aufgestellt haben, hängen bei näherer
Betrachtung sehr eng miteinander zusammen. Das führt uns zu folgender Feststellung:
\vspace{10pt}

\begin{theorem}
Jede kollisionsresistente Hashfunktion $H_k \colon \{0,1\}^* \rightarrow \{0,1\}^k$ ist eine Einwegfunktion bzgl. der Gleichverteilung auf $\{0,1\}^{2k}$.
\end{theorem}

\begin{beweisidee}
Bei $X \in \{0,1\}^{2k}$ hat fast jedes Urbild $X$ viele "`Nachbarn"' $X'$ mit $H(X) = H(X')$. Also gilt für die Wahrscheinlichkeit, dass ein Element $H(X)$
der Bildmenge nur ein einziges Urbild $X$ besitzt:

\begin{equation*}
\Pr \left[ | H^{-1}(H(X))| = 1\right] \leq \frac{2^k}{2^{2k}} = \frac{1}{2^k}
\end{equation*}
\end{beweisidee}

\begin{beweis}
Zu jedem $H$-Invertierer $\A$ geben wir nun einen $H$-Kollisionsfinder $\B$ an mit
\begin{equation*}
Adv^{cr}_{H,\B}(k) \geq \frac{1}{2} \cdot Adv^{ow}_{H,\A}(k) - \frac{1}{2^{k+1}}
\end{equation*} 
Nun wählt $\B$ ein $X \leftarrow \{0,1\}^{2k}$ gleichverteilt zufällig und gibt $H(X)$ als Eingabe an $\A$. $\B$ setzt nun $X' \leftarrow \A(1^k, H(X))$ und
gibt $(X, X')$ aus.\\
Dann gilt für $\B$s Erfolgswahrscheinlichkeit:
\begin{align*}
	&\Pr \left[\B \text{ gewinnt}\right]\\
	&= \Pr\left[H(X) = H(X') \land X \not = X'\right]\\
	&= \Pr \left[\A \text{ invertiert} \land X \not = X'\right]\\
	&\geq \Pr \left[\A \text{ invertiert} \land X \not = X' \land \vert H^{-1}(H(X)) \vert > 1\right]\\
	&= \underbrace{\Pr \left[X \not = X' \mid \A \text{ invertiert} \land \vert H^{-1}(H(X)) \vert > 1\right]}_{\geq \frac{1}{2}}
	\cdot \underbrace{\Pr \left[\A \text{ invertiert} \land \vert H^{-1}(H(X)) \vert > 1\right]}_{\geq \Pr \left[\A \text{ invertiert}\right] - \frac{1}{2^k}}\\
	&\geq \frac{1}{2} \cdot Adv^{ow}_{H,\A}(k) - \frac{1}{2^{k+1}}
\end{align*}
\qed
\end{beweis}


\subsection{Target Collision Resistance}
Die \textit{Target Collision Resistance} (auch \textit{second pre-image resistance} oder \textit{universal one-way}) ist eine weitere Eigenschaft, die zur
Bewertung von Hashfunktionen herangezogen wird. Genügt eine Hashfunktion $H$ der Target Collision Resistance, ist es \textit{schwierig}, für ein gegebenes
Urbild $X$ ein $X' \not = X$ zu finden, für das gilt: $H(X') = H(X)$.

Die Target Collision Resistance stellt einen Zwischenschritt zwischen Kollisionsresistenz und Einwegeigenschaft dar: Kollisionsresistenz impliziert die
Target Collision Resistance, welche wiederum die Einwegeigenschaft impliziert. Formal ergibt sich:\\

\begin{definition}[Target Collision Resistance]
Eine über $k$ parametrisierte Funktion $H$ genügt der Target Collision Resistance, falls für jeden PPT-Angreifer $\A$ bei gegebenem X die Wahrscheinlichkeit
\begin{equation*}
Adv^{tcr}_{H,\A}(k):= \Pr \left[ X' \leftarrow \A(X, 1^k) : X \neq X' \land H_k(X) = H_k(X')\right]
\end{equation*}
in $k$ vernachlässigbar ist.
\end{definition}

\begin{beispiel}
Gegeben sei ein Zertifikat\footnote{Ein Public-Key-Zertifikat bindet einen öffentlichen Schlüssel digital an einen Eigentümer und eine Menge an Eigenschaften, wie
zum Beispiel Gültigkeit, Informationen über den Aussteller und Informationen über den Inhaber.} für einen  gehashten Public Key. Für eine Hashfunktion,
die keine Target Collision Resistance garantiert, ist es einfach, einen zweiten Public Key zu finden, der den gleichen Hashwert hat. Da das Zertifikat an den
Hashwert des Schlüssels und nicht an den Schlüssel selbst gekoppelt ist, ist es auch für den zweiten Schlüssel gültig.
\end{beispiel}

\section{Merkle-Damgård-Konstruktion}
\label{ch:hash:merkledamgard}
In der Praxis werden Hashfunktionen benötigt, die nicht nur die Eigenschaften aus den obigen Abschnitten berücksichtigen, sondern auch flexibel in ihrer
Eingabelänge und konstant in ihrer Ausgabelänge sind. Typischerweise werden für diesen Zweck Merkle-Damgård-Konstruktionen eingesetzt. 

\subsection{Struktur von Merkle-Damgård}
Die Eingabenachricht wird bei einer Merkle-Damgård-Konstruktion $H_{MD}$ zunächst in Blöcke $X_0, \ldots , X_n$ mit fester Länge $m$ aufgeteilt (z.B. 512 Bit).
Auf diese Blöcke wird dann nacheinander eine Kompressionsfunktion $F$ angewendet, die die Blöcke mithilfe eines Eingabeparameters $Z$ auf eine festgelegte Länge
$k < m$ verkürzt.

Aus dem ersten Nachrichtenblock $X_0$ und dem Initialisierungsvektor IV der Länge $k$ wird durch die Kompressionsfunktion ein neuer Wert $Z_0$ der Länge $k$
erzeugt. $Z_0$ wird daraufhin gemeinsam mit dem zweiten Block $X_1$ zur Berechnung von $Z_1$ benutzt und so fort. Wenn alle Blöcke $X_0 \ldots X_n$ der
Nachricht abgearbeitet sind, ergibt sich aus $Z_{n}$ der Hashwert. Der Ablauf ist in Abbildung \ref{fig:md-konstruktion} gezeigt.

Der Initialisierungsvektor IV wird dabei für jede Hashfunktion fest gewählt. Ist der letzte Block $X_n$ zu kurz, wird er auf die benötigten $m$ Bits ergänzt.
Das Padding enthält dabei die Nachrichtenlänge, um zu verhindern, dass Verlängerungen der Nachricht für einen Angriff genutzt werden können.
\begin{figure}[h]
\begin{center}
\unitlength=1mm
\linethickness{0.4pt}
\begin{picture}(110,20)

\put(5,5){\vector(1,0){15}}
\put(12,6){\makebox(0,0)[cb]{IV}}

\put(25,15){\vector(0,-1){7.5}}
\put(28,10){\makebox(0,0)[cb]{$X_0$}}

\put(20,2.5){\framebox(10,5){$F$}}

\put(30,5){\vector(1,0){12}}
\put(36,6){\makebox(0,0)[cb]{$Z_0$}}

\put(47,15){\vector(0,-1){7.5}}
\put(50,10){\makebox(0,0)[cb]{$X_1$}}

\put(42,2.5){\framebox(10,5){$F$}}

\put(52,5){\vector(1,0){12}}
\put(58,6){\makebox(0,0)[cb]{$Z_1$}}

\put(68,4){\makebox(0,0)[cb]{\ldots}}

\put(72,5){\vector(1,0){12}}
\put(78,6){\makebox(0,0)[cb]{$Z_{n-1}$}}

\put(84,2.5){\framebox(10,5){$F$}}

\put(89,15){\vector(0,-1){7.5}}
\put(92,10){\makebox(0,0)[cb]{$X_{n}$}}

\put(94,5){\vector(1,0){12}}
\put(100,6){\makebox(0,0)[cb]{$Z_{n}$}}

\end{picture}
\end{center}
\caption{\emph{Merkle-Damgård}-Konstruktion $H_{MD}$}
\label{fig:md-konstruktion}
\end{figure}

\subsection{Sicherheit von Merkle-Damgård}
Die Sicherheit einer Merkle-Damgård-Konstruktion $H_{MD}$ hängt stark von der verwendeten Kompressionsfunktion $F$ ab:\\

\begin{theorem}
Ist $F$ kollisionsresistent, so ist auch $H_{MD}$ kollisionsresistent.\\
\end{theorem}

\begin{beweis}
Seien $X, X'$ zwei Urbilder von $H_{MD}$ mit $X \not = X'$ und
\begin{equation*}
H_{MD}(X) = Z_{n} = Z'_{n} = H_{MD}(X')
\end{equation*}
Wir suchen nun eine Kollision in $F$.\\
Falls $(Z_{n-1}, X_n) \not = (Z'_{n-1}, X'_n)$, wurde eine Kollision in F gefunden, da $X_n \not = X'_n$, aber $Z_{n} = F(Z_{n-1}, X_{n}) = F(Z'_{n-1},
X'_{n}) = Z'_{n}$.\\
Falls nicht, prüfe, ob $(Z_{n-2}, X_{n-1}) \not = (Z'_{n-2}, X'_{n-1})$ und damit eine Kollision in $F$ gefunden ist.\\
Falls nicht, prüfe, ob $(Z_{n-3}, X_{n-2}) \not = (Z'_{n-3}, X'_{n-2})$. \\
\ldots\\
Da nach Voraussetzung $X \not = X'$ ist, muss irgendwann $(Z_{i-1}, X_{i}) \not = (Z'_{i-1}, X'_{i})$ gelten. Damit ist die Kollision von $H_{MD}$ auf eine
Kollision in $F$ zurückführbar.\\
\qed
\end{beweis}

\subsection{Bedeutung von Merkle-Damgård}

\subsubsection{Secure Hash Algorithm (SHA)}
Im Jahr 1995 veröffentlichte die NIST den von der NSA entworfenen, auf der Merkle-Damgård-Transformation beruhenden, kryptographischen Hashalgorithmus \textit{Secure Hash Algorithm 1} (SHA-1) \cite{NIST_SHA95}. Lange Zeit war SHA-1 die wichtigste kryptographische Hashfunktion, bis der Algorithmus im Jahr 2005 zumindest theoretisch gebrochen wurde. Es existieren also Angriffe, die schneller als eine Brute-Force-Suche sind, eine explizite Kollision wurde bislang allerdings nicht gefunden. In Folge des Bekanntwerdens der Schwachstellen empfiehlt die NIST auf die Verwendung von SHA-1 zu verzichten. Dennoch hat SHA-1 wenig von seiner Verbreitung eingebüßt und wird heutzutage immer noch weitreichend verwendet, z.B. bei Prüfsummen.

\paragraph*{Ablauf des Hash-Vorgangs}

\begin{enumerate}
	\item Teile die Nachricht in $n$ 512-Bit große Blöcke $M_1, \dots, M_n$ auf und padde den letzten Block bei Bedarf
	\item Initialisiere $H_0^{(0)}, \dots, H_4^{(0)}$ mit fest gewählten Konstanten und setze $a = H_0^{(0)}, \dots, e = H_4^{(0)}$
	\item Für alle Nachrichtenblöcke $M_i$ von $i = 1, \dots, n$:
	\begin{enumerate}
		\item Führe 80 Berechnungsrunden $t = 0, \dots, 79$ aus, um die neuen Hashwerte für $a, \dots, e$ zu bestimmen		
		\begin{figure}[h]
			\centering
			\tikzstyle{every circle node}= [draw]
			\begin{tikzpicture}
			\begin{scope}[>=latex] %for filled arrow tips
			\newcommand{\n}{5}
			
			\pgfmathsetmacro\minBoxWidth{(2.0)} // 2.2 old
			\pgfmathsetmacro\minBoxHeight{(0.6)} // 0.5 old
			
			\pgfmathsetmacro\xA{int(0)}
			\pgfmathsetmacro\xB{(\minBoxWidth * 1)}
			\pgfmathsetmacro\xC{(\minBoxWidth * 2)}
			\pgfmathsetmacro\xD{(\minBoxWidth * 3)}
			\pgfmathsetmacro\xE{(\minBoxWidth * 4)}
			
			\node (h0) at (\xA, {(\n -1)*2 + 2}) [draw,thick,minimum width=\minBoxWidth cm, minimum height=\minBoxHeight cm] {$H_0^{(i-1)}$};
			\node (h1) at (\xB, {(\n -1)*2 + 2}) [draw,thick,minimum width=\minBoxWidth cm, minimum height=\minBoxHeight cm] {$H_1^{(i-1)}$};
			\node (h2) at (\xC, {(\n -1)*2 + 2}) [draw,thick,minimum width=\minBoxWidth cm, minimum height=\minBoxHeight cm] {$H_2^{(i-1)}$};
			\node (h3) at (\xD, {(\n -1)*2 + 2}) [draw,thick,minimum width=\minBoxWidth cm, minimum height=\minBoxHeight cm] {$H_3^{(i-1)}$};
			\node (h4) at (\xE, {(\n -1)*2 + 2}) [draw,thick,minimum width=\minBoxWidth cm, minimum height=\minBoxHeight cm] {$H_4^{(i-1)}$};
			
			\pgfmathsetmacro\abcdeLineYCoord{(\n -1)*2 + 0.9}
			
			\node (a) at (\xA, \abcdeLineYCoord) [draw,thick,minimum width=\minBoxWidth cm, minimum height=\minBoxHeight cm] {$a$};
			\node (b) at (\xB, \abcdeLineYCoord) [draw,thick,minimum width=\minBoxWidth cm, minimum height=\minBoxHeight cm] {$b$};
			\node (c) at (\xC, \abcdeLineYCoord) [draw,thick,minimum width=\minBoxWidth cm, minimum height=\minBoxHeight cm] {$c$};
			\node (d) at (\xD, \abcdeLineYCoord) [draw,thick,minimum width=\minBoxWidth cm, minimum height=\minBoxHeight cm] {$d$};
			\node (e) at (\xE, \abcdeLineYCoord) [draw,thick,minimum width=\minBoxWidth cm, minimum height=\minBoxHeight cm] {$e$};
			
			\draw[->,semithick] (h0) -- (a);
			\draw[->,semithick] (h1) -- (b);
			\draw[->,semithick] (h2) -- (c);
			\draw[->,semithick] (h3) -- (d);
			\draw[->,semithick] (h4) -- (e);
			
			\pgfmathsetmacro\numAddBoxes{int(4)}
			\pgfmathsetmacro\lastAddBoxIndex{int(\numAddBoxes)}
			\pgfmathsetmacro\firstAddBoxYCoord{\abcdeLineYCoord - 1.5}
			
			%\node (plus0)[circle, draw] at (\xE,\firstAddBoxYCoord) {$+$};
			
			\foreach \nr in {1, ..., \numAddBoxes}{
				\node (plus\nr)[circle] at (\xE,{\abcdeLineYCoord - (1.5 * \nr)}) {$+$};
			}	
			
			\pgfmathsetmacro\lastAddBoxYCoord{\abcdeLineYCoord - 1.5 * \numAddBoxes}
			\draw[->,semithick] (e) -- (plus1);
			
			\foreach \nr in {1, ..., 3}{
				\pgfmathsetmacro\cnr{int(\nr + 1)}
				\draw[->,semithick] (plus\nr) -- (plus\cnr);
			}
			
			%draw Key and Message arrow
			\node (KeyArrow) at (\xE + 2.0, {\abcdeLineYCoord - (1.5 * 2)}) {$K_t$};
			\draw[->, semithick] (KeyArrow) -- (plus2);
			
			\node (MessageArrow) at (\xE + 2.0, {\abcdeLineYCoord - (1.5 * 4)}) {$W_t$};
			\draw[->, semithick] (MessageArrow) -- (plus4);
			
			\node (shiftL5)[ellipse, draw] at (\minBoxWidth * 0.5 + 0.1, \firstAddBoxYCoord) {$<< 5$};
			
			\node (p1)[circle, fill, inner sep=0cm, minimum size=0.12cm] at (\xA, \firstAddBoxYCoord) {};
			\draw[->, semithick] (p1) -- (shiftL5);
			\draw[->, semithick] (shiftL5) -- (plus1);
			
			\node (shiftR2)[ellipse, draw] at (\xB, \lastAddBoxYCoord) {$>> 2$};
			\draw[->, semithick] (b) -- (shiftR2);
			
			\node (rectF)[draw,thick,rounded corners,minimum width=0.6cm, minimum height=2.1cm] at (\xD + 0.7, {\abcdeLineYCoord - (1.5 * 3)}) {$f_t$};
			\node (p2)[circle, fill, inner sep=0cm, minimum size=0.12cm] at (\xC, {\abcdeLineYCoord - (1.5 * 3)}) {};
			\draw[->, semithick] (p2) |- (rectF);
			\draw[->, semithick] (rectF) -- (plus3);
			
			\node (p3)[circle, fill, inner sep=0cm, minimum size=0.12cm] at (\xB, {\abcdeLineYCoord - (1.5 * 3) + 0.5}) {};
			\draw[->, semithick] (p3) -- ($(rectF) + (-0.3, 0.5)$);
			
			\node (p4)[circle, fill, inner sep=0cm, minimum size=0.12cm] at (\xD, {\abcdeLineYCoord - (1.5 * 3) - 0.5}) {};
			\draw[->, semithick] (p4) -- ($(rectF) + (-0.3, -0.5)$);
			
			
			\node (aNew) at (\xA, {\lastAddBoxYCoord - 2.5}) [draw,thick,minimum width=\minBoxWidth cm, minimum height=\minBoxHeight cm] {$a$};
			\node (bNew) at (\xB, {\lastAddBoxYCoord - 2.5}) [draw,thick,minimum width=\minBoxWidth cm, minimum height=\minBoxHeight cm] {$b$};
			\node (cNew) at (\xC, {\lastAddBoxYCoord - 2.5}) [draw,thick,minimum width=\minBoxWidth cm, minimum height=\minBoxHeight cm] {$c$};
			\node (dNew) at (\xD, {\lastAddBoxYCoord - 2.5}) [draw,thick,minimum width=\minBoxWidth cm, minimum height=\minBoxHeight cm] {$d$};
			\node (eNew) at (\xE, {\lastAddBoxYCoord - 2.5}) [draw,thick,minimum width=\minBoxWidth cm, minimum height=\minBoxHeight cm] {$e$};
			%\draw[->, semithick] (shiftL5) -- (plus1);
			
			\draw[->, semithick] (a) -- (\xA, {\lastAddBoxYCoord - 1.0}) 
			-- (\xB, {\lastAddBoxYCoord - 1.5}) -- (bNew);
			
			\draw[->, semithick] (shiftR2) -- ($(shiftR2) + (0, -1)$) 
			-- (\xC, {\lastAddBoxYCoord - 1.5}) -- (cNew);
			
			\draw[->, semithick] (c) -- (\xC, {\lastAddBoxYCoord - 1.0}) 
			-- (\xD, {\lastAddBoxYCoord - 1.5}) -- (dNew);
			
			\draw[->, semithick] (d) -- (\xD, {\lastAddBoxYCoord- 1.0}) 
			-- (\xE, {\lastAddBoxYCoord - 1.5}) -- (eNew);
			
			\draw[->, semithick] (plus\lastAddBoxIndex) -- (\xE, {\lastAddBoxYCoord- 0.7}) 
			-- (\xA, {\lastAddBoxYCoord - 1.5}) -- (aNew);
			
			\node (AddModText) at (\xE - 1.4, {\lastAddBoxYCoord - 3.0}) {\scriptsize{Addition ist modulo $2^{32}$ zu verstehen}};
			
			\end{scope}
			\end{tikzpicture}
			\caption{Schema der Berechnungsrunde}
		\end{figure}
		\item Setze $H_0^{(i)} = H_0^{(i-1)} + a, \dots, H_4^{(i)} = H_4^{(i-1)} + e$
	\end{enumerate}
	\item Gebe $H_0^{(n)} \concat \dots \concat H_4^{(n)}$ als 160-Bit Hashwert (\textit{message digest}) aus
\end{enumerate} 

In jeder der 80 Berechnungsrunden zum Berechnen eines Zwischenergebnisses werden folgende Funktionen, Konstanten und Variablen verwendet:
\begin{itemize}
	\item Rundenfunktion $f_t$ führt, je nach Index, unterschiedliche Elementaroperationen auf den 32-Bit langen Variablen $b, c, d$ aus
	\item Konstante $K_t$ hat, je nach Index, vier verschiedene Werte
	\item Variable $W_t$, als \textit{message schedule} bezeichnet, besteht in den ersten 16 Runden jeweils aus einem 32-Bit Wort des aktuellen 512-Bit großen Nachrichtenblocks $M_i$ und in den verbleibenden 64 Runden aus einem rekursiv berechneten Wert vergangener message schedules des gleichen Blocks.
\end{itemize}

Für die Eingangs erwähnten Angriffe auf die beschriebene Konstruktion wird die Möglichkeit ausgenutzt, für eine Runde Kollisionen zu finden und versucht, diese auf mehrere Runden auszuweiten. Dabei sind auch ähnliche Ausgaben hilfreich. Der schnellste der im Jahr 2005 vorgestellten Algorithmen benötigt mit ungefähr $2^{63}$-Schritten (vgl. $2^{80}$-Schritte für einen Brute-Force-Angriff) zwar noch immer einen beträchtlichen Rechenbedarf, erzeugt jedoch Kollisionen über alle 80 Berechnungsrunden.

Neben SHA-1 ist der im Jahr 1992 von Ronald Rivest veröffentlichte MD5-Algorithmus eine bekannte Hashfunktion, die auf dem Merkle-Damgård-Konstrukt beruht und für eine Vielzahl kryptographischer Anwendungen und Datenintegritäts-Sicherung eingesetzt wurde. Von der Verwendung von MD5 sollte für sicherheitsrelevante Anwendungsszenarien mittlerweile jedoch abgesehen werden: Im Unterschied zu SHA-1 können bei MD5 explizite Kollisionen gefunden werden. Im Jahr 2013 stellten Xie Tao, Fanbao Liu und Dengguo Feng den bis dato besten Angriff vor, der aus einer Menge von etwa $2^{18}$ MD5 Hashwerten ein Kollisionspaar findet. Heutige Prozessoren benötigen dafür weniger als eine Sekunde.
%fact check mit Hofheinz

Aufgrund der Verwundbarkeit von MD5 und SHA-1 empfiehlt die NIST heutzutage mindestens eine Hashfunktion der SHA-2-Familie zu verwenden. Ähnlich zu SHA-1 basieren die Funktionen dieser Hash-Familie auf der Merkle-Damgård-Konstruktion, bieten jedoch in der Praxis, aufgrund des größeren Bildraums, ein höheres Maß an Sicherheit. Theoretisch aber bleiben die Funktionen, wegen großen Ähnlichkeiten in der Konstruktion, verwundbar. Deshalb wurde im Jahr 2013 mit SHA-3 (\glqq Keccak\grqq{}-Algorithmus) der Versuch gestartet, eine grundlegend andere kryptographische Hashfunktion zu standardisieren.\footnote{Die Übersicht über den Standardisierungsprozess findet sich auf \url{http://csrc.nist.gov/groups/ST/hash/sha-3/sha-3_standardization.html}.}

\section{Angriffe auf Hashfunktionen}
\subsection{Birthday-Attack}
Für diesen Angriff berechnen wir möglichst viele $Y_i = H(X_i)$.
Danach suchen wir unter diesen Hashwerten nach Gleichheit (und finden so $X \not = X'$ mit $H(X) = Y = Y' = H(X')$).
\vspace{10pt}

\textbf{Vorgehen:}
\begin{enumerate}
  \item Schreibe $(X_i, Y_i)$ in Liste. Dabei ist $X_i \in \{0,1\}^{2k}$ gleichverteilt und $Y_i = H(X_i)$.
  \item Sortiere die Liste nach $Y_i$.
  \item Untersuche die Liste auf $Y_i$-Kollisionen.
\end{enumerate}
\vspace{10pt}

\begin{theorem}
Sei $n \leq 2^{\frac{k}{2}}$ und $Y_1, \ldots , Y_n \in \{0,1\}^k$ unabhängig gleichverteilt. Dann gibt es $i \not = j$ mit $Y_i = Y_j$ mit Wahrscheinlichkeit
$p > \frac{1}{11} \cdot \frac{n^2}{2^k}$.
\end{theorem}
\vspace{10pt}

Wir haben also schon für $n = 2^{\frac{k}{2}}$ zufällige, verschiedene $X_i$ mit einer Wahrscheinlichkeit von $p > \frac{1}{11}$ Kollisionen unter den
den dazugehörigen $Y_i$. Für die Berechnung brauchen wir $\Theta(k \cdot 2^{\frac{k}{2}})$ Schritte und haben einen Speicherbedarf von $\Theta(k \cdot
2^{\frac{k}{2}})$ Bits.
\vspace{10pt}

\begin{beispiel}
Alice möchte ein paar Tage Urlaub buchen und holt dafür Angebote ein. Mallory würde gern eine Weltreise machen, möchte die Kosten aber nicht tragen und nutzt
jetzt aus, dass Alice gerade einen Urlaub buchen möchte. Er berechnet die Hashwerte von unterschiedlichen Kurztrips für Alice und einige Buchungen über
eine Weltreise für sich selbst auf Kosten von Alice. Wenn er eine Kollision gefunden hat, also einen Hashwert, der sowohl einen Kurztrip als auch eine
Weltreise abdeckt, lässt er von Alice den Hashwert der harmlosen Kurztripbuchung signieren. Bevor er diese jedoch an das Reisebüro weiterleitet, tauscht er die
Buchungen aus. Das Reisebüro prüft die Buchung der Weltreise samt Hashwert und Signatur und befindet sie für gültig.
\end{beispiel}

\subsection{Weitere Angriffe}
Auch ein Meet-in-the-Middle-Angriff kann die Zeit zum Auffinden einer Kollision verkürzen. Allerdings setzt dieser Angriff voraus, dass die
Hashfunktion eine "`Rückwärtsberechnung"' zulässt.
\vspace{10pt}

\textbf{Vorgehen:}
\begin{enumerate}
  \item Gehe aus von $M = M_1M_2$.
  \item Verändere $M_1$ möglichst oft, erzeuge eine Liste aller $Z$.
  \item Sortiere die Liste aller $Z$.
  \item Rechne von $Y=H(M)$ zurück zu $Z$ für viele Kandidaten für $M_2$.
\end{enumerate}

\begin{figure}[h]
\begin{center}
\unitlength=1mm
\linethickness{0.4pt}
\hspace{-3 cm}
\begin{picture}(50,10)

\put(0,2){\makebox(0,0)[cb]{$IV$}}

\put(5,3){\vector(1,0){15}}
\put(12,4){\makebox(0,0)[cb]{$M_1$}}

\put(20,0.5){\makebox(10,5){$Z$}}

\put(30,3){\vector(1,0){15}}
\put(37,4){\makebox(0,0)[cb]{$M_2$}}

\put(55,0.5){\makebox(10,5){$Y = H(M_1M_2)$}}

\end{picture}
\end{center}
\caption{Hilfsskizze für Meet-in-the-Middle-Angriff auf eine Hashfunktion $H$}
\label{fig:md-meet-in-the-middle-attack}
\end{figure}

Als Ergebnis aus diesem Angriff erhalten wir $M'_1$ und $M'_2$ mit $H(M'_1M'_2) = H(M_1M_2)$. Der Aufwand für diesen Angriff nähert sich asymptotisch dem für
die Geburtstagsattacke an.

\subsection{Fazit}
Die vorgestellten Angriffe zeigen, dass sich der Aufwand zum Finden einer Kollision gegenüber einer Brute-Force-Attacke stark verringern lässt. Bei einer
Hash-Ausgabe mit einer Länge $\geq k$ Bits kann man nur mit einer "`Sicherheit"' von $\frac{k}{2}$ Bits rechnen.
