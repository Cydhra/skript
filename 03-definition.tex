\chapter{Kryptographische  Sicherheitsbegriffe}
\label{chap:krypt-begriffe}
\section{Sicherheitsparameter und effiziente  Angreifer}
\label{sec:secparam} 
Zu einer Funktion, für die sicherheitsrelevante Eigenschaften gefordert
werden, wird in der Kryptographie oft ein \emph{Sicherheitsparameter}
\indexSecParam $k$ definiert. Informell gesagt, legt $k$ das
Sicherheitsniveau der Funktion fest. Beispielsweise parametrisiert er
den Schlüsselraum eines Verschlüsselungsverfahrens, was es schwieriger
macht, den korrekten Schlüssel zu raten oder per Brute-Force zu
berechnen.
\begin{beispiel} Wir betrachten ein symmetrisches
  Verschlüsselungsverfahren für das als Schlüsselraum $\{0,1\}^k$
  verwendet wird und die Schlüssel gleichverteilt zufällig gezogen
  werden. Die Schlüssel sind also Bitstrings der Länge $k$ und es
  existieren $2^k$ mögliche Schlüssel. Somit muss ein Angreifer im
  Worst-Case bei der Brute-Force Methode $2^k$ Schlüssel durchprobieren
  oder kann den korrekten Schlüssel durch (einmaliges) Raten mit
  Wahrscheinlichkeit $1/2^k$ bestimmen.
  \begin{description}
  \item[Fall $k=128$:] Es existieren $2^{128} > 3.4 \cdot
    10^{38}$ mögliche Schlüssel.
  \item[Fall $k=512$:] Es existieren $2^{512} > 1.3 \cdot
    10^{154}$ mögliche Schlüssel. Zum Größenvergleich: Die Anzahl an Atomen
    im sichtbaren Universum wird häufig auf $10^{80}$ geschätzt.
  \end{description}
  
\end{beispiel}

Zur Analyse der Sicherheitseigenschaften eines kryptographischen
Verfahrens betrachtet man hauptsächlich Angreifer, die \emph{effizient}
\indexEfficientAdv, das heißt in ihrer Rechenzeit geeignet eingeschränkt
sind. In der Komplexitätstheorie und auch in der Kryptographie wird ein
effizienter Algorithmus mit einer polynomial-beschränkten Laufzeit in
der Eingabegröße gleichgesetzt.  In anderen Worten ist ein Algorithmus
bei Eingabe eines Bit-Strings der Länge $n$ genau dann effizient, wenn
es ein $c \in \mathbbm{N}$ gibt, so dass seine Laufzeit im schlechtesten
Fall in $O(n^c)$ liegt. Wir betrachten in der Kryptographie also
asymptotische Sicherheit, ähnlich wie die asymptotische
Laufzeitbetrachtung in der Algorithmik.

Um präzise über die Laufzeit im Bezug auf den Sicherheitsparameter
argumentieren zu können, erhalten Algorithmen und Angreifer den
Bit-String $1^k$ als Eingabe (hiermit ist der Bitstring bestehende aus
$k$ Einsen gemeint). Ihre Rechenzeit ist damit also durch den
Sicherheitsparameter $k$ begrenzt.\footnote{Deshalb wird auch $1^k$
  statt $k$ übergeben, da $k$ mithilfe von nur $O(\log(k))$ Bits
  repräsentierbar ist. Die Laufzeit der Algorithmen könnte so also nur
  abhängig von $O(\log(k)) \neq k$ betrachtet werden.}

Ein effizienter Angreifer muss also in $O(k^c)$-Schritten, $c \in
\mathbbm{N}$, eine Lösung berechnen. Somit ist beispielsweise die
eingangs erwähnte Brute-Force-Attacke auf einen Schlüsselraum $\{0,
1\}^{k}$ ausgeschlossen, da die Laufzeit in $O(2^{k})$ liegt, also
exponentiell ist.  Neben der Begrenzung der Rechenzeit erlauben wir
einem Angreifer probabilistische Algorithmen zu verwenden. Einen solchen
Angreifer bezeichnen wir als \emph{probabilistic polynomial time} (PPT)
Angreifer \indexPPTAdv.

Damit ein kryptographisches Verfahren als sicher gelten kann, muss die
Erfolgswahrscheinlichkeit eines Angreifers möglichst \glqq klein\grqq{}
sein. In der Kryptographie hat sich hier der Begriff der
\emph{Vernachlässigbarkeit} \indexNegl durchgesetzt:
\begin{definition}[Vernachlässigbarkeit]\label{def:negl}
 Eine Funktion $f: \N \to \R$
  ist \emph{vernachlässigbar in $k$}, wenn gilt:
  \begin{align*} \forall c \in \N_0 \; \exists k_0 \in \N \;
    \forall k \geq k_0 : \vert f(k) \vert \leq \frac{1}{k^c}
  \end{align*}
\end{definition} Eine vernachlässigbare Funktion \glqq
verschwindet\grqq{} (d.h. geht gegen Null) also schneller als der
Kehrwert jedes Polynoms. Beispielsweise ist $f = \frac{1}{2^k}$
vernachlässigbar in $k$, $f = \frac{1}{k^2}$ jedoch nicht.

Die Wahl eines für die Praxis geeigneten Sicherheitsparameters ist nicht
trivial. Hierbei müssen viele Faktoren beachtet werden. Beispielsweise
gelten für uns Angriffe mit exponentieller Laufzeit als nicht effizient,
die stetig schneller werdende Hardware macht aber immer mehr solche
Angriffe praktikabel durchführbar.  Außerdem darf nicht nur der naive
Brute-Force-Angriff in Betracht gezogen werden. Für viele Verfahren gibt
es weitere nicht effiziente Angriffe, unter anderem die schon
vorgestellte lineare Kryptoanalyse. Diese Angriffe haben zwar ebenfalls
exponentielle Laufzeit, sind aber effizienter als der Brute-Force
Angriff. Der Sicherheitsparameter muss also so gewählt werden, dass alle
bekannten ineffizienten Angriffe auch tatsächlich nicht in praktikabler
Zeit durchführbar sind.

Der Sicherheitsparameter ist hauptsächlich ein theoretisches Werkzeug,
um über Laufzeiten und Erfolgswahrscheinlichkeiten argumentieren zu
können. In der Praxis wird er implizit durch die Wahl der Schlüssellänge
festgelegt. Aus den obigen Gründen ist es ratsam, sich bei der Wahl der
Schlüssellänge an die Empfehlungen von vertrauenswürdigen Instanzen oder
Standards zu halten. Solche Empfehlungen gibt es beispielsweise vom
\href{https://www.bsi.bund.de/DE/Publikationen/TechnischeRichtlinien/tr02102/index_htm.html}{Bundesamt
  für Sicherheit in der Informationstechnik} oder der
\href{https://www.enisa.europa.eu/activities/identity-and-trust/library/deliverables/algorithms-key-size-and-parameters-report-2014}{European
  Union Agency for Network and Information Security}.

\section{Semantische  Sicherheit}
\label{ch:sicherheitsbegriffe:semantischesicherheit} Nachdem
wir uns bereits mit Verschlüsselungssystemen auseinandergesetzt haben,
stellt sich natürlich die Frage, welche Form von Sicherheit wir
erreichen möchten.  Eines der primären Ziele war es bisher, dass ein
PPT-Angreifer \indexPPTAdv durch das Chiffrat keinerlei Informationen
über den Klartext erhält. Dies entspricht dem Begriff der
\emph{semantischen Sicherheit}, welcher 1983 in einer Arbeit von Shafi
Goldwasser und Silvio Micali \cite{Goldwasser1984} definiert wurde und
besagt umgangssprachlich:
\begin{quote} \emph{Alle Informationen, die mit $\ciphert$ effizient
    über $\plaint$ berechnet werden können, sind auch ohne das Chiffrat
    berechenbar.}
\end{quote} Dabei ist zu beachten, dass diese Form von Sicherheit
lediglich passive Angriffe abdeckt.

Um semantische Sicherheit formal zu beschreiben, verwenden wir die Idee
eines \emph{Orakels} \indexOracle. Ein Orakel funktioniert als
\emph{black box}, bei dem der Fragende zwar das Ergebnis, jedoch nichts
über dessen Berechnung in Erfahrung bringt. Betrachten wir
beispielsweise ein Verschlüsselungsorakel, so liefert es bei Eingabe
eines Klartextes $\plaint$ das entsprechende Chiffrat $\enc(\key,
\plaint)$, wobei $\key$ fest in das Orakel implementiert ist. Wir
schreiben $\A^{\enc (\key, \cdot)}$, wenn einem Angreifer $\A$ ein
solches Orakel zur Verfügung steht.

\begin{definition}[Semantische Sicherheit]\label{def:semsec} Ein
  symmetrischer Verschlüsselungsalgorithmus ist semantisch sicher, wenn es
  für jede $\plaint$-Verteilung von Nachrichten gleicher Länge, jede
  Funktion $f$ und jeden PPT-Algorithmus $\A$ einen PPT-Algorithmus $\B$
  gibt, so dass
  \begin{align*} \text{Pr}\left[ \A^{\enc (\key,\cdot)} \left( \enc \left(
    \key, \plaint \right) \right) = f(\plaint) \right] - \text{Pr} \left[ \B
    (\epsilon) = f (\plaint) \right]
  \end{align*} vernachlässigbar ist.
\end{definition}

Allerdings impliziert die Existenz von mehrfach benutzbaren, semantisch
sicheren Verfahren damit $P \neq NP$. Das bedeutet, falls $P = NP$
gelten sollte, kann es kein solches Verfahren geben. Außerdem ist diese
Definition technisch schwer zu handhaben, da sie viele Quantoren
enthält. Hierfür wurden handlichere, aber äquivalente Begriffe
eingeführt, wie beispielsweise \emph{IND-CPA}.

\section{Der IND-CPA-Sicherheitsbegriff}\label{sec:ind-cpa} IND-CPA
steht für \emph{indistinguishability under chosen-plaintext attacks}
\indexINDCPA. Bei einem Verfahren, welches diese Sicherheit besitzt,
kann ein polyomiell beschränkter Angreifer $\A$ die Chiffrate von
selbstgewählten Klartexten nicht unterscheiden.
\begin{definition}[IND-CPA-Sicherheit] Betrachte folgendes Experiment
  mit einem Herausforderer $\C$ und einem PPT-Algorithmus $\A$, bei dem
  $\C$ einen Schlüssel $K$ zufällig gleichverteilt wählt und $\A$ ein
  Verschlüsselungsorakel $\enc(\key,\cdot)$ bereitstellt:
  \begin{itemize}
  \item $\A$ kann sich zu jedem Zeitpunkt jedes beliebige
    $\plaint$ vom Orakel verschlüsseln lassen.
  \end{itemize}
  \begin{enumerate}
  \item $\A$ wählt zwei Nachrichten $\plaint_1, \plaint_2$
    gleicher Länge.
  \item $\A$ erhält $\ciphert^{*} := \enc(\key,
    \plaint_{b})$ für ein von $\C$ zufällig gleichverteilt gewähltes $b \in
    \{1, 2\}$.
  \item $\A$ gewinnt, wenn er $b$ korrekt errät.
  \end{enumerate} Ein symmetrisches Verfahren $(\enc, \dec)$ heißt
  IND-CPA-sicher, wenn der Vorteil des PPT-Algorithmus gegenüber dem Raten
  einer Lösung, also $\Pr \left[ \A \textnormal{ gewinnt} \right] -
  \frac{1}{2}$, für alle PPT-Algorithmen $\A$ vernachlässigbar im
  Sicherheitsparameter $k$ ist.
\end{definition} Abbildung \ref{fig:ind-cpa} stellt den Ablauf dieses
Sicherheitsexperimentes noch einmal grafisch dar.

Der Orakelbegriff \indexOracle ermöglicht es uns einem Angreifer neben
$C^{\ast}$ zusätzliche Informationen zu geben und dementsprechend einen
stärkeren Sicherheitsbegriff zu erhalten. So sind beispielsweise
deterministische Verfahren grundsätzlich nicht IND-CPA-sicher.
\begin{figure}[h] \centering \scalebox{1.2}{
    \begin{tikzpicture}[x=2em, y=2em] \draw (-6,0) node (C) {$\C$};
      \draw (0,0) node (A) {$\A$}; \draw (7,0) node (Ork) {Orakel};
      
      \draw[dashed] (C) -- (-6,-7); \draw[dashed] (A) --
      (0,-8); \draw[dashed] (Ork) -- (7,-8);
      
      \draw[->, thick] (0.3,-1) -- (6.7,-1.5)
      node[sloped,above,pos=0.5] {$\plaint$}; \draw[->, thick] (6.7,-3) --
      (0.3,-3.5) node[sloped,above,pos=0.5] {$\ciphert = \enc(\key,
        \plaint)$};
      
      \draw (3.5,-5) node (exp) {(Anfragen immer erlaubt!)};
      
      \draw[->, thick] (-0.3, -2.5) -- (-5.7,-3)
      node[sloped,above,pos=0.5] {$\plaint_1, \plaint_2$}; \draw (-7.5,-3.5)
      node (b1) {$ b \randUnif \{1,2\}$};
      
      \draw[->, thick] (-5.7, -4) -- (-.3,-4.5)
      node[sloped,above,pos=0.5] {$\ciphert^{*} = \enc(\key, \plaint_b)$};
      \draw[->, thick] (-0.3, -5.5) -- (-5.7,-6) node[sloped,above,pos=0.5]
      {$b'$};
      
      \draw (-5.8,-7.5) node (b2) {$ b = b'$?};
    \end{tikzpicture} }
  \caption{Ablauf des IND-CPA-Experiments.}
  \label{fig:ind-cpa}
\end{figure} Wir bemerken, dass der IND-CPA-Sicherheitsbegriff
beispielsweise impliziert, dass der Schlüssel $\key$ schwer, also nicht
in Polynomialzeit, berechenbar ist: Angenommen $\A$ kennt $\key$, dann
kann der Angreifer $C^{\ast}$ entschlüsseln, mit $\plaint_1$ und
$\plaint_2$ vergleichen und gewinnt somit immer. Es gilt also $\Pr
\left[ \A \textnormal{ gewinnt} \right] - \frac{1}{2} = 1 - \frac{1}{2}$
und damit ist die IND-CPA-Sicherheit des zugrundeliegenden
Verschlüsselungsverfahrens gebrochen.

\begin{theorem} Ein Verfahren ist genau dann semantisch sicher, wenn es
  IND-CPA-sicher ist.
\end{theorem}
\begin{beweis} ohne Beweis
\end{beweis} Da der Beweis dieser Aussage über das Niveau einer
einführenden Kryptographie-Vorlesung hinausgeht, wollen wir an dieser
Stelle auf eine Ausführung verzichten und verweisen den interessierten
Leser auf die Arbeit von Goldwasser und Micali
\cite{Goldwasser1984}. Bemerken möchten wir jedoch, dass die Autoren
nicht von der IND-CPA-Sicherheit eines Verschlüsselungsverfahrens
sprechen, sondern ein entsprechendes Verfahren als "`polynomial secure"'
bezeichnen.

\subsection{Beispiel ECB-Modus}
\begin{description}
\item[Behauptung] Keine Blockchiffre ist im ECB-Modus \indexECB
  IND-CPA-sicher.
\item[Beweis] Betrachte folgenden PPT-Angreifer $\A$:
  \begin{itemize}
  \item $\A$ wählt zwei Klartextblöcke $\plaint_1 \neq
    \plaint_2$ beliebig.
  \item $\A$ erhält $\ciphert^{*} := \enc(\key,
    \plaint_{b})$ für ein von $\C$ zufällig gleichverteilt gewähltes $b \in
    \{1, 2\}$.
  \item $\A$ erfragt $\ciphert_1 = \enc(\key, \plaint_1)$
    durch sein Orakel.
  \item $\A$ gibt 1 aus, genau dann, wenn $\ciphert_1 =
    \ciphert^{*}$, sonst gibt er 2 aus.
  \item $\Pr \left[ \A \text{ gewinnt} \right] = 1$, also
    ist das Schema nicht IND-CPA-sicher.
  \end{itemize}
\end{description} Bei diesem Beispiel nutzt der Angreifer die Schwäche
des ECB-Modus, dass gleiche Klartextblöcke immer zu gleichen
Chiffrat-Blöcken werden, aus.

\subsection{Beispiel CBC-Modus} 
\begin{description}
\item[Behauptung] Eine Blockchiffre ist im CBC-Modus \indexCBC
  genau dann IND-CPA-sicher, wenn die Verschlüsselungsfunktion
  $\enc(\key,\cdot) \colon \{0,1\}^n \rightarrow \{0,1\}^n$ nicht von
  einer zufälligen Funktion $R \colon \{0,1\}^n \rightarrow \{0,1\}^n$
  unterscheidbar ist.
\item[Beweisidee]~
  \begin{description}
  \item[(IND-CPA-sicher $\Rightarrow$
    Ununterscheidbarkeit)]~\\ $\Leftrightarrow$ IND-CPA-unsicher
    $\Leftarrow$ Unterscheidbarkeit~\\ Wenn ein Angreifer $\enc(\key,
    \cdot)$ von einer Zufallsfolge unterscheiden kann, ist zwischen
    mindestens zwei Verschlüsselungsergebnissen ein Zusammenhang
    erkennbar. Es gibt somit mindestens einen Fall, bei dem der Angreifer
    zusätzliche Informationen für das Zuordnen des Chiffrats besitzt. Daher
    gilt für zufällig gewählte Nachrichten im IND-CPA-Experiment: Pr$[\A$
    gewinnt$] > \frac{1}{2} \Rightarrow$ IND-CPA-unsicher.
  \item[IND-CPA-sicher $\Leftarrow$
    Ununterscheidbarkeit]~\\ Wenn die Verschlüsselungsfunktion aus Sicht des
    Angreifers nicht von einer Zufallsfunktion unterscheidbar ist, gibt es
    keine bekannten Zusammenhänge der Verschlüsselungen. Somit ist die
    Wahrscheinlichkeit, dass der Angreifer ein Chiffrat korrekt zuordnet,
    genau $\frac{1}{2}$.
  \end{description}
\end{description}

\section{Der IND-CCA-Sicherheitsbegriff}\label{sec:ind-cca} Der
CPA-Angreifer ist mit Zugriff auf ein Verschlüsselungsorakel
ausgestattet. Er kann sich jedmöglichen Klartext verschlüsseln lassen
und versuchen, Muster in den Ausgaben des Orakels zu
erkennen. Eingeschränkt ist er dennoch, da ihm die Möglichkeit fehlt, zu
beliebigen Ciphertexten den Klartext zu berechnen. Ein stärkerer
Sicherheitsbegriff ist daher IND-CCA (\emph{indistinguishability under
  chosen-ciphertext attacks}) \indexINDCCA. Dabei suggeriert das Akronym
CCA bereits einen mächtigeren Angreifer. Das in \ref{sec:ind-cpa}
vorgestellte Experiment können wir problemlos auf einen
IND-CCA-Angreifer anpassen.

\begin{definition}[IND-CCA-Sicherheit] Betrachte folgendes Experiment
  mit einem Herausforderer $\C$ und einem PPT-Algorithmus $\A$, bei dem
  $\C$ einen Schlüssel $K$ zufällig gleichverteilt wählt und $\A$ ein
  Verschlüsselungsorakel $\enc(\key,\cdot)$ sowie ein
  Entschlüsselungsorakel $\dec(\key,\cdot)$ bereitstellt:
  \begin{itemize}
  \item $\A$ kann sich zu jedem Zeitpunkt jedes beliebige
    $\plaint$ vom Verschlüsselungsorakel verschlüsseln lassen.
  \item $\A$ kann sich zu jedem Zeitpunkt jedes beliebige
    $\ciphert$ vom Entschlüsselungsorakel entschlüsseln lassen. Nachdem $\A$
    das Challenge-Chiffrat $C^*$ in Schritt 2 erhalten hat, darf er $C^*$
    nicht an das Orakel schicken.
  \end{itemize}
  \begin{enumerate}
  \item $\A$ wählt zwei Nachrichten $\plaint_1 \neq
    \plaint_2$ gleicher Länge.
  \item $\A$ erhält $\ciphert^{\ast} \coloneqq \enc(\key,
    \plaint_{b})$ für ein von $\C$ zufällig gleichverteilt gewähltes $b \in
    \{1, 2\}$.
  \item $\A$ gewinnt, wenn er $b$ korrekt errät.
  \end{enumerate} Ein symmetrisches Verfahren $(\enc, \dec)$ heißt
  IND-CCA-sicher, wenn der Vorteil des PPT-Algorithmus gegenüber dem Raten
  einer Lösung, also $\Pr \left[\A \textnormal{ gewinnt} \right] -
  \frac{1}{2}$, für alle PPT-Algorithmen $\A$ vernachlässigbar im
  Sicherheitsparameter $k$ ist.
\end{definition} Abbildung \ref{fig:ind-cca} stellt den Ablauf des
IND-CCA-Experimentes noch einmal grafisch dar.

\begin{figure}[h] \centering \scalebox{1.2}{
    \begin{tikzpicture}[x=2em, y=2em] \draw (-6,0) node (C) {$\C$};
      \draw (0,0) node (A) {$\A$}; \draw (7,0) node (Ork) {Orakel};
      
      \draw[dashed] (C) -- (-6,-7); \draw[dashed] (A) --
      (0,-8); \draw[dashed] (Ork) -- (7,-8);
      
      
      \draw[->, thick] (0.3,-1) -- (6.7,-1.5)
      node[sloped,above,pos=0.5] {$\plaint$ bzw. $\ciphert \neq
        \ciphert^{\ast}$}; \draw[->, thick] (6.7,-3) -- (0.3,-3.5)
      node[sloped,above,pos=0.5] {$\ciphert = \enc(\key, \plaint)$ bzw.}
      node[sloped,below,pos=0.5] {$\plaint = \dec(\key, \ciphert)$}; \draw
      (3.5,-5) node (exp) {(Anfragen immer erlaubt!)};
      
      \draw[->, semithick] (-0.3, -2.5) -- (-5.7,-3)
      node[sloped,above,pos=0.5] {$\plaint_1, \plaint_2$}; \draw (-7.5,-3.5)
      node (b1) {$ b \randUnif \{1,2\}$};
      
      \draw[->, thick] (-5.7, -4) -- (-.3,-4.5)
      node[sloped,above,pos=0.5] {$\ciphert^{*} = \enc(\key, \plaint_b)$};
      \draw[->, thick] (-0.3, -5.5) -- (-5.7,-6) node[sloped,above,pos=0.5]
      {$b'$};
      
      \draw (-5.8,-7.5) node (b2) {$ b = b'$?};
    \end{tikzpicture} }
  \caption{Ablauf des IND-CCA-Experiments.}
  \label{fig:ind-cca}
\end{figure}