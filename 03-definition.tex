\chapter{Kryptographische Sicherheitsbegriffe}
\section{Semantische Sicherheit}\label{ch:sicherheitsbegriffe:semantischesicherheit}
Nachdem wir uns bereits mit Verschlüsselungssystemen auseinandergesetzt haben, stellt sich natürlich die Frage, welche Form von Sicherheit wir erreichen möchten. 
Eines der primären Ziele war es bisher, dass ein Angreifer durch das Chiffrat keinerlei Informationen über den Klartext erhält. Dies entspricht dem Begriff der \emph{semantischen Sicherheit},
welcher 1983 von Goldwasser und Micali \cite{Goldwasser1984} definiert wurde und besagt konkret:
\begin{quote}
	\emph{Alles was mit $\ciphert$ (effizient) über $\plaint$ berechnet werden kann, kann auch ohne das Chiffrat berechnet werden.}
\end{quote}
Dabei ist zu beachten, dass diese Form von Sicherheit lediglich passive Angriffe abdeckt.

\begin{definition}[Semantische Sicherheit]\label{def:semsec}
Ein symmetrischer Verschlüsselungsalgorithmus ist semantisch sicher, wenn es für jede $\plaint$-Verteilung von Nachrichten gleicher Länge, jede Funktion $f$
und jeden \emph{effizienten} Algorithmus $\A$ einen \emph{effizienten} Algorithmus $\B$ gibt, so dass
\begin{align*}
	\text{Pr}\left[ \A^{\enc (\key,\cdot)} \left( \enc \left( \key, \plaint \right) \right) = f(\plaint) \right] - \text{Pr} \left[ \B (\epsilon) = f (\plaint) \right]
\end{align*}
\emph{klein} ist.
\end{definition}

Allerdings impliziert die Existenz von mehrfach benutzbaren, semantisch sicheren Verfahren damit $P \neq NP$. Das bedeutet, falls $P = NP$ gelten sollte, kann
es kein solches Verfahren geben. Außerdem ist diese Definition technisch schwer zu handhaben, da sie viele Quantoren enthält. Hierfür wurden handlichere
aber äquivalente Begriffe eingeführt wie beispielsweise \emph{IND-CPA}.

\section{Der IND-CPA-Sicherheitsbegriff}\label{def:ind-cpa}
IND-CPA steht für \emph{indistinguishability under chosen-plaintext attacks}. Bei einem Verfahren, welches diese Sicherheit besitzt, kann ein Angreifer $\A$
die Chiffrate von selbstgewählten Klartexten nicht unterscheiden. Um ein Verschlüsselungsverfahren $\enc$ auf die IND-CPA Eigenschaft zu überprüfen, führen wir folgendes Experiment, bei dem $\A$ Zugriff auf ein $\enc(\key,\cdot)$-Orakel\footnote{Ein Orakel kann man sich als \textit{black box} vorstellen, bei der der Fragende zwar das Ergebnis, jedoch nichts über dessen Berechnung in Erfahrung bringt.} hat, durch:
\begin{itemize}
\item $\A$ wählt zwei Nachrichten $\plaint_1$ und $\plaint_2$ gleicher Länge
\item $\A$ erhält $\ciphert^{*} := \enc(\key, \plaint_{b})$ für zufällig gleichverteilt gewähltes $b \in \{1, 2\}$.
\item $\A$ gewinnt, wenn er $b$ korrekt errät.
\end{itemize}
Ein Verfahren ist nun IND-CPA-sicher, wenn der Vorteil des Angreifers gegenüber dem Raten einer Lösung, also $\Pr \left[ \A \text{
gewinnt} \right] - \frac{1}{2}$, für alle Angreifer $\A$ vernachlässigbar klein ist. 
\begin{theorem}
Ein Verfahren ist genau dann semantisch sicher, wenn es IND-CPA-sicher ist.
\end{theorem}

\subsection{Beispiel ECB-Modus}
\begin{description} 
	\item[Behauptung] Keine Blockchiffre ist im ECB-Modus IND-CPA-sicher.
	\item[Beweis] Betrachte folgenden Angreifer $\A$:
	\begin{itemize}
		\item $\A$ wählt zwei Klartextblöcke $\plaint_1 \neq \plaint_2$ beliebig.
		\item $\A$ erhält  $\ciphert^{*} := \enc(\key, \plaint_{b})$ für zufällig gleichverteilt gewähltes $b \in \{1, 2\}$.
		\item $\A$ erfragt $\ciphert_1 = \enc(\key, \plaint_1)$ durch sein Orakel.
		\item $\A$ gibt 1 aus, genau dann, wenn $\ciphert_1 = \ciphert^{*}$, sonst gibt er 2 aus.
		\item $\Pr \left[ \A \text{ gewinnt} \right] = 1$, also ist das Schema nicht IND-CPA-sicher.
	\end{itemize}
\end{description}
Bei diesem Beispiel nutzt der Angreifer die Schwäche des ECB-Modus, dass gleiche Klartextblöcke immer zu gleichen Chiffrat-Blöcken werden, aus.

\subsection{Beispiel CBC-Modus}
Wie wir bereits im Abschnitt~\ref{ssec:cbc} gelernt haben, gilt bei der Verwendung des CBC-Modus: 
\begin{align*}
\ciphert_i = \enc(\key, \plaint_i \oplus \ciphert_{i-1}) \text{, mit } \ciphert_0 := IV.
\end{align*}
Im Folgenden nehmen wir an, dass der Initialisierungsvektor $IV$ bei jeder Verschlüsselung zufällig gleichverteilt gewählt wird. Somit wird verhindert, dass sich der Modus bei einem einzelnen Datenblock gleich dem ECB-Modus verhält. Im CBC-Modus existiert diese Problematik für längere Nachrichten, selbst bei festem und öffentlichem $IV$, generell nur für den ersten Datenblock, da darauffolgende Datenblöcke mit den vorausgehenden Chiffraten verkettet werden und somit kein direkter Zusammenhang mehr zwischen Daten- und Chiffratblock besteht. Allerdings werden bei der IND-CPA-Sicherheit zwei einzelne Blöcke verwendet, weshalb wir hier auf die Variante mit zufälligem $IV$ ausweichen.

Das zufällige Wählen eines $IV$ löst zwar ein Problem, erzeugt jedoch ein anderes: Um eine korrekte Entschlüsselung zu gewährleisten, muss der gewählte $IV$ mitgesendet werden. Das resultierende Problem ist nun, dass ein Angreifer eben diesen $IV$ verändern kann und somit beliebige Manipulationen am ersten Block einer bekannten Nachricht vornehmen kann.

\begin{description} 
	\item[Behauptung] Eine Blockchiffre ist im CBC-Modus genau dann IND-CPA-sicher, wenn die Verschlüsselungsfunktion $\enc(\key,\cdot): \{0,1\}^l \rightarrow
	\{0,1\}^l$ nicht von einer Zufallsfunktion $R: \{0,1\}^l \rightarrow \{0,1\}^l$ unterscheidbar ist.
	\item[Beweisidee]~
	\begin{description}
		\item[IND-CPA-sicher $\Rightarrow$ Ununterscheidbarkeit]~\\
		$\Leftrightarrow$ IND-CPA-unsicher $\Leftarrow$ Unterscheidbarkeit~\\
		Wenn ein Angreifer $\enc(\key, \cdot)$ von einer Zufallsfolge unterscheiden kann, ist zwischen mindestens zwei Verschlüsselungsergebnissen ein Zusammenhang erkennbar. Es gibt somit mindestens einen Fall, bei dem der Angreifer zusätzliche Informationen für das Zuordnen des Chiffrats besitzt. Daher gilt für zufällig gewählte Nachrichten im IND-CPA-Experiment: Pr$[\A$ gewinnt$] > \frac{1}{2} \Rightarrow$ IND-CPA-unsicher.
		\item[IND-CPA-sicher $\Leftarrow$ Ununterscheidbarkeit]~\\
		Wenn die Verschlüsselungsfunktion aus Sicht des Angreifers nicht von einer Zufallsfunktion unterscheidbar ist, gibt es keine bekannten Zusammenhänge der Verschlüsselungen. Somit ist die Wahrscheinlichkeit, dass der Angreifer ein Chiffrat korrekt zuordnet, genau $\frac{1}{2}$.
	\end{description}
\end{description}

\section{Der IND-CCA-Sicherheitsbegriff}
Der CPA-Angreifer ist mit Zugriff auf ein Verschlüsselungsorakel ausgestattet. Er kann sich jedmöglichen Klartext verschlüsseln lassen und
versuchen, Muster in den Ausgaben des Orakels zu erkennen. Eingeschränkt ist er dennoch, da ihm die Möglichkeit fehlt, zu beliebigen
Ciphertexten den Klartext zu berechnen. Ein stärkerer Sicherheitsbegriff ist daher IND-CCA. Ähnlich wie IND-CPA, steht IND-CCA abkürzend für \emph{indistinguishability under chosen-ciphertext attacks}. Dabei suggeriert das Akronym CCA bereits einen mächtigeren Angreifer. Das in \ref{def:ind-cpa} vorgestellte Experiment können wir problemlos auf einen IND-CCA-Angreifer anpassen. Modifiziert ergibt sich:

\begin{itemize}
	\item Der Angreifer $\A$ besitzt Zugriff auf ein $\enc(\key,\cdot)$-Orakel und ein $\dec(\key,\cdot)$-Orakel
	\item $\A$ wählt zwei Nachrichten $\plaint_1$ und $\plaint_2$ gleicher Länge
	\item $\A$ erhält $\ciphert^{*} := \enc(\key, \plaint_{b})$  für ein zufällig gleichverteilt gewähltes $b \in \{1, 2\}$.
	\item $\A$ gewinnt, wenn er $b$ korrekt errät.
\end{itemize} 

Natürlich darf der Angreifer den Klartext zum Chiffrat $\ciphert^{*}$ nicht bei dem Entschlüsselungsorakel erfragen. Ein Angriff wäre ansonsten trivial und es gäbe kein Verfahren, dass dem IND-CCA-Sicherheitsbegriff genügt. 

%Wieso braucht man hier das hardcoded Leerzeichen und oben nicht?%
Analog heißt ein Verfahren nun IND-CCA-sicher, wenn der Vorteil des Angreifers gegenüber dem Raten einer Lösung, also $\text{Pr} \left[ \A\ \text{gewinnt} \right] - \frac{1}{2}$, für alle Angreifer vernachlässigbar klein ist.

Etwas granularer kann bei IND-CCA zwischen einer adaptiven und einer nichtadaptiven Variante unterschieden werden. Erstere erlaubt dem Angreifer weitere Berechnungen, auch nachdem $\ciphert^{*}$ bereits erhalten wurde, wohingegen die nichtadaptive Variante solche Berechnungen explizit verbietet. In der Theorie kann sich ein adaptiver Angreifer also auch von $\ciphert^{*}$ abhängige Ciphertexte entschlüsseln lassen. Sprechen wir in diesem Skript von IND-CCA, so gehen wir von der adaptiven Variante aus.

\section{Sicherheitsparameter und effiziente Angreifer}\label{sec:secparam}
Zu einer Funktion, für die sicherheitsrelevante Eigenschaften gefordert werden, wird in der Kryptographie oft ein \emph{Sicherheitsparameter} $k$ definiert. Informell gesagt, legt $k$ das Sicherheitsniveau der Funktion fest. Beispielsweise parametrisiert er den Bildbereich einer \hyperref[cha:hash]{Hashfunktion}, wo ein größerer Bildbereich bei gleichem Urbildbereich zu einer geringeren Kollisionswahrscheinlichkeit führt.
\begin{beispiel}
	Um die Intuition rechnerisch zu belegen, definieren wir eine Funktion
	\begin{align*}
		H : \{0, 1\}^{1024} \rArrow \{0, 1\}^{k}\, ,
	\end{align*}
	wobei, falls $k = 512$,
	\begin{align*}
		\Pr \left[ \exists x' \neq x : H(x') = H(x) \mid x \stackrel{\textdollar}{\longleftarrow} \{0, 1\}^{1024} \right] > \frac{1}{2}
	\end{align*}
	und falls $k = 256$,
	\begin{align*}
		\Pr \left[ \exists x' \neq x : H(x') = H(x) \mid x \stackrel{\textdollar}{\longleftarrow} \{0, 1\}^{1024} \right] > \frac{3}{4}
	\end{align*}
\end{beispiel}
Ein weiteres Beispiel für die Anwendung des Sicherheitsparameters findet sich bei symmetrischen Verschlüsselungsverfahren, bei welchen die durch $k$ festgelegte Länge des Schlüssels den Aufwand einer Brute-Force-Attacke bestimmt. Dabei wird $k$ vor der Schlüsselerzeugung ausgehandelt und es kann davon ausgegangen werden, dass eine ehrliche Partei sich an die Abmachung hält.
Die Sicherheit des Verfahrens darf allerdings nicht von der Geheimhaltung des Sicherheitsparameters abhängen. 

Zur Analyse der Sicherheitseigenschaften eines kryptographischen Verfahrens betrachtet man üblicherweise einen Angreifer, der \emph{effizient}, das heißt in seiner Rechenzeit eingeschränkt ist. Intuitiv wird in der Komplexitätstheorie ein effizienter Algorithmus mit einer polynomial-beschränkten Laufzeit in der Eingabegröße gleichgesetzt. 
In anderen Worten ist ein Algorithmus bei Eingabe eines Bit-Strings der Länge $n$ genau dann effizient, wenn es ein $c \in \mathbbm{N}$, so dass seine Laufzeit im schlechtesten Fall in $O(n^c)$ liegt.
Um mit dieser Idee konsistent zu bleiben, wird in der Kryptographie einem Angreifer auf ein Verfahren ein Bit-String $1^k$ von $k$ Einsen übergeben, um die Rechenzeit durch den zugrundeliegenden Sicherheitsparameter zu begrenzen. 
Ein effizienter Angreifer muss also in $O(k^c)$-Schritten, $c \in \mathbbm{N}$, eine Lösung berechnen. Somit ist beispielsweise die eingangs erwähnte Brute-Force-Attacke auf einen Schlüsselraum $\{0, 1\}^{k}$ ausgeschlossen, da die Laufzeit in $O(2^{k})$ liegt, also exponentiell ist.
Neben der Begrenzung der Rechenzeit erlauben wir einem Angreifer, um ein realistisches Szenario abzubilden, zu jedem Zeitpunkt eine \emph{faire} Münze zu werfen, das heißt, eine Lösung zu raten. Einen solchen Angreifer bezeichnen wir als \emph{probabilistic polynomial time} (PPT) Angreifer.
%Was ist mit den Längen der anderen Eingaben, die ein Angreifer bekommt?

Damit ein Verschlüsselungsverfahren grundsätzlich als sicher angenommen werden kann, muss die Erfolgswahrscheinlichkeit eines Angreifers möglichst klein sein und darf idealerweise nur vom Zufall abhängen.
Da der Sicherheitsparameter für jedes Verfahren beliebig variiert werden kann, erscheint es intuitiv logisch, die Sicherheit nicht in Abhängigkeit der Größe von $k$ zu definieren. Vielmehr fordern wir für eine bestimmte Sicherheitseigenschaft, dass ein PPT Angreifer die Charakteristik nur mit \emph{vernachlässigbarer} Wahrscheinlichkeit in $k$ brechen kann.
\begin{definition}[Vernachlässigbarkeit]
	Es sei $f$ eine Funktion, $k$ eine Konstante und $p(\cdot)$ ein beliebiges Polynom. Dann heißt $f$ vernachlässigbar in $k$, wenn
	\begin{align*}
		\exists k_0 : \forall k \geq k_0 : \vert f(k) \vert \leq \frac{1}{p(k)} 
	\end{align*}
\end{definition}
Beispielsweise ist $f = \frac{1}{2^k}$ vernachlässigbar in $k$, $f = \frac{1}{k^2}$ jedoch nicht.

Dass ein Verfahren in der Praxis bei zu klein gewähltem Sicherheitsparameter, trotz vernachlässigbarer Erfolgswahrscheinlichkeit eines PPT Angreifers, keine Sicherheit bieten kann, ist trivial.