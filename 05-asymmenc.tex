\chapter{Asymmetrische Verschlüsselung}
\label{ch:asymmenc}

Symmetrische Verschlüsselung, wie wir sie in den letzten Kapiteln behandelt haben, funktioniert über ein gemeinsames Geheimnis $K$ (siehe Abbildung
\ref{fig:symmenc}).
Das verursacht uns einige Unannehmlichkeiten:

\begin{itemize}
  \item das gemeinsame Geheimnis $K$ muss auf einem sicheren Kanal übertragen werden
  \item bei $n$ Benutzern werden im System $\binom{n}{2} = \frac{n \cdot (n-1)}{2}$ Schlüssel verwendet  (für jedes Teilnehmerpaar einen)
\end{itemize}

\begin{figure}[h]
\begin{center}
\unitlength=1mm
\linethickness{0.4pt}
\hspace{-3 cm}
\begin{picture}(30,10)
\put(0,2){\makebox(0,0)[cb]{$\text{Alice}_K$}}
\put(50,3){\vector(-1,0){40}}
\put(30,4){\makebox(0,0)[cb]{$C := \text{Enc}(K, M)$}}
\put(55,0.5){\makebox(10,5){$\text{Bob}_K$}}
\end{picture}
\end{center}
\caption{schematischer Ablauf einer symmetrisch verschlüsselten Kommunikation}
\label{fig:symmenc}
\end{figure}


\section{Idee}
Public-Key-Kryptographie basiert auf der Grundidee, für die Verschlüsselung (öffentlich) einen anderen Schlüssel zu verwenden als für die Entschlüsselung
(privat). Abbildung \ref{fig:asymmenc} zeigt den Ablauf einer asymmetrisch verschlüsselten Kommunikation.

Die Vorteile eines Public-Key-Verfahrens sind offensichtlich. Wir benötigen für den Schlüsselaustausch keinen sicheren Kanal mehr, sondern könnten sogar ähnlich
einem Telefonbuch ein öffentliches Verzeichnis mit den öffentlichen Schlüsseln anlegen. Außerdem müssen nicht mehr so viele Schlüssel gespeichert werden: Bei
$n$ Benutzern gibt es nur noch $n$ öffentliche (und $n$ geheime) Schlüssel.

Die Sicherheit eines solchen Verfahrens hängt davon ab, wie schwierig es für einen Angreifer ist, vom (allgemein bekannten) öffentlichen Schlüssel $pk$ auf den
(geheim gehaltenen) privaten Schlüssel $sk$ zu schließen. Um das praktisch unmöglich zu machen, werden Probleme aus der Mathematik verwendet, die anerkannt
schwierig zu lösen sind.

\begin{figure}[h]
\begin{center}
\unitlength=1mm
\linethickness{0.4pt}
\hspace{-3 cm}
\begin{picture}(30,10)
\put(0,2){\makebox(0,0)[cb]{$\text{Alice}_{sk}$}}
\put(50,3){\vector(-1,0){40}}
\put(30,4){\makebox(0,0)[cb]{$C := \text{Enc}(pk, M)$}}
\put(55,0.5){\makebox(10,5){$\text{Bob}_{pk}$}}
\end{picture}
\end{center}
\caption{schematischer Ablauf einer asymmetrisch verschlüsselten Kommunikation}
\label{fig:asymmenc}
\end{figure}

\section{RSA}
Das bekannteste Public-Key-Verfahren ist RSA (1977). Benannt nach seinen Erfindern Rivest, Shamir und Adleman macht es sich den enormen Aufwand zunutze,
eine Zahl in ihre Primfaktoren zu zerlegen.

\subsection{Vorgehen}
\label{ch:asymmenc:rsa:vorgehen}
Für die Erstellung eines Schlüsselpaares werden zwei große Primzahlen benötigt. Die Berechnung von öffentlichem und privatem Schlüssel funktioniert
folgendermaßen:

\begin{itemize}
  \item wähle zwei große Primzahlen $P, Q$ mit $P \not = Q$ und vorgegebener Bitlänge $k$
  \item berechne $N = P \cdot Q$
  \item setze $\varphi(N) = (P - 1)(Q - 1)$
  \item wähle $e$ mit $\text{ggT}(e, \varphi(N)) = 1$\footnote{Ist $\text{ggT}(e, \varphi(N)) = 1$ kann das zu $e$ multiplikativ inverse Element $d$, also $e \cdot d \equiv 1 \mod \varphi(N)$, mit Hilfe des erweiterten euklidischen Algorithmus ermittelt werden. Für zwei Zahlen $a$ und $b$ berechnet der erweiterte euklidische Algorithmus die Koeffizienten $x$ und $y$, so dass $ax + by = \text{ggT}(a, b)$. Sind $a$ und $b$ teilerfremd ist $ax + by = 1 \Leftrightarrow a \cdot x \equiv 1 \mod \varphi(b)$, also $x$ das zu $a$ multiplikativ inverse Element bezüglich dem Modulus $\varphi(b)$.}
  \item wähle $d$, sodass $d$ und $e$ zueinander invers, also $e \cdot d \equiv 1 \mod \varphi(N)$
  \item setze den geheimen Schlüssel $sk = (N, d)$ und den öffentlichen Schlüssel $pk = (N, e)$
\end{itemize}

Üblicherweise werden $P$ und $Q$ zufällig gleichverteilt aus den ungeraden Zahlen der Länge $k$ gezogen, bis $P$ und $Q$ prim sind. Aus $e \in \{3, \dotsc ,
\varphi(N) - 1\}$ wird dann mit dem erweiterten euklidischen Algorithmus das multiplikative Inverse berechnt: $d \equiv e^{-1} \mod \varphi(N)$. Der Nachrichtenraum ist $\mathcal
M := \Z N$. Für die Ver- und Entschlüsselungsfunktionen gilt:
\begin{align*}
& \enc(pk, M) = M^e \mod N\\
& \dec(sk, C) = C^d \mod N
\end{align*}

Wie immer muss $\dec(\enc(M)) = M$ gelten. Für die Korrektheit von RSA bedeutet das, dass $(M^e)^d \equiv M^{ed} \equiv M \pmod N$ erfüllt sein muss. Um das zu beweisen,
verwenden wir den Kleinen Satz von Fermat und den Chinesischen Restsatz.

\vspace{10pt}
\begin{theorem}[Kleiner Satz von Fermat]
Für primes $P$ und $M \in \{1, \dotsc, P-1\}$ gilt: $M^{P-1} \equiv 1 \mod P$.
\end{theorem}

~\\
Daraus folgt auch: $\forall M \in \Z P, \alpha \in \mathbbm{Z} : (M^{P-1})^{\alpha} \cdot M \equiv M \mod P$.

\vspace{10pt}
\begin{theorem}[Chinesischer Restsatz]
Sei $N = P \cdot Q$ mit $P, Q$ teilerfremd. Dann ist die Abbildung $\mu : \Z N \rightarrow \Z P \times \Z Q$ mit $\mu(M) \equiv (M \mod P, M \mod Q)$ bijektiv.
\end{theorem}

~\\
Daraus folgt auch: $(X \equiv Y \mod P) \land (X \equiv Y \mod Q) \Rightarrow X \equiv Y \mod N$.

\vspace{10pt}
\begin{theorem}[Korrektheit von RSA]
Sei $N = P \cdot Q$ mit $P, Q$ teilerfremd und prim. Seien weiter $e, d$ teilerfremd wie oben. Dann ist $M^{ed} \equiv M \mod N$ für alle $M \in \Z N$.
\end{theorem}
\vspace{10pt}

\begin{beweis}
Nach Definition gilt $e \cdot d \equiv 1 \mod (P-1)(Q-1)$. Daraus folgt:
\begin{align*}
(P-1)(Q-1) \mid ed - 1 \quad
& \Rightarrow \quad P-1 \mid ed - 1\\
& \Rightarrow \quad ed = \alpha (P-1) + 1 \quad (\text{für } \alpha \in \mathbbm{Z})\\
& \Rightarrow \quad M^{ed} = (M^{(P-1)})^{\alpha} \cdot M \stackrel{\text{Fermat}}\equiv M \mod P
\end{align*}
Analog ist $M^{ed} \equiv M \mod Q$.\\
Da $N = P \cdot Q$ ergibt sich mithilfe des Chinesischen Restsatzes:
\begin{align*}
(M^{ed} \equiv M \mod P) \land (M^{ed} \equiv M \mod Q) \Rightarrow M^{ed} \equiv M \mod N
\end{align*}
\qed
\end{beweis}

~\\
Das bisher behandelte Verfahren nennt sich \textit{Textbook-RSA} und umfasst das grundlegende Prinzip von RSA. Textbook-RSA weist einige Schwächen auf und
sollte daher in der Praxis nicht verwendet werden.


\subsection{Sicherheit von RSA}
\label{ch:asymmenc:rsa:sicherheit}
Bevor wir die Sicherheit von RSA betrachten, benötigen wir einen Sicherheitsbegriff, an dem wir uns bei der Beurteilung von asymmetrischen
Verschlüsselungsverfahren orientieren können. Wir definieren semantische Sicherheit, vergleichbar mit der Definition für symmetrische Chiffren in Kapitel
\ref{ch:sicherheitsbegriffe:semantischesicherheit} und äquivalent zu IND-CPA.

\vspace{10pt}
\begin{definition}[Semantische Sicherheit für Public-Key-Verfahren]
Ein Pub\-lic-""Key-""Ver\-schlüs\-sel\-ungs\-sche\-ma ist \textit{semantisch sicher}, wenn es für jede $M$-Verteilung von Nachrichten gleicher Länge, jede
Funktion $f$ und jeden PPT-Algorithmus $\A$ einen PPT-Algorithmus $\B$ gibt, so dass
\begin{equation*}
\Pr\left[\A(1^k, pk, \enc(pk, M)) = f(M)\right] - \Pr\left[\B(1^k) = f(M)\right]
\end{equation*}
vernachlässigbar (als Funktion im Sicherheitsparameter) ist.
\end{definition} 

~\\
Umgangssprachlich formuliert bedeutet semantische Sicherheit, dass jeder Angreifer über ein Chiffrat $C$ nur die Länge der Eingabe lernt.

RSA ist deterministisch, d.h. eine Nachricht $M$ wird unter Verwendung desselben Schlüssels $pk$ immer zu $C_M$ verschlüsselt. Ein Angreifer kann zwei Chiffrate effizient voneinander unterscheiden (z.B. $\enc(pk, \text{annehmen})$ und $\enc(pk, \text{ablehnen})$). RSA ist also nicht semantisch sicher. Die Sicherheit von RSA beruht darauf, dass das Faktorisieren in Primzahlen nicht effizient berechenbar ist, das heißt es ist kein Algorithmus bekannt, der dieses Problem in Polynomialzeit löst. Diese Überlegungen basieren auf der nicht-bewiesenen Annahme $P \neq NP$.
Es gibt noch einige andere Angriffspunkte, die im Folgenden umrissen werden.

\begin{description}
    \item[Wahl von $e$:] Aus Effizienzgründen liegt es auf den ersten Blick nahe, den Parameter $e$ aus dem öffentlichen Schlüssel nicht für jeden Benutzer neu
    zu berechnen, sondern für alle gleich zu wählen. Da diese Wahl nur den öffentlichen Schlüssel betrifft, scheint diese Einschränkung nicht kritisch zu sein,
    führt jedoch zu Problemen, wenn dieselbe Nachricht $M$ an mindestens drei unterschiedliche Benutzer verschlüsselt gesendet wird. Setzen wir für dieses Beispiel $e =
    3$. Ein Angreifer, der die drei öffentlichen Schlüssel $pk_1, pk_2, pk_3$ kennt, mit denen $M$ verschlüsselt wurde, kann sich die Nachricht $M$ berechnen:
    \begin{align*}
    & M^3 \mod N_i  && \text{für } 1 \leq i \leq 3\\
    & \equiv M^3 \mod N_1N_2N_3  && \text{(Chinesischer Restsatz)}\\
    & \equiv M^3 && (\text{wegen } 0 \leq M < N_1,N_2,N_3)
    \end{align*}
    Wurzelziehen über $\mathbbm{Z}$ liefert die Nachricht $M$.
    
    \item[Wahl von $N$:] Auch $N$ für alle Benutzer gleich zu vergeben schwächt das Verschlüsselungssystem. Wird wieder dieselbe Nachricht $M$ mit zwei
    öffentlichen Schlüsseln $(e_1, N)$ und $(e_2, N)$ chiffriert und gilt weiterhin $\text{ggT}(e_1, e_2) = 1$ in $\Z{}$, kann ein Angreifer aus den Chiffraten
    $M$ berechnen:
    \begin{align*}
    re_1 + se_2 & = 1\\ 
    \Longrightarrow C_1^rC_2^s \mod N & \equiv M^{re_1}M^{se_2} \mod N\\
    & \equiv M^{re_1 + se_2} \mod N\\
    & \equiv M
    \end{align*}
    Es besteht in diesem Szenario die noch gravierendere Gefahr, dass ein Teilnehmer $A$ aus seinem eigenen öffentlichen Schlüssel $pk_A = (e_A, N)$ und
    dem eines Teilnehmers $B$ $pk_B = (e_B, N)$ ein $d'_B$ berechnen kann, das äquivalent zu $d_B$ aus $B$s privatem Schlüssel ist. $A$ ist also durch einfache
    Anwendung des Erweiterten Euklidischen Algorithmus in der Lage, sich ein $sk'_B = (d'_B, N)$ zu erstellen, mit dem sie alle an $B$ verschlüsselten    
    Nachrichten dechiffrieren kann.
    
    \item[Homomorphie:] Anhand 
     \begin{align*}
     \enc(pk, M_1) \cdot \enc(pk, M_2)
     &= (M_1^e \text{ mod } N) \cdot (M_2^e \text{ mod } N)\\
     &= (M_1^e \cdot M_2^e) \mod N\\
     &= \enc(pk, M_1 \cdot M_2)
     \end{align*}
    
    sehen wir, dass RSA homomorph ist. 
    Folgendes Beispiel veranschaulicht, zu welchen Zwecken die Homomorphie ausgenutzt werden kann:
    \begin{beispiel}
    	Wir betrachten eine Auktion mit dem Auktionsleiter $A$ und zwei Bietern $B_1$ und $B_2$. Damit keiner der Interessenten einen
    	anderen knapp überbietet oder sich von den Geboten anderer in seiner eigenen Abgabe beeinflussen lässt, nimmt der Auktionator die Gebote verschlüsselt entgegen. Dafür hat er seinen öffentlichen Schlüssel $pk_A$ zur Verfügung gestellt. Das Gebot eines Bieters wird chiffriert und zur Aufbewahrung an den Auktionator geschickt. Wenn die Zeit abgelaufen ist, werden keine neuen Preisvorschläge mehr angenommen, die eingegangenen Gebote entschlüsselt und der Höchstbietende ermittelt.
    
    	Der unehrliche Bieter $B_2$ kann nun seinen Preisvorschlag mithilfe des verschlüsselten Gebots von $B_1$ zu seinen Gunsten wählen. Dafür setzt er z.B. $C_2 =
    	C_1 \cdot \enc({pk_A, 2})$ oder, wenn er besonders sparsam ist, $C_2 = C_1 \cdot \enc({pk_A, 1001/1000 \mod N})$. Damit kann er das Gebot von $B_1$ verdoppeln
    	bzw. knapp überbieten, ohne dass der Auktionator und der ehrliche Bieter $B_1$ ihm Betrug nachweisen können.
    \end{beispiel}
    
    
%    \begin{figure}[h]
%    \begin{center}
%    \unitlength=1mm
%    \linethickness{0.4pt}
%    \hspace{-3 cm}
%    \begin{picture}(60,30)
%    
%    \put(40,25){\makebox(0,0)[cb]{$\text{Auktionator } A$}}
%    
%    \put(10,0){\makebox(0,0)[cb]{$\text{Bieter } B_1$}}
%    \put(10,13){\makebox(0,0)[cb]{$C_1 = \enc(pk_A, b_1)$}}
%    \put(18,5){\vector(1,1){18}}
%    
%    \put(70,0){\makebox(0,0)[cb]{$\text{Bieter } B_2$}}
%    \put(70,13){\makebox(0,0)[cb]{$C_2 = \enc(pk_A, b_2)$}}
%    \put(62,5){\vector(-1,1){18}}
%    
%    \end{picture}
%    \end{center}
%    \caption{schematischer Ablauf einer Auktion mit verschlüsselten Geboten}
%    \label{fig:auktion}
%    \end{figure}
\end{description} 

\subsection{Sicheres RSA}
Wir haben festgestellt, dass RSA deterministisch und damit nicht semantisch sicher ist. Die gepaddete RSA-Variante RSA-OAEP dagegen ist IND-CCA-sicher. Wir
verwenden dabei eine Zufallszahl $R$, mit deren Hilfe wir die Nachricht $M$ vor dem Verschlüsseln abwandeln. Zu diesem Zweck wird die in Abbildung
\ref{fig:rsa-oaep} dargestellte Konstruktion von Hashfunktionen $G, H$ verwendet. Wir können $R$ nach dem Entschlüsseln wieder entfernen, aber $\enc_R(M)$
lässt sich nun nicht mehr so einfach mit anderen Chiffraten abgleichen.
Diese Konstruktion ist heuristisch genau so sicher, wie $N$ zu faktorisieren.

\begin{figure}[h]
    \begin{center}
    \unitlength=1mm
    \linethickness{0.4pt}
    \hspace{-3 cm}
        \begin{picture}(60,60)
        
        \put(0,50){\framebox(30,5){$m$}}
        \put(32,50){\framebox(15,5){$000$}}
        \put(55,50){\framebox(20,5){$R$}}
        
        \put(15,45){\line(0,1){5}}
        \put(39,45){\line(0,1){5}}
        \put(15,45){\line(1,0){24}}
        \put(25,45){\vector(0,-1){40}}
        
        \put(65,50){\vector(0,-1){45}}
                
        \put(45,35){\circle{7}}
        \put(45,34){\makebox(0,0)[cb]{$G$}}
        \put(25,35){\circle{4}}
        \put(23,35){\line(1,0){18.5}}
        \put(65,35){\vector(-1,0){16.5}}
        
        \put(45,20){\circle{7}}
        \put(45,19){\makebox(0,0)[cb]{$H$}}
        \put(65,20){\circle{4}}
        \put(25,20){\vector(1,0){16.5}}
        \put(48.5,20){\line(1,0){18.5}}
        
        \put(0,0){\framebox(45,5){$X$}}
        \put(55,0){\framebox(20,5){$Y$}}
            
        \end{picture}
    \end{center}
    \caption{pad-Funktion von RSA-OAEP ($G,H$ sind Hashfunktionen)}
    \label{fig:rsa-oaep}
\end{figure}

\subsection{Bedeutung von RSA}
Im Gegensatz zu den meisten symmetrischen Chiffren basiert RSA als Beispiel einer asymmetrischen Verschlüsselungstechnik nicht auf einfachen, bit-orientierten
sondern auf einer mathematischen Funktion. Der für Ver- und Entschlüsselung sowie für die Schlüsselerzeugung nötige Rechenaufwand steigt dadurch ungemein: ein
naiver Exponentiationsalgorithmus benötigt für die Berechnung einer modulo $l$-Bit-Zahl $\omega(l)$ Bitoperationen.

Nichtsdestotrotz wird RSA in der Praxis häufig eingesetzt. Es macht sich relativ relativ einfache Arithmetik zunutze und die Ähnlichkeit zwischen Ver- und
Entschlüsselungsfunktion vereinfachen die Implementierung zusätzlich. Mit einfachen Anpassungen ($e = 3$ bei Verschlüsselung, Chinesischer Restsatz nutzen bei Entschlüsselung)
kann RSA so weit beschleunigt werden, dass es die Laufzeit betreffend gegenüber anderen Verschlüsselungsverfahren konkurrenzfähiger wird.

\section{ElGamal}
\label{ch:asymenc:elgamal}
Das ElGamal-Verfahren (1985) macht sich die Schwierigkeit zunutze, das Diffie-Hellman-Problem, also die Berechnung von diskreten Logarithmen in zyklischen Gruppen zu lösen. Unter einer zyklischen Gruppe versteht man eine Gruppe $\mathbbm{G}$, bei der ein sogenanntes Erzeugerelement $g$ existiert, so dass $\mathbbm{G} = \langle g \rangle := \{g^k \mid k \in \mathbbm{Z}\}$.

\subsection{Vorgehen}
Für die Erzeugung der Schlüssel benötigen wir eine angemessen große, zyklische Gruppe $\mathbbm{G}$ mit dem Erzeuger $g$.
Geeignete Kandidaten für eine solche Gruppe sind (echte) Untergruppen von $\mathbbm{Z}^*_p$ mit $p$ prim oder allgemeiner Untergruppen von $\mathbbm{F}^*_q$ mit $q$ Primpotenz mit einer Gruppengröße von $|\mathbbm{G}| \approx 2^{2048}$. Effizienter sind Untergruppen von elliptischen Kurven
$\boldsymbol{\mathsf{E}}(\mathbbm{F}^*_q)$ mit einer Gruppengröße von $|\mathbbm{G}| \approx 2^{200}$. Wir betrachten das Verfahren beispielhaft für eine Untergruppe von $\mathbbm{Z}^*_p$ mit Ordnung $q$, weshalb alle Operationen auf der Gruppenstruktur, also modulo $p$, berechnet werden.

Wir wählen außerdem eine Zufallszahl $x \in \{2, \dots, q - 1\}$, so dass $x$ und $q$ teilerfremd und berechnen $h \equiv g^x$. Bemerke, dass in Gruppen primer Ordnung jedes Element der Menge gewählt werden kann. In einer Gruppe, die zusätzlich auf Äquivalenzklassen arbeitet, kann aufgrund der Modulorechnung $x \in \mathbbm{Z}$ sogar zufällig gewählt werden. Für unser Schlüsselpaar gilt damit: 
\begin{align*}
  (pk, sk) = ((\mathbbm{G}, g, h), (\mathbbm{G}, g, x))
  %pk = (\mathbbm{G}, g, h),\ sk = (\mathbbm{G}, g, x)
\end{align*}
Wenn Alice uns eine Nachricht $M \in \mathbbm{G}$ schicken möchte, wählt sie analog zu $x$ ein $y$ zufällig gleichverteilt, berechnet damit $C \equiv h^y M$ und sendet das Tupel $(g^y ,C)$. Wir können die Nachricht entschlüsseln, indem wir auflösen:
\begin{align*}
C &\equiv h^y M \\
\Leftrightarrow \quad M& \equiv \frac{C}{h^y}
 \equiv \frac{C}{g^{xy}}
 \equiv \frac{C}{(g^y)^x}
\end{align*}
Für Ver- und Entschlüsselung gilt also:
\begin{align*}
  &\enc(pk, M) = (g^y, h^y \cdot M) \\
  &\dec(sk, (g^y, C)) = \frac{C}{(g^y)^x}
\end{align*}
Durch die zufällige Wahl von $y$ ist das Chiffrat $\enc({pk, M})$ randomisiert. ElGamal ist somit semantisch sicher. Allerdings ist ElGamal wie RSA homomorph:
\begin{align*}
\enc({pk,M}) \cdot \enc({pk,M'})
&= (g^y, g^{xy} \cdot M) \cdot (g^{y'}, g^{xy'} \cdot M')\\
&= (g^{y+y'}, g^{x(y + y')} \cdot M \cdot M')\\
&= \enc({pk, M \cdot M'})
\end{align*}
Es existieren allerdings bereits nicht-homomorphe Varianten von ElGamal. %TODO Verweis auf so ein Verfahren?

\subsection{Erweiterung des Urbildraums}
Ein Problem des klassischen ElGamal-Verfahrens ist, dass nur Nachrichten $M \in \mathbbm{G}$ verschlüsselt werden können. In der Praxis sind jedoch die meisten Nachrichten ausserhalb der gewählten Gruppe, weshalb die Korrektheit der notwendigen Operationen nicht garantiert werden kann. Es existieren jedoch verschiedene Ansätze, dieses Problem zu lösen und den Raum möglicher Nachrichten flexibler zu gestalten.

\subsubsection{Nachrichtenumwandlung}
Die Nachrichtenumwandlung erlaubt es, beliebige Nachrichten fester Länge zu verschlüsseln, ohne den eigentlichen Algorithmus anpassen zu müssen. Die Länge der möglichen Nachrichten wird dabei durch die Größe der zugrundeliegenden Gruppe festgelegt.

\paragraph*{Verfahren}
Im Folgenden werde $M$ zunächst als Bit-String aufgefasst. Wir wählen $p > 2 $ prim und setzen $\mathbbm{G} \subset \mathbbm{Z}^*_p$ als Untergruppe der Quadrate von $\mathbbm{Z}^*_p$, wobei $\mathbbm{G}$ die Ordnung $q = \frac{(p - 1)}{2}$ hat.\footnote{Die Untergruppe der Quadrate von $\mathbbm{Z}^*_p$ besteht aus den Elementen $\{y = x^2 \text{ mod } p\ \vert\ x \in \mathbbm{Z}^*_p\}$. Falls $p > 2$ prim ist, besteht diese Untergruppe aus $\frac{p - 1}{2}$ Elementen. Jedes Element, mit Ausnahme der Eins, kann als Gruppengenerator dienen.}
Es sei $n$ die Länge des Bit-Strings der Gruppenordnung $q$. Dann können wir die Nachricht $M \in \{0, 1\}^{n - 1}$ beliebig wählen und interpretieren sie im weiteren Verlauf als ganze Zahl äquivalent zu ihrer Binärdarstellung. Da $M$ auch die Null darstellen kann und die Null in multiplikativen Gruppen nicht vorhanden ist, setzen wir $\tilde{M} = M + 1$. Folglich ist $ 1 \leq \tilde{M} \leq q$ und daher $\tilde{M} \in \mathbbm{Z}^*_p$. Nach der Eigenschaft einer quadratischen Untergruppe ist somit $\hat{M} = \tilde{M}^2 \text{ mod } p \in \mathbbm{G}$. 

Damit kann $\hat{M}$ analog zum obigen Verfahren verschlüsselt werden. Zum Entschlüsseln berechnet der Empfänger aus $\hat{M}$ als Zwischenschritt $\tilde{M} = \sqrt{\hat{M}}\ \text{mod}\ p\ \in [1, q]$ und erhält mit $M = \tilde{M} - 1$ die ursprüngliche Nachricht $M$ in der Binärdarstellung. $\hat{M}$ ist durch normales Entschlüsseln mit ElGamal zu berechnen.

Ein Nachteil dieses Verfahrens ist, dass die Nachrichtenumwandlung, je nach gewählter Gruppe, nicht effizient möglich ist.

\subsubsection{Hash-ElGamal}
Eine weitere Variante, die Einschränkung der Nachrichten auf Elemente der gewählten Gruppe aufzuheben, ist das Hash-ElGamal-Kryptosystem. Es realisiert ein Verfahren, dass zu allen Nachrichten $M \in \{0, 1\}^l$ mit Hilfe der bereits bekannten Bausteine und einer Hashfunktion ein Chiffrat der gleichen Länge bestimmt. Im Gegensatz zur Nachrichtenumwandlung bilden wir $M$ dabei nicht auf die Gruppe ab. Die Sicherheit des Kryptosystems beruht dabei ausschließlich auf der Annahme, dass der diskrete Logarithmus nicht effizient berechnet werden kann und ist nicht abhängig von der Wahl der Hashfunktion. Das Hash-El-Gamal-Verfahren bietet somit Sicherheit auf gleichem Niveau, ist in der Verwendung, aufgrund des größeren Urbildraums, jedoch deutlich flexibler.

\paragraph*{Verfahren}
Es seien die Gruppe $\mathbbm{G} \subset \mathbbm{Z}^*_p$ und das Schlüsselpaar $(pk,sk)$ analog zu ElGamal gewählt und berechnet. Sei zudem $H \colon \mathbbm{G} \rightarrow \{0,1\}^l$ eine beliebige Hashfunktion, die in Bitfolgen der Länge $l$ abbildet.

Wähle, um eine Nachricht $M \in \{0,1\}^l$ zu verschlüsseln, $y \leftarrow \mathbbm{Z}_p$ zufällig gleichverteilt, berechne $Y = g^y\ \text{mod}\ p$ und sende das Tupel
\begin{align*}
(Y, H(h^y) \oplus M) = (Y, C)
\end{align*}

Unter zuhilfenahme des privaten Schlüssels $sk = (\mathbbm{G}, g, x)$ kann der Ursprungstext $M$ aus dem Chiffrat-Tupel zurückgerechnet werden:
\begin{align*}
M = H(Y^x) \oplus C
\end{align*}

\section{Fazit}
Wie wir bei RSA und ElGamal gesehen haben, ist der Nachrichtenraum abhängig von den gewählten Primzahlen beziehungsweise der gewählten Gruppe. Um sehr große Nachrichten zu verschlüsseln, werden beide Verfahren folglich schnell sehr unhandlich und deswegen in der Praxis dafür nicht verwendet. Zur Nachrichtenverschlüsselung werden bevorzugt die in \hyperref[cha:symencryption]{Kapitel 2} vorgestellten symmetrischen Kryptosysteme benutzt, die zusätzlich um Größenordnungen schneller sind als die asymmetrischen Verfahren. In Anwendungen, wo beliebig große Nachrichten verschlüsselt werden sollen, aber ein problematischer Schlüsselaustausch vorliegt, werden häufig symmetrische und asymmetrische Verfahren zu Hybridverschlüsselungen kombiniert. Dabei wird zuerst der symmetrische Schlüssel über asymmetrische Kryptosysteme ausgetauscht und anschließend für die symmetrische Verschlüsselung der Nachrichten verwendet.\footnote{Hybridverschlüsselungen, wie sie z.B das Protokoll \textit{Transport Layer Security} (TLS) verwendet, werden genauer in \hyperref[cha:keyexchange]{Kapitel 8} behandelt.}

Eine weitere Praxisanwendung von asymmetrischen Verfahren findet sich z.B in der Erzeugung von Signaturen zum Authentifizieren von Nachrichten.\footnote{Die asymmetrische Authentifikation von Nachrichten wird in \hyperref[cha:asymmauth]{Kapitel 7} vorgestellt.}