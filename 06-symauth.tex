\chapter{Symmetrische Authentifikation von Nachrichten}
\label{cha6}

Bisher haben wir uns nur mit der Frage beschäftigt, wie ein Kommunikationsteilnehmer Bob eine Nachricht an Alice für einen unbefugten Lauscher unverständlich
machen, also verschlüsseln kann. Wir haben uns noch nicht der Frage nach der Authentifikation einer Nachricht gewidmet. Der Angreifer könnte mit dem
entsprechenden Zugriff auf den Übertragungskanal sogar eine verschlüsselte Kommunikation beeinflussen, deren Inhalt er nicht versteht. Er kann Nachrichten
abfangen, verändern und wieder auf den Weg bringen, ohne dass Alice oder Bob etwas von dem Zwischenstopp der Nachricht bemerken.

Falls ein Angreifer trotz der Verschlüsselungsmaßnahmen außerdem in der Lage ist, die Kommunikation zu verstehen, könnte er sogar \textit{gezielt} den Inhalt
von Nachrichten verändern. Gezieltes, aber auch ungezieltes Verändern von Nachrichten bezeichnen wir als \emph{aktive Angriffe}.
Es kann jedoch auch ohne Angreifer geschehen, dass der Kommunikationskanal gestört und Bobs Nachricht durch technische Einwirkungen abgewandelt wird.

Im besten Fall erhält Alice dann eine unbrauchbare Nachricht und kann bei Bob eine Wiederholung anfordern. Im schlechtesten Fall ist die Veränderung zufällig
(oder vom Angreifer gewollt) sinnvoll und beeinflusst damit das weitere Vorgehen der beiden Kommunikationsteilnehmer.

\section{Ziel}
Angesichts dessen, dass wir uns unseren Kommunikationskanal nicht immer aussuchen können, hätten wir gern einen Mechanismus, der uns ermöglicht, eine erhaltene
Nachricht auf Fehler und Veränderungen zu überprüfen (Integrität) und den Absender zu bestimmen (Authentizität). Dafür erstellt Bob für seine Nachricht $M$
zusätzlich eine "`Unterschrift"' $\sigma$ und überträgt diese gemeinsam mit $M$. Alice erhält das Paar $(M,\sigma)$ und überprüft, ob die Unterschrift auf die
erhaltene Nachricht passt.

Um ein funktionierendes und gegen einen Angreifer möglichst sicheres Unterschriftensystem zu erhalten, müssen einige Anforderungen erfüllt sein:
\begin{itemize}
  \item Bob muss $\sigma$ aus der bzw. für die Nachricht $M$ berechnen können
  \item Alice muss $\sigma$ zusammen mit $M$ verifizieren können
  \item ein Angreifer soll kein gültiges $\sigma$ für ein selbst gewähltes $M$ berechnen können
\end{itemize} 

\section{MACs}
\label{ch:symauth:macs}
Die Idee von \textit{Message Authentication Codes} (MACs) basiert auf der Grundannahme, dass Alice und Bob bereits ein gemeinsames Geheimnis $K$ besitzen. Es
ist also ein symmetrisches Verfahren. Signatur und Verifikation der Unterschrift funktionieren dann wie folgt:
\begin{itemize}
  \item \textbf{Signieren:} $\sigma \leftarrow \sig(K,M)$
  \item \textbf{Verifizieren:} $\ver(K,M,\sigma) \in \{0,1\}$
\end{itemize}
Es muss immer gelten: $\ver(K,M, \sig(K,M)) = 1$. Eine mit $\sig$ erstellte Signatur muss also unter demselben Schlüssel $K$ von $\ver$ als gültig ausgezeichnet
werden.

\section{Der EUF-CMA-Sicherheitsbegriff}
\label{ch:symauth:sicherheit}
Damit ein MAC uns nicht nur vor Übertragungsfehlern sondern auch vor einem Angreifer schützt, verlangen wir, dass kein Angreifer selbstständig ein
Nachrichten-Signatur-Paar finden kann, das gültig ist.

Er bekommt dafür ein Signaturorakel mit vor ihm verborgenem Schlüssel $K$, mit dem er Nachrichten seiner Wahl signieren kann. Er
gewinnt, wenn er die Signatur einer Nachricht $M$ korrekt vorhersagen kann, ohne $M$ vorher an das Orakel gegeben zu haben. Etwas strukturierter sieht der
Angriff für einen PPT-Angreifer $\mathcal{A}$ so aus:
\begin{enumerate}
  \item $\mathcal{A}$ erhält Zugriff auf ein Signaturorakel $\sig(K,\cdot)$
  \item $\mathcal{A}$ wählt ein Nachrichten-Signatur-Paar $(M^*,\sigma^*)$
  \item $\mathcal{A}$ gewinnt, wenn das Orakel zu $M^*$ $\sigma^*$ ausgibt, wenn also $\ver(K,M^*,\sigma^*) = 1$, und das Signaturorakel vorher $M^*$ noch
  nicht von $\mathcal{A}$ erhalten hat
\end{enumerate}
Dieses Schema bildet passive Angriffe ab, bei denen der Angreifer keinen Zugriff auf die $\ver$-Funktion hat, sondern "`blind"' signiert. Wenn es für ein $M$
nur ein einziges, gültiges $\sigma$ gibt, ist dieser Angriff dagegen äquivalent zu einer Attacke mit Zugriff auf die $\ver$-Routine. 

Einen Signaturalgorithmus, der sicherstellt, dass $\mathcal{A}$ dieses Spiel nur vernachlässigbar oft gewinnt, nennen wir \textit{EUF-CMA-sicher} (EUF:
Existential Unforgeability, CMA: Chosen Message Attack).

\section{Konstruktionen}

\subsection{Hash-then-Sign Paradigma}
Eine Signatur, die so lang ist, wie das ursprüngliche Dokument, erfüllt zwar ihren Zweck, was Datenintegrität und Authentizität angeht, ist jedoch
unpraktikabel. Wir suchen also einen Mechanismus, der uns bei variabler Nachrichtenlänge eine kurze Signatur einer Nachricht $M$ ermöglicht. Dafür bieten sich
sofort Hashfunktionen an.

Die Idee des Hash-then-Sign Paradigmas ist also, nicht die vollständige Nachricht $H~\in~\{0,1\}^*$ zu signieren, sondern den aus dieser Nachricht berechneten
Hashwert $H(M) \in \{0,1\}^k$. Die Sicherheit des MACs ist dabei sowohl von der verwendeten Hashfunktion als auch vom Signaturalgorithmus
abhängig.~\\

\begin{theorem}
Sei $(\sig, \ver)$ EUF-CMA-sicher und $H$ eine kollisionsresistente Hashfunktion. Dann ist der durch 
\begin{align*}
\sig(K,M) &= \sig(K,H(M))\\
\ver(K,M,\sigma) &= \ver(K,H(M),\sigma) 
\end{align*}
erklärte MAC EUF-CMA-sicher.~\\
\end{theorem}

\begin{beweis}[Entwurf]
\label{ch:symauth:eufcma-beweis}
Der EUF-CMA-Angreifer $\mathcal{A}$ muss entweder eine Kollision in der Hashfunktion $H$ finden und kann damit auf dieselbe Signatur von zwei Nachrichten
schließen oder er muss einen einzelnen Hashwert $H(M)$ finden, für den er eine Signatur errechnen kann.
\end{beweis}

\subsection{Pseudorandomisierte Funktionen}
Wenn man sich die Berechnung eines MACs als eine einfache Funktion im mathematischen Sinne vorstellt und damit die Errechnung eines "`frischen"' MACs zum
Finden eines unbekannten Funktionswertes wird, erkennt man schnell, dass Regelmäßigkeit in einer solchen Funktion zu Sicherheitslücken führt. Zielführender ist
es, die Funktionswerte möglichst zufällig auf ihre Urbilder zu verteilen.
\vspace{10pt}

\begin{definition}[Pseudorandomisierte Funktion (PRF)]
Sei\\ 
$\text{\textit{PRF}}\colon\{0,1\}^k \times \{0, 1\}^k \rightarrow \{0,1\}^k$
eine über $k \in \mathbbm{N}$ parametrisierte Funktion. $PRF$ heißt Pseudorandom Function (PRF), falls für jeden PPT-Algorithmus $\mathcal{A}$ die Funktion
\begin{equation*}
\text{Adv}_{\text{\textit{PRF}},\mathcal{A}}^{prf}(k) := \Pr \lbrack \mathcal{A}^{\text{\textit{PRF}}(K,\cdot)}(1^k) = 1 \rbrack - \Pr \lbrack \mathcal{A}^{R(\cdot)}(1^k) = 1 \rbrack
\end{equation*}
vernachlässigbar ist, wobei $R: \{0,1\}^k \rightarrow \{0,1\}^k$ eine echt zufällige Funktion ist.~\\
\end{definition}

Ein Kandidat für eine solche PRF ist eine Hash-Konstruktion: $PRF(K,X) = H(K,X)$. Allerdings lässt sich eine solche Konstruktion manchmal, wie bereits in
Abschnitt \ref{ch:hash:merkledamgard} bei Merkle-Damgård ausgenutzt, nach der Berechnung von $H(K,X)$ auch ohne Zugriff auf das Geheimnis $K$ noch auf
$H(K,X,X')$ erweitern. Das führt dazu, dass die PRF-Eigenschaft für Eingaben unterschiedlicher Länge nicht mehr hält. Abbildung \ref{fig:symauth:prf} verdeutlicht das.

\begin{figure}[h]
\begin{center}
\unitlength=1mm
\linethickness{0.4pt}
\hspace{-3 cm}
\begin{picture}(70,40)

\put(10,27){\vector(1,0){15}}
\put(17,28){\makebox(0,0)[cb]{IV}}

\put(30,37){\vector(0,-1){7.5}}
\put(33,31){\makebox(0,0)[cb]{$K$}}

\put(25,24.5){\framebox(10,5){$F$}}

\put(35,27){\vector(1,0){12}}
\put(52,37){\vector(0,-1){7.5}}
\put(55,31){\makebox(0,0)[cb]{$X$}}

\put(47,24.5){\framebox(10,5){$F$}}

\put(57,27){\vector(1,0){8}}
\put(75,25){\makebox(0,0)[cb]{$H(K,X)$}}


\put(0,10){\vector(1,0){15}}
\put(7,11){\makebox(0,0)[cb]{IV}}

\put(20,20){\vector(0,-1){7.5}}
\put(23,15){\makebox(0,0)[cb]{$K$}}

\put(15,7.5){\framebox(10,5){$F$}}

\put(25,10){\vector(1,0){12}}
\put(42,20){\vector(0,-1){7.5}}
\put(45,15){\makebox(0,0)[cb]{$X$}}

\put(37,7.5){\framebox(10,5){$F$}}

\put(47,10){\vector(1,0){12}}
\put(64,20){\vector(0,-1){7.5}}
\put(67,15){\makebox(0,0)[cb]{$X'$}}

\put(59,7.5){\framebox(10,5){$F$}}

\put(69,10){\vector(1,0){8}}
\put(89,8){\makebox(0,0)[cb]{$H(K,X,X')$}}

\put(53,4){\makebox(0,0)[cb]{\Large{$\Uparrow$}}}
\put(53,-2){\makebox(0,0)[cb]{$H(K,X)$ bekannt}}

\end{picture}
\end{center}
\caption{\emph{Merkle-Damgård}-Konstruktion $H_{MD}$. Es ist möglich, an einen bereits bekannten Hashwert $H(K,X)$ einen Wert $X'$ anzuhängen und trotzdem
einen korrekten Hashwert zu erzeugen.}
\label{fig:symauth:prf}
\end{figure}

\begin{theorem}
Sei $\text{PRF}\colon\{0,1\}^k \times \{0,1\}^k \rightarrow \{0,1\}^k$ eine PRF und $H \colon \{0,1\}^* \rightarrow \{0,1\}^k$ eine kollisionsresistente
Hashfunktion.
Dann ist der durch $\sig(K,M) = \text{\textit{PRF}}(K,H(M))$ gegebene MAC EUF-CMA-sicher.
\end{theorem}
\vspace{10pt}

\begin{beweis}[Entwurf]
Sei $\mathcal{A}$ ein erfolgreicher EUF-CMA-Angreifer auf ein durch $\sig(K,M) = \text{\textit{PRF}}(K,H(M))$ gegebenen MAC. Dann können wir annehmen,
dass $\mathcal{A}$ eine Fälschung $(M^*,\sigma^*)$ mit einer noch nicht signierten Nachricht $M^*$ berechnen kann. Wir können also $\mathcal{A}$ als
\textit{PRF}-Unterscheider auffassen, der mit nicht-vernachlässigbarer Wahrscheinlichkeit $\text{\textit{PRF}}(K,H(M^*))$ vorhersagt. Eine Vorhersage ist jedoch
nur dann möglich, wenn \textit{PRF} keinen Zufall ausgibt. Da \textit{PRF} jedoch nach Definition nur mit vernachlässigbarer Wahrscheinlichkeit von echtem
Zufall unterscheidbar ist, kann es einen solchen \textit{PRF}-Unterscheider nicht geben.
\end{beweis}

\subsection{HMAC}
%% Alte Version von HMAC %%
%Die am häufigsten verwendete MAC-Konstruktion ist der HMAC. Das Signaturverfahren einer Nachricht $M$ mit einer parametrisierten Hashfunktion $H$ und einem
%Schlüssel $K$ funktioniert dabei folgendermaßen:
%\begin{equation*}
%\sig(K,M) = H(K \oplus \textit{opad}, H(K \oplus \textit{ipad}, M))
%\end{equation*}
%Dabei sind \emph{opad}, \emph{ipad} $\in \{0,1\}^k$ feste Konstanten, die nacheinander in die Berechnung des MACs einfließen. Durch den geschachtelten Aufbau
%der Konstruktion werden Angriffe wie der in Abbildung \ref{fig:symauth:prf} verhindert. Außerdem führt eine Kollision in der Hashfunktion $H$ nicht zum
%Bruch des Signaturalgorithmus.

%% Neuer Entwurf zu HMAC %%
% Ein Komma finde ich als Zeichen der Konkatenation unintuitiv und unsinnig
% parametrisierte Hashfunktion?
% Welche Länge haben die Konstanten jetzt wirklich? Eigentlich macht nur die Blocklänge m und nicht die Länge der Ausgabe k Sinn (Siehe Merkle-Dåmgard Konstruktion).

Im Abschnitt zu Pseudorandomisierten Funktionen haben wir gesehen, dass Signaturverfahren, die \emph{Merkle-Dåmgard} nutzen,
dem EUF-CMA-Sicherheitsbegriff im Allgemeinen nicht genügen. Ein Angreifer, dem $\sigma = H(K,M)$ bekannt ist, erhält durch Anfügen eines Blockes $X$ problemlos den korrekten Hashwert $H(K,M,X)$ und somit die Signatur der Nachricht $M,X$.
Dennoch ist es möglich, EUF-CMA-sichere MACs zu konstruieren, die mittels einer \emph{Merkle-Dåmgard}-Konstruktion signieren.
Eines der verbreitetsten Verfahren ist der \textit{Keyed-Hash Message Authentication Code}, der HMAC. Das Signieren einer Nachricht
funktioniert dabei wie folgt:
\begin{equation*}
\sig(K,M) = H(K \oplus \textit{opad}, H(K \oplus \textit{ipad}, M))
\end{equation*}
Dabei sind \textit{opad}, das \textit{outer padding} und \textit{ipad}, das \textit{inner padding} zwei Konstanten der Blocklänge $m$ der Hashfunktion, die bei jedem Signaturvorgang gleich bleiben. Üblich\footnote{Sowohl im RFC 2104, sowie in einer Veröffentlichung des NIST und in diverser Fachliteratur werden diese Werte (als Standard) vorgeschlagen. Siehe: ~\\~\\ \url{http://tools.ietf.org/html/rfc2104} \\ \url{http://csrc.nist.gov/publications/fips/fips198-1/FIPS-198-1_final.pdf}} ist es, $\textit{opad} = \{0x5C\}^{m/8}$ und $\textit{ipad} = \{0x36\}^{m/8}$ zu wählen. Das Verifizieren funktioniert analog zu der in \ref{ch:symauth:macs} gegebenen Definition.

Immun gegen den in Abbildung \ref{fig:symauth:prf} vorgestellten Angriff ist HMAC aufgrund seiner verschachtelten Struktur. Die Nachricht
$M$, die es zu Signieren gilt, wird in jeweils zwei Hashvorgängen verarbeitet. Für eine Nachricht $M,X$ ist 
\begin{equation*}
H(K \oplus \textit{opad}, H(K \oplus \textit{ipad}, M), X)
\end{equation*} 
aber offensichtlich keine gültige Signatur. Der Angreifer müsste einen Nachrichtenblock $X$ bereits im inneren Hashvorgang unterbringen. Da er dafür allerdings $H$ invertieren, oder das Geheimnis $K$ kennen müsste, schlägt der Angriff fehl. 

Selbstverständlich muss, obwohl HMAC das eben angesprochene Problem löst, keine \textit{Merkle-Dåmgard}-Funktion verwendet werden. Für jede pseudorandomisierte Hashfunktion genügt HMAC dem EUF-CMA-Sicherheitsbegriff.