\chapter{Symmetrische Verschlüsselung}
\label{symenc}

Ein symmetrisches Verschlüsselungsverfahren sichert eine Kommunikation zwischen (typischerweise zwei) Parteien durch einen geheimen Schlüssel, den alle Parteien kennen. Der Schlüssel dient sowohl der Chiffrierung als auch der Dechiffrierung. Er wird keiner bestimmten Partei, sondern einer bestimmten Kommunikationsverbindung zugeordnet. Alle klassischen Verschlüsselungsverfahren sind symmetrisch.\\ \ \\
Um eine sichere Kommunikation zu beginnen, müssen sich beide Parteien zuvor auf einen gemeinsamen Schlüssel einigen. Diesen Vorgang nennen wir \emph{Schlüsselaustausch}. Bei \emph{offenen} digitalen Systemen wie dem Internet können wir nicht davon ausgehen, dass die Kommunikationspartner schon vorher in Kontakt standen: Prinzipiell kann jeder an einem offenen System teilnehmen und hat Zugriff auf die im System angebotenen Dienste. Daher muss der
Schlüsselaustausch innerhalb des Systems selbst erfolgen. Schlüsselaustauschverfahren betrachten wir erst  in Kapitel~\ref{ch:keyexchange}. Der Einfachheit halber gehen wir zunächst davon aus, dass beide Kommunikationspartner bereits über einen gemeinsamen geheimen Schlüssel verfügen.\\ \ \\
Eine Verschlüsselungsfunktion erwartet in der Regel eine Eingabe fester Länge. Daher wird ein Klartext beliebiger Länge vor der Verarbeitung in eine Folge von Blöcken oder Zeichen fester Länge aufgeteilt, die dann einzeln chiffriert werden. Wird für jeden Block die Verschlüsselungsoperation mit dem selben Schlüssel verwendet, so spricht man von \emph{Blockchiffren}. Diese werden in Kapitel~\ref{block} ausführlich behandelt. Als \emph{sequentielle Chiffren} oder \emph{Stromchiffren} bezeichnet man Verschlüsselungsverfahren, bei denen die Klartextzeichen nacheinander mit einem in jedem Schritt variierendem Element eines Schlüsselstrom kombiniert werden.

\section{Stromchiffren}

Wir können einen Klartext $\plaint$ als eine endliche Folge $\plaint = (\plaint_i) = (\plaint_1,\plaint_2,\ldots,\plaint_n) $ von Zeichen $\plaint_i$ aus einem Klartextalphabet auffassen. Eine Stromchiffre verschlüsselt einen Klartext, indem sie jedes Klartextzeichen $\plaint_i$ durch ein Chiffratzeichen $\ciphert_i$ aus einem Chiffratalphabet in geeigneten Weise ersetzt. Dabei wird dasselbe Klartextzeichen an verschiedenen Positionen nicht notwendigerweise durch das gleiche Chiffratzeichen codiert: Im Allgemeinen folgt für $i \ne j$ aus $\plaint_i = \plaint_j$ nicht $\ciphert_i = \ciphert_j$. Eine derartige Zeichenersetzung nennt man auch \emph{polyalphabetische Substitution}.\\ \ \\
An dieser Stelle sei erwähnt, dass eine Stromchiffre nicht auf dem ursprünglichen Alphabet des Klartextes arbeiten muss. Sie verwendet jedoch elementare Einheiten "`kleiner"' Länge, aus denen der Klartext durch Konkatenation aufgebaut werden kann. Solche Einheiten nennen wir im folgenden Zeichen.\\ \ \\
In den meisten Fällen wird die Verschlüsselungsfunktion $\enc$  mit einer einfachen Funktion realisiert, die unabhängig vom Schlüssel ist. Bei binären
Klartextströmen findet häufig die XOR"=Funktion Anwendung, der Schlüsselstrom wird also bitweise modulo 2 zum entsprechenden Teil des Klartextes hinzuaddiert ($\ciphert_i = \plaint_i\oplus\key'_i$). Der Schlüsselstrom $\key' = (\key'_i) = (\key'_1,\key'_2,\ldots,\key'_n)$ wird dabei durch einen Generator $G$ aus den Schlüssel $\key$ erzeugt. Das Verfahren ist in Abbildung~\ref{fig:streamcipher} dargestellt.

\begin{figure}[h]
\begin{center}
\unitlength=1mm
\linethickness{0.4pt}
\begin{picture}(70,50)
\put(5,5){\makebox(0,0)[cc]{$\plaint_i$}}
\put(10,5){\vector(1,0){20}}
\put(31,2){\framebox(8,6)[cc]{$\enc$}}
\put(40,5){\vector(1,0){20}}
\put(65,5){\makebox(0,0)[cc]{$\ciphert_i$}}
\put(35,20){\vector(0,-1){10}}
\put(35,13){\makebox(6,6)[cc]{$\key'_i$}}
\put(32,22){\framebox(6,6)[cc]{$G$}}
\put(35,40){\vector(0,-1){10}}
\put(35,45){\makebox(0,0)[cc]{$\key$}}
\end{picture}
\end{center}
\caption{Prinzip einer Stromchiffre. Der Klartextstrom $(\plaint_1,\plaint_2,\ldots,\plaint_n)$ wird zeichenweise mit einem, aus dem Schlüssel $\key$ mit
Generator $G$ erzeugten, Schlüsselstrom $(\key'_1, \key'_2, \ldots, \key'_n)$ durch $\enc$ verschlüsselt.}
\label{fig:streamcipher}
\end{figure}

Das klassische Beispiel einer Stromchiffre ist die in Abschnitt~\ref{Vigenere} vorgestellte \emph{Vigen\`ere"=Chiffre}. Im Gegensatz zur Vigen\`ere"=Chiffre
bietet eine Stromchiffre, die auf einer wirklich zufälligen Schlüsselfolge basiert, perfekte Geheimhaltung der verschlüsselten Nachricht. Dieses Verfahren heißt \emph{One-Time-Pad} und wird im Abschnitt~\ref{OTP} vorgestellt.

\subsection{Caesar-Chiffre}
Eine der ersten schriftlich belegten Chiffren ist die Caesar-Chiffre. Der Name stammt vom römischen Feldherrn \emph{Julius Caesar}, der nach Aufzeichnungen des
römischen Schriftstellers \emph{Sueton} seine militärische Korrespondenz verschlüsselte indem er jeden Buchstaben des Klartextalphabets um 3 nach rechts
verschob.\\
\begin{beispiel} Caesar-Chiffre:
\vspace{0.2cm}

\setlength{\tabcolsep}{2pt}
\begin{tabular}{l*{26}{c}}
Klartextalphabet: &A&B&C&D&E&F&G&H&I&J&K&L&M&N&O&P&Q&R&S&T&U&V&W&X&Y&Z\\
Geheimtextalphabet: &D&E&F&G&H&I&J&K&L&M&N&O&P&Q&R&S&T&U&V&W&X&Y&Z&A&B&C\\
\end{tabular}
\end{beispiel} \ \\
Aus dem Klartext "`CHIFFRE"' wird das Chiffrat "`FKLIIUH"'. Zur Entschlüsselung werden die Buchstaben im Geheimtextalphabet entsprechend um 3 nach links
verschoben. Das Problem bei dieser Art von Verschlüsselung ist unmittelbar ersichtlich, die Methode verändert sich nicht. Daher kann jeder der einmal erkannt
hat, wie Caesar seine Nachrichten verschlüsselte, diese ohne Probleme entschlüsseln. Es gibt keinen Schlüssel und die Sicherheit des Verfahrens hängt allein von
der Geheimhaltung der Chiffre ab.\\ \ \\
Manchmal wird auch die allgemeine \emph{Verschiebe-Chiffre} als Caesar-Chiffrierung bezeichnet. Bei dieser Chiffre gibt es einen Schlüssel, der die Anzahl der
Stellen angibt um die verschoben wird. Für die Verschlüsselung gilt dann: $\enc_\key(\plaint_i) = (\plaint_i + \key)\ mod\ 26$ und für die Entschlüsselung:
$\dec_\key(\ciphert_i) = (\ciphert_i - \key)\ mod\ 26$. Da es allerdings nur 26 mögliche Schlüssel gibt, ist es selbst ohne Computerunterstützung möglichen
jeden Schlüssel auszuprobieren. Ein solcher Angriff wird als \emph{exhaustive search} oder \emph{Brute-Force-Angriff} bezeichnet.

\begin{figure}[h]
\begin{center}
\unitlength=1mm
\linethickness{0.4pt}
\begin{picture}(100,30)
\put(5,5){\makebox(0,0)[cc]{$\plaint_i$}}
\put(10,5){\vector(1,0){20}}
\put(32,2){\framebox(35,6)[cc]{$(\plaint_i + \key) \mod 26$}}
\put(70,5){\vector(1,0){20}}
\put(95,5){\makebox(0,0)[cc]{$\ciphert_i$}}
\put(50,20){\vector(0,-1){10}}
\put(47,22){\makebox(6,6)[cc]{$\key$}}
\end{picture}
\end{center}
\caption{Prinzip einer Verschiebechiffre. Der Klartextstrom $(\plaint_1,\plaint_2,\ldots,\plaint_n)$ wird zeichenweise um den Schlüssel $\key \in \{ 1,\ldots,26\}$ verschoben.}
\label{fig:caesarcipher}
\end{figure}
Dies führt zu dem wichtigen Prinzip, dass jedes sichere Verschlüsselungsverfahren einen Schlüsselraum besitzen muss, der nicht durch exhaustive search angreifbar ist. Im heutigen
Zeitalter, in dem für einen Brute-Force-Angriff mehrere tausende Computer benutzt werden können, muss die Anzahl der möglichen Schlüssel sehr groß sein
(mindestens $2^{90}$ \cite{report:minkeylength}). Es ist jedoch wichtig zu verstehen, dass das obige Prinzip lediglich eine notwendige und keine hinreichende
Bedingung für ein sicheres Verschlüsselungsverfahren darstellt.\\ \ \\
Interessanterweise ist eine Variante der Caesar-Verschlüsselung heute weit verbreitet. Sie wird \emph{ROT-13} genannt und führt eine Verschiebung
um 13 anstatt um 3 Stellen durch. Es ist bekannt das diese Art von Verschlüsselung keine kryptographische Sicherheit bietet. \emph{ROT-13} wird lediglich dazu
verwendet, um Spoiler oder Pointen bis zu einer bewussten Entschlüsselung zu verschleiern. Der Vorteil von \emph{ROT-13} besteht darin, dass Ver- und
Entschlüsselung exakt die selbe Funktion verwendet, was für eine einfach Implementierung sorgt.

\subsection{Vigen\`ere-Chiffre}
\label{Vigenere}

Eine Weiterentwicklung der Caesar-Chiffre, die mehr Sicherheit bietet, ist die sogenannte \emph{Vigen\`ere-Chiffre}, benannt nach einem Franzosen des
sechzehnten Jahrhunderts, Blaise de Vigen\`ere. Der Unterschied zur Caesar-Chiffre besteht darin, dass nicht ein konstanter Schlüssel (und damit ein
einzelnes Alphabet) zur Chiffrierung jedes einzelnen Zeichens verwendet wird, sondern eine (möglichst lange) Folge von Schlüsseln. Der Zeichenvorrat ist das
lateinische Alphabet mit seinen 26 Buchstaben. Die Verknüpfung der Schlüsselfolge mit der Klartextfolge geschieht durch die zeichenweise Addition modulo~26.
Als Schlüsselwort dient eine periodisch wiederkehrende Zeichenfolge. Das Wiederholen einer im Verhältnis zum Klartext kurzen Schlüsselfolge
ermöglicht allerdings erst die Kryptoanalyse des Vigen\`ere"=Systems.
\clearpage

\begin{figure}[h]
\begin{center}
\unitlength=1mm
\linethickness{0.4pt}
\begin{picture}(100,50)
\put(5,5){\makebox(0,0)[cc]{$\plaint_i$}}
\put(10,5){\vector(1,0){20}}
\put(32,2){\framebox(35,6)[cc]{$(\plaint_i + \key'_i) \mod{} 26$}}
\put(70,5){\vector(1,0){20}}
\put(95,5){\makebox(0,0)[cc]{$\ciphert_i$}}
\put(50,20){\vector(0,-1){10}}
\put(50,13){\makebox(6,6)[cc]{$\key'_i$}}
\put(30,22){\framebox(40,6)[cc]{$\key'_i = \key_{((i - 1) \mod{} l) + 1}$}}
\put(50,40){\vector(0,-1){10}}
\put(50,45){\makebox(0,0)[cc]{$\key$}}
\end{picture}
\end{center}
\caption{Prinzip einer Vigen\`ere"=Chiffre. Der Schlüsselstrom $\key_i'$ entsteht durch Wiederholung des Schlüssels $\key$: $\key'_i =
\key_{((i - 1) \;  \mod{}  \; l) + 1}$ ($l$ ist die Länge des Schlüssels $\key$). Der Klartextstrom $(\plaint_1,\plaint_2,\ldots,\plaint_n)$ wird zeichenweise entsprechend der Verschiebe-Chiffre mit dem entsprechenden Buchstaben aus dem Schlüsselstrom verschlüsselt.}
\label{fig:vigerecipher}
\end{figure}

\noindent Ist $\underline k = (k_1,\ldots,k_{l})$ eine Schlüsselfolge der Länge $l$, so ist die Chiffrierabbildung der Vigen\`ere-Chiffre gegeben durch:

 \begin{equation}
   E_{\underline k} \colon m_1\ldots m_{r} \mapsto t_{k_1}(m_1)\ldots t_{k_{l}}(m_{l})t_{k_1}(m_{l+1})\ldots t_{k_{r\bmod{l}}}(m_{r})
 \end{equation}

 \begin{equation}
   \begin{array}{rp{14cm}}
     {\rm mit\colon\ }	& $m_1\ldots m_{r}$ Klartextfolge der Länge $r$ und\\
			& $t_{k_j}(m_i) := m_i + k_j \bmod{n}$, wobei die Indices von $k$ als die\\
			& Repräsentanten $1,\dots,l$ der Restklassen modulo $l$ zu verstehen sind.
   \end{array}
 \end{equation}
\vspace{0.2cm}

\begin{beispiel} Vigen\`ere-Chiffre:

Schlüssel: SICHER
\vspace{0.2cm}

\setlength{\tabcolsep}{2pt}
\begin{tabular}{l*{26}{c}}
Klartext:
&A&B&C&D&E&F&G&H&I&J&K&L&M&N&O&P&Q&R&S&T&U&V&W&X&Y&Z\\
Schlüsselfolge:
&S&I&C&H&E&R&S&I&C&H&E&R&S&I&C&H&E&R&S&I&C&H&E&R&S&I\\
Geheimtext:
&T&K&F&L&J&X&Z&Q&L&R&P&D&F&W&R&X&V&J&L&C&X&D&B&P&R&I\\
\end{tabular}
\end{beispiel}
\vspace{0.2cm}

Der Weg über die Analyse der Häufigkeitsverteilung der Zeichen im Chiffretext (Aufstellen der Histogramme) führt hier nicht zum Ziel, da die Histogramme für
lange Schlüssel verflachen, d.h. sich einander angleichen. Daher ist eine Vigen\`ere"=Chiffre wesentlich sicherer als eine einfache Substitution von
Buchstaben; sie wurde sogar bis Mitte des vorletzten Jahrhunderts für unbrechbar gehalten und als \emph{Le Chiffre ind\'{e}chiffrable} bezeichnet.
%TODO: {Citation needed}

Allerdings ist das Brechen der Vigen\`ere-Chiffre relativ einfach, sobald man die Länge $l$ des Schlüssels kennt. Diese kann man z.B. durch die
Kasiski-Friedmann-Methode ermitteln. %TODO:{Citation needed}
Nun kann das Chiffrat in $l$ unterschiedliche Teile aufgespalten werden ($t_{k_j}(m_j),$ $t_{k_j}(m_{l+j}),$ $t_{k_j}(m_{2 \cdot
l+j}),\ldots$ entspricht einem Teil), wobei die Verschlüsselung der einzelnen Teile einer Verschiebe-Chiffre entspricht (jeder enthaltene Chiffretextbuchstabe
wurde mit dem selben $k_j$ verschlüsselt), die leicht mit Hilfe von Histogrammen gebrochen werden kann. Die Länge der Schlüsselfoge kann auf folgende Weise
bestimmt werden. Man betrachte für $\tau = 1,2,\ldots$ die entsprechenden Geheimtextbuchstaben $t_{k_j}(m_j),t_{k_j}(m_{\tau+j}),t_{k_j}(m_{2 \cdot
\tau+j}),\ldots$ und die folgende Gleichung:

\begin{equation}
 S_{\tau}=	\sum_{i=0}^{25} q^2_i
\end{equation}
Wobei $q_i$ die Anzahl der Vorkommen des i-ten Buchstaben des Alphabets in der Sequenz geteilt durch die Summe aller Buchstaben der Sequenz ist. Sollte für die
Schlüsselänge $l$, $l = \tau$ gelten, so wäre zu erwarten, dass $S_{\tau}$ ungefähr den gleichen Wert hat wie unter den Wahrscheinlichkeiten eines
natürlichsprachlichen Textes (da eine Verschiebe-Chiffre die Häufigkeitsverteilung nicht verschleiert), was ungefähr $0.075$ entspricht. Für $l \neq \tau$ ist
dagegen zu erwarten, dass alle Buchstaben mit ungefähr gleicher Wahrscheinlichkeit in der Folge $t_{k_j}(m_j),$ $t_{k_j}(m_{\tau+j}),$ $t_{k_j}(m_{2 \cdot
\tau+j}),\ldots$ auftreten. Das bedeutet für alle $i$ gilt: $q_i \approx 1/26$.

\begin{equation}
 S_{\tau} \approx \sum_{i=0}^{25} (1/26)^2 \approx 0.038
\end{equation}

Was einen ausreichenden Unterschied darstellt damit diese Methode funktioniert. Abschließend bleibt noch zu erwähnen, dass für einen solchen Angriff das
Chiffrat selbstverständlich eine gewisse Länge aufweisen muss.

\subsection{One-Time-Pad}
\label{OTP}
Das One-Time-Pad ist wie bereits gesagt eine spezielle Form einer Stromchiffre und wie im Folgenden definiert:
\begin{itemize}
 \item Der zu Verschlüsselung verwendete Schlüssel $\key$ besitzt die gleich Länge $\ensuremath{l}$ wie der  Klartext $\plaint$.
 \item Der Schlüssel wird gleichverteilt aus dem Schlüsselraum $K=\{0,1\}^{l}$ ausgewählt.   Jeder beliebige Schlüssel wird also mit einer Wahrscheinlichkeit
 von $\frac{1}{2^{l}}$ ausgewählt.
 \item Zur Verschlüsselung wird der Klartext und der Schlüssel bitweise mit XOR verknüpft:  $\forall i: \, \ciphert_i = \plaint_i\oplus\key_i$.
 \item Zur Entschlüsselung wird das Chiffrat und der Schlüssel bitweise mit XOR verknüpft:  $\forall i: \, \plaint_i = \ciphert_i\oplus\key_i$.
  \item Der Schlüssel darf weder vollständig noch teilweise wiederverwendet werden.
\end{itemize}
Ist die obige Definition erfüllt, bietet das One-Time-Pad bewiesenermaßen perfekte Geheimhaltung.

\begin{figure}[h]
\begin{center}
\unitlength=1mm
\linethickness{0.4pt}
\begin{picture}(70,30)
\put(5,5){\makebox(0,0)[cc]{$\plaint_i$}}
\put(10,5){\vector(1,0){20}}
\put(35,5){\circle{6}}
\put(32,5){\line(1,0){6}}
\put(35,2){\line(0,1){6}}
\put(40,5){\vector(1,0){20}}
\put(65,5){\makebox(0,0)[cc]{$\ciphert_i$}}
\put(35,20){\vector(0,-1){10}}
\put(32,22){\makebox(6,6)[cc]{$\key_i$}}
\end{picture}
\end{center}
\caption{Prinzip eines One-Time-Pads. Der Klartextstrom $(\plaint_1,\plaint_2,\ldots,\plaint_n)$ wird zeichenweise mit den Schlüsselstrom $(\key_1, \key_2,
\ldots, \key_n)$ binär addiert (auch XOR genannt).}
\label{fig:OTP}
\end{figure}

Allerdings hat das One-Time-Pad auch etliche Nachteile. Der Hauptnachteil besteht darin, dass der Schlüssel echt zufällig gewählt und genauso lang sein muss,
wie das Chiffrat, was das Problem der sicheren Übertragung der Nachricht in das Problem der sicheren Übertragung des Schlüssels umwandelt. Weiterhin ist es
zeitlich und/oder technisch aufwändig "`echten Zufall"' zu erzeugen, da dieser aus physikalischen Phänomenen erhalten wird. Außerdem bietet das Verfahren zwar
perfekte Geheimhaltung allerdings absolut keinen Schutz gegen eine Veränderung der Nachricht.

Die obigen Gründe machen die Verwendung des One-Time-Pad sehr aufwändig, weswegen es auch nur sehr selten eingesetzt wird. Moderne Stromschiffren funktionieren
prinzipiell wie ein One-Time-Pad, benutzen jedoch einen Pseudozufallszahlengenerator, der aus einem kurzen Seed, dem Schlüssel, eine Folge aus
Pseudozufallszahlen erzeugt. Diese Art von Chiffren wird kurz im nächsten Abschnitt vorgestellt.

\subsection{Stromchiffren mit Pseudozufallszahlen}
\label{pseudorandomstreamcipher}

Wir wissen bereits, dass die Zufallsfolge, die beim One-Time-Pad als Schlüssel dient,
 mindestens so lang sein muss wie die zu verschlüsselnde Nachricht und nur ein einziges Mal verwendet werden darf. Hieraus folgt, dass dieses Verfahren einen
 extrem hohen Aufwand für die sichere Schlüsselverteilung erfordert und aus diesem Grund für die meisten Anwendungen unpraktikabel ist. Es liegt nun nahe, die
 genannte Schwierigkeit zu umgehen, indem man nach dem Vorbild des One-Time-Pad Stromchiffren konstruiert, die statt einer wirklichen Zufallsfolge sogenannte
 Pseudozufallsfolgen verwenden. Unter einer \emph{Pseudozufallsfolge} versteht man dabei eine Folge von Zeichen, die mittels eines deterministischen Prozesses
 aus einem relativ kurzen Initialisierungswert erzeugt wird und gewisse Eigenschaften einer echt zufälligen Folge aufweist. Wie im letzten Abschnitt
 angesprochen wird dieser Initialisierungswert \emph{Seed} genannt. Wenn beide Kommunikationspartner über identische Generatoren verfügen, muss nur noch der
 Initialwert und die gewählte Parametrisierung des Generators als Schlüssel verteilt werden. Die eigentliche Schlüsselfolge kann dann an beiden Enden des
 Kanals erzeugt werden. Eine Voraussetzung der Konstruktion ist offensichtlich, dass der Pseudozufallsgenerator effizient berechenbar ist. Außerdem soll hier
 noch auf den Umstand hingewiesen werden, dass es sich bei der Schlüsselfolge nicht um den Schlüssel des Verfahrens handelt, da die Folge ein Menge von internen
 Werten des Algorithmus ist. In Abbildung~\ref{fig:pseudorandomstreamcipher} ist der prinzipielle Aufbau einer derartigen Stromchiffre gezeigt.

\begin{figure}[h]
\begin{center}
\unitlength=1mm
\linethickness{0.4pt}
\hspace{-3 cm}
\begin{picture}(100,50)

%Baseline m -> \oplus -> c -> \oplus -> m

\put(5,5){\makebox(0,0)[cc]{$\plaint_i$}}
\put(10,5){\vector(1,0){20}}
\put(35,5){\circle{6}}
\put(32,5){\line(1,0){6}}
\put(35,2){\line(0,1){6}}
\put(40,5){\vector(1,0){20}}
\put(65,5){\makebox(0,0)[cc]{$\ciphert_i$}}
\put(70,5){\vector(1,0){20}}
\put(95,5){\circle{6}}
\put(92,5){\line(1,0){6}}
\put(95,2){\line(0,1){6}}
\put(100,5){\vector(1,0){20}}
\put(125,5){\makebox(0,0)[cc]{$\plaint_i$}}

%Encryption

\put(35,20){\vector(0,-1){10}}
\put(35,13){\makebox(6,6)[cc]{$\key'_i$}}
\put(27.5,22){\framebox(10,6)[cc]{$SC$}}
\put(35,40){\vector(0,-1){10}}
\put(35,42){\makebox(0,0)[cc]{$\key_0$}}
\put(35,46){\makebox(0,0)[cc]{(\emph{Seed})}}
\put(30,20){\line(0,-1){5}}
\put(30,15){\line(-1,0){10}}
\put(20,15){\line(0,1){20}}
\put(20,35){\line(1,0){10}}
\put(30,35){\vector(0,-1){5}}
\put(25,37.5){\makebox(0,0)[cc]{$\key_{n+1}$}}

%Decryption

\put(95,20){\vector(0,-1){10}}
\put(95,13){\makebox(6,6)[cc]{$\key'_i$}}
\put(87.5,22){\framebox(10,6)[cc]{$SC$}}
\put(95,40){\vector(0,-1){10}}
\put(95,42){\makebox(0,0)[cc]{$\key_0$}}
\put(95,46){\makebox(0,0)[cc]{(\emph{Seed})}}
\put(90,20){\line(0,-1){5}}
\put(90,15){\line(-1,0){10}}
\put(80,15){\line(0,1){20}}
\put(80,35){\line(1,0){10}}
\put(90,35){\vector(0,-1){5}}
\put(85,37.5){\makebox(0,0)[cc]{$\key_{n+1}$}}
\end{picture}
\end{center}
\caption{Prinzip einer Stromchiffre mit Pseudozufall. Der Klartextstrom wird zeichenweise mit einem aus dem \emph{Seed} $\key_0$ generierten pseudozufälligen
Schlüsselstrom ver- und entschlüsselt. Wichtig ist dass sowohl bei Ver- als auch Entschlüsselung der selbe \emph{Seed} und die selbe Funktion $SC$ verwendet
wird. Nennenswert ist außerdem, dass die Funktion $SC$ nicht in jedem Iterationsschritt ein Schlüsselbit erzeugen muss, weshalb die Zählvariablen $i$ und $n$ nicht synchron sein müssen.}
\label{fig:pseudorandomstreamcipher}
\end{figure}

Eine einfache und beliebte Möglichkeit zur Implementation bieten \emph{Linear Feedback Shift Register} (\emph{LFSR}). Die Arbeitsweise ist in
Abbilding~\ref{fig:LFSR} dargestellt und funktioniert wie folgt: Der Mechanismus besitzt einen Zustand, welcher in $k$ Registern gespeichert ist. In einem
Arbeitsschritt wird die Summe der Produkte der Register und einem je Register konstantem Koeffizienten $\alpha$ gebildet. Dieses Ergebnis wird nun im
höchstwertigsten Register gespeichert. Der Inhalt dieses Registers wird dabei im nächsten Register gespeichert usw. Damit werden also alle Register \glqq
weitergeschoben\grqq, wobei das niederwertigste Register ausgegeben wird, d.h. der Inhalt des Registers wird nun als Teil des Schlüsselstroms zur
Verschlüsselung der Nachricht verwendet.

\clearpage

\begin{figure}[h]
\begin{center}
\unitlength=1mm
\linethickness{0.4pt}
\hspace{-3 cm}
\begin{picture}(100,40)
\put(0,35){\makebox(20,5)[cc]{Initialwert}}
\put(20,35){\framebox(10,5)[cc]{$K_0$}}
\put(30,35){\framebox(10,5)[cc]{$K_1$}}
\put(40,35){\framebox(10,5)[cc]{$\ldots$}}
\put(50,35){\framebox(10,5)[cc]{$K_{n-1}$}}
\put(60,35){\framebox(10,5)[cc]{$K_n$}}
\put(20,30){\makebox(10,5)[cc]{$\cdot \alpha_0$}}
\put(30,30){\makebox(10,5)[cc]{$\cdot \alpha_1$}}
\put(40,30){\makebox(10,5)[cc]{\ldots}}
\put(50,30){\makebox(10,5)[cc]{$\cdot \alpha_{n-1}$}}
\put(60,30){\makebox(10,5)[cc]{$\cdot \alpha_n$}}
\put(20,30){\vector(1,0){50}}
\put(82.5,27.5){\makebox(20,5)[cc]{$z = \sum\nolimits^n_{i=0} \alpha_i \cdot K_i$ mod 2}}
\put(0,10){\makebox(20,5)[cc]{Zustand $k_1$}}
\put(20,10){\framebox(10,5)[cc]{$K_1$}}
\put(30,10){\framebox(10,5)[cc]{$K_2$}}
\put(40,10){\framebox(10,5)[cc]{$\ldots$}}
\put(50,10){\framebox(10,5)[cc]{$K_n$}}
\put(60,10){\framebox(10,5)[cc]{$z$}}
\put(0,6.5){\vector(1,0){10}}
\put(27,4){\makebox(10,5)[cc]{Schlüsselstrom $k'_1 = K_0$}}
\end{picture}
\end{center}
\caption{Prinzip von \emph{Linear Feedback Shift Registern}. In jedem Schritt wird $z$ aus den in den Registern vorhandenen Schlüsselbits und den Konstanten $\alpha_i$ berechnet. $z$ ersetzt dann das Schlüsselbit $\key_n$, welches an Stelle von $\key_{n-1}$ rückt. $\key_{n-1}$ selbst rückt wiederum an Stelle von $\key_{n-2}$, usw. Schließlich wird $\key_0$ durch $\key_1$ ersetzt. $\key_0$ wird ausgegeben und als nächstes Bit im Schlüsselstrom verwendet.}
\label{fig:LFSR}
\end{figure}

Wenn man für die Zustände der \emph{LFSR} nun die Gestalt $(K_1, K_2, \cdots, K_n)^T$ wählt, lassen sich alle Zustände wie folgt darstellen:

\begin{equation*}
k_{i+1} = A \cdot k_i, \; \;
A:=\begin{pmatrix}
0 & 1 & 0 & \cdots & 0 \\
0 & 0 & 1 & & 0 \\
\vdots & \vdots & & \ddots & \\
0 & 0 & 0 & & 1 \\
\alpha_1 & \alpha_2 & \alpha_3 & \cdots & \alpha_n
\end{pmatrix}
\end{equation*}

Daraus ergibt sich für den Schlüsselstrom:

\begin{align*}
k'_{i+1}  & =  (1,0,\cdots,0) \cdot k_i \\ & = (1,0,\cdots,0) \cdot (A^i \cdot k_0) \\ & = ((1,0,\cdots,0) \cdot A^i) \cdot k_0
\end{align*}

Die Beziehung des \emph{LFSR} zu einem Klartext-Chiffrat-Paar lässt sich nun als Gleichungssystem darstellen:


\begin{equation*}
	\begin{array}{lcllllclllrlll}
	\ciphert_1 &=& \plaint_1 & \oplus \, ( & (1,0,\cdots,0) \cdot A^0 &)\cdot k_0 & = & \plaint_1 & \oplus &(1,0,\cdots, & 0) &&\cdot k_0\\
	\ciphert_2 &=& \plaint_2 & \oplus \, ( & (1,0,\cdots,0) \cdot A^1 &)\cdot k_0 & = & \plaint_2 & \oplus &(0,1,0,\cdots, & 0) &&\cdot k_0\\
	&\vdots &&&&&\vdots &&&&&&\\
	\ciphert_{n} &=& \plaint_{n} & \oplus \, ( & (1,0,\cdots,0) \cdot A^{n - 1} &) \cdot k_0 & = & \plaint_{n} & \oplus & (0,\cdots, &0, 1) && \cdot k_0 \\
	\ciphert_{n+1} & = & \plaint_{n+1} & \oplus \, ( & (1,0,\cdots,0) \cdot A^n & )\cdot k_0 & = & \plaint_{n+1} & \oplus & (\alpha_1,\alpha_2,\cdots, & \alpha_n) && \cdot k_0 \\
	\ciphert_{n + 2} & = & \plaint_{n + 2} & \oplus \, ( & (1,0,\cdots,0) \cdot A^{n + 2} & ) \cdot k_0 & = & \plaint_{n + 2} & \oplus & (\alpha_1,\alpha_2,\cdots, &\alpha_n) &\cdot A^1 & \cdot k_0 \\
	&\vdots &&&&&\vdots &&&&\\
	\end{array}
\end{equation*}
Besitzt der Angreifer also ein Klartext-Chiffrat-Paar mindestens der Länge $n$ (die Anzahl der Register des \emph{LFSR}s), kann er dadurch direkt $k_0$
berechnen. Entsprechend ist ein solches Schieberegister alleine eher unsicher. Hilfe bietet hier eine möglichst strukturzerstörende Verbindung mehrerer
Schieberegister. Beispielsweise könnte man zwei \emph{LFSR} verwenden, wobei das zweite \emph{LFSR} nur dann arbeitet, wenn die Ausgabe des ersten \emph{LFSR}
1 ist.

Das Thema wird in der Vorlesung \glqq symmetrische Verschlüsselungsverfahren\grqq\ tiefer behandelt.

\section{Blockchiffren}
\label{block}

%\subsection{Funktionsweise}
%Im Gegensatz zu Stromchiffren, werden bei Blockchiffren in jedem Schritt nicht ein Zeichen sondern eine feste Anzahl, ein Block, an Zeichen verschlüsselt.
%Schematisch ergibt sich nahezu das selbe Bild wie bei Stromchiffren (siehe Abbildung~\ref{fig:streamcipher}), allerdings ergeben sich gerade in der
%Implementierung große Unterschiede.
%
%Einerseits kann die tatsächliche Verschlüsselungsfunktion nun komplexer sein als ein einfaches XOR, da es bei Blöcken mehr Möglichkeiten für
%Strukturänderungen gibt. Andererseits benötigen diese Verfahren natürlich mehr Rechenleistung als die Schieberegister und XOR-Netze von Stromchiffren, wodurch
%der Datendurchsatz sinkt. 

\subsection{Verschlüsselungsverfahren}
%TODO Sicherheitsdiskussion: "Außerdem bieten komplexere Verfahren mehr Möglichkeiten für Angreifer, weshalb es auch schwieriger ist, eine beweisbare Sicherheit festzustellen. "
Im Gegensatz zu Stromchiffren, werden bei Blockchiffren eine feste Anzahl an Bits - ein Block - verschlüsselt. Schematisch ergibt sich nahezu dasselbe Bild wie bei Stromchiffren (siehe Abbildung~\ref{fig:streamcipher}), allerdings ergeben sich in der Implementierung große Unterschiede. Einerseits ist die tatsächliche Verschlüsselungsfunktion in der Praxis nun komplexer als ein einfaches XOR, da es bei Blöcken mehr Möglichkeiten zur Strukturänderung gibt. Andererseits benötigen diese Verfahren mehr Rechenleistung als die Schieberegister und XOR-Netze von Stromchiffren, wodurch der Datendurchsatz sinkt.
Formal dargestellt ist eine Blockchiffre eine Funktion
\begin{align*}
	\enc \colon \{0, 1\}^k \ctsProd \{0, 1\}^l \rArrow \{0, 1\}^l,
\end{align*}
wobei \(k\) die Schlüssellänge und \(l\) die Blocklänge ist. Blockchiffren stellen also Permutationen der Menge 
\(\{0, 1\}^l\) dar.
Bevor wir eine erste Blockchiffre anschauen, müssen wir uns überlegen, welche Eigenschaften wir von einer sicheren Blockchiffre fordern.

Das übergeordnete Design-Kriterium, welches Blockchiffren unterliegen sollen, ist die Nichtunterscheidbarkeit\footnote{Damit meinen wir, dass das Ergebnis der Blockchiffre durch keinen in Polynomialzeit laufenden Algorithmus von echtem Zufall unterschieden werden kann.} von einer echt zufälligen Funktion. Präziser gesagt darf sich die Permutation einer Blockchiffre nicht von einer echt zufälligen Permutation derselben Menge unterscheiden. Daraus folgt sofort, dass kleine Änderungen in der Eingabe zu großen Änderungen in der Ausgabe einer Blockchiffre führen müssen. Betrachten wir kurz ein Szenario, in dem eine Blockchiffre verwendet wird, die diese Eigenschaft nicht erfüllt: Ein Angreifer ist im Besitz zweier Chiffratblöcke, die sich nur in einem Bit unterscheiden. Er kann nun davon ausgehen, dass sich auch die beiden dazugehörigen Nachrichten nur geringfügig voneinander unterscheiden und kann dieses Wissen für einen Angriff auf das Chiffrat bzw. das Verschlüsselungsverfahren ausnutzen. Wie aber können wir eine Blockchiffre konstruieren, die obige Eigenschaft erfüllt?

Zum Beantworten dieser Frage fordern wir zunächst zwei Eigenschaften\footnote{Beide Eigenschaften wurden zuerst von Claude Shannon in \href{http://netlab.cs.ucla.edu/wiki/files/shannon1949.pdf}{Communication Theory of Secrecy Systems} definiert und stellen eine wichtige Grundlage der heutigen Kryptographie dar.}, die unsere Blockchiffre erfüllen soll:
Die erste Eigenschaft besagt, dass jedes Zeichen des Chiffrats von mehreren Teilen des verwendeten Schlüssels abhängig sein soll. Im Englischen wird diese als \textit{confusion} bezeichnet. Sie erschwert es einem Angreifer, Zusammenhänge zwischen einem Schlüssel und eines damit generierten Chiffrates zu erkennen.
Die zweite Eigenschaft fordert, dass das Ändern eines einzelnen Zeichens in der Nachricht bzw. dem Chiffrat zu großen Änderungen im Chiffrat bzw. der Nachricht führt. Diese Eigenschaft wird als \textit{diffusion} bezeichnet.

Eine Umsetzung dieser Eigenschaften in eine Blockchiffre führt uns zu dem Konzept der \textit{Feistel networks}.
Die Grundidee hinter so einem Netzwerk ist, dass wir unsere Blockchiffre \(\enc\) aus mehreren Rundenfunktionen \(F_1, F_2,\dots, F_n\) zusammenbauen, die nacheinander ausgeführt werden. Die einzelnen Funktionen müssen dabei nicht notwendigerweise verschieden sein, wie wir im Abschnitt zu \hyperref[sssec:des]{\textit{DES}} sehen werden. Die Funktion \(F_i\) wird in der $i$-ten Runde des Algorithmus ausgeführt und ihre Ausgabe dient als Eingabe für die Funktion \(F_{i+1}\). Die Rundenfunktionen sind dabei so konstruiert, dass sich kleine Änderungen in der Eingabe exponentiell über die Runden ausbreiten.

\tikzset{XOR/.style={fill=black!15,draw,minimum size=13pt,circle,append after command={
			[shorten >=\pgflinewidth, shorten <=\pgflinewidth,]
			(\tikzlastnode.north) edge (\tikzlastnode.south)
			(\tikzlastnode.east) edge (\tikzlastnode.west)
		}
	}
}

\begin{figure}[h]
	\centering
	\tikzstyle{encrypt}=[draw,fill=black!15,rectangle,minimum size=20pt,inner sep=0pt]
	\begin{tikzpicture}
	\begin{scope}[>=latex] %for filled arrow tips
	\newcommand{\n}{3}
	\draw [decorate,decoration={brace,amplitude=10pt},xshift=0,yshift=0pt] (6.5, {(\n -1)*2 + 2.25}) -- (6.5,-3.25) node [black,midway,xshift=28.0, yshift=0.0] {$\enc$};
	
	\node (rectM) at (2, {(\n -1)*2 + 2}) [draw,thick,minimum width=8cm, minimum height=0.5cm] {$M$};
	\node (rect1) at (0, {(\n -1)*2 + 1}) [draw,thick,minimum width=4cm, minimum height=0.5cm] {$L_0$};
	\node (rect2) at (4, {(\n -1)*2 + 1}) [draw,thick,minimum width=4cm, minimum height=0.5cm] {$R_0$};
	\draw[->,semithick] (rectM) -- (2, {(\n -1)*2 + 1.30});
	\foreach \nr in {1, ..., \n}{
		\node (x\nr)[XOR] at (0,{(\n-\nr)*2}) {};
		\node (F\nr)[encrypt] at (2.0, {(\n-\nr)*2}) {$F_\nr$};
		\draw[->,semithick] (F\nr) -- (x\nr);
	}
	\node (rectL3) at (0, -2.0) [draw,thick,minimum width=4cm, minimum height=0.5cm] {$L_3$};
	\node (rectR3) at (4, -2.0) [draw,thick,minimum width=4cm, minimum height=0.5cm] {$R_3$};
	% draw first connection lines
	\draw[->,semithick] (rect1) -- (x1);
	\draw[->, semithick] (rect2) |- (F1);
	
	% draw connection lines between F-Boxes
	\foreach \nr in {1, ..., 2}{
		\pgfmathsetmacro\cnr{int(\nr + 1)}
		\draw[->, semithick] (4, {(\n - \nr)*2}) -- (4, {(\n-\nr)*2 - 0.6}) -- (0, {(\n-\nr)*2 - 1.2}) -- (x\cnr);
		
		\draw[->, semithick] (x\nr) -- (0, {(\n-\nr)*2 - 0.6}) -- (4, {(\n-\nr)*2 - 1.2}) |- (F\cnr);
	}
	\draw[->, semithick] (x3) -- (0, {(\n-3)*2 - 0.6}) -- (4, {(\n-3)*2 - 1.2}) -- (rectR3);
	\draw[->, semithick] (4, {(\n - (3))*2}) -- (4, {(\n-(3))*2 - 0.6}) -- (0, {(\n-(3))*2 - 1.2}) -- (rectL3);
	
	\node (rectC) at (2, -3) [draw,thick,minimum width=8cm, minimum height=0.5cm] {$C$};
	\draw[->,semithick] (2, -2.25) -- (rectC);
	
	\end{scope}
	\end{tikzpicture}
	\caption{3-ründige \textit{Feistel}-Struktur}
\end{figure}

Eine Rundenfunktion \(F_i\) besteht typischerweise aus Permutationen und mehreren Funktionen, auf denen die Eingabe aufgeteilt wird. Diese Funktionen werden als \textit{S-Boxen} (substitution boxes) bezeichnet und sind der Grundbaustein der \textit{Feistel}-Struktur. Die hier betrachteten \textit{S-Boxen} realisieren eine Funktion der Form
\begin{align*}
	S \colon \{0, 1\}^m \rArrow \{0, 1\}^n
\end{align*}
mit \(m > n\). Dabei werden alle $m$-Bit langen Wörter (\(2^m\) viele) auf $n$-Bit lange Wörter (\(2^n\) viele) abgebildet und wir erkennen, dass diese \textit{S-Boxen} nicht invertierbar sind. Als Folgerung ergibt sich, dass die Rundenfunktionen nicht invertierbar sein können. Diese Eigenschaft ist signifikant für die Sicherheit von \textit{Feistel}-Netzwerken und der sie verwendenden Blockchiffren. Zusätzlich zur Nichtinvertierbarkeit haben die \textit{S-Boxen} und die Rundenfunktionen weitere folgende Eigenschaften:

\begin{enumerate}
	\item Wird in der Eingabe für eine \textit{S-Box} ein Bit verändert, so ändern sich mindestens zwei Bit in der Ausgabe.
	\item Die Ausgabe-Bits einer Rundenfunktion $F_i$ werden so permutiert, dass alle Ausgabe-Bits einer \textit{S-Box} auf unterschiedliche \textit{S-Boxen} der nächsten Runde verteilt werden.
\end{enumerate}

Beide Merkmale stellen die \textit{confusion}-Eigenschaft der \textit{Feistel}-Struktur sicher. Betrachten wir kurz folgendes vereinfachendes Beispiel:
Gegeben seien zwei Eingaben $X$ und $X'$, die sich in genau einem Bit unterscheiden. In wie vielen Bits unterscheiden sich $\enc(X)$ und $\enc(X')$? 
In der ersten Runde unterscheidet sich die Eingaben $X_1 = X$ und $X_1' = X'$ in genau einem Bit, die Ausgaben $X_2 = F_1(X_1)$ und $X_2' = F_1(X_1')$ unterscheiden sich also mindestens in 2 Bits. In der zweiten Runde unterscheiden sich die beiden Eingaben $X_2$, $X_2'$ in mindestens 2 Bit, die Ausgaben $X_3 = F_2(X_2)$ und $X_3' = F_2(X_2')$ sind in mindestens 4 Bits unterschiedlich. Führen wir das für weitere Runden fort, stellen wir fest, dass sich die Anzahl der Bits, die von der ursprünglichen 1-Bit Änderung betroffen sind, exponentiell erhöht (nach $n$ Runden mindestens $2^n$ viele Bits).
Wichtig ist, dass sich die Ausgaben $F_n(X_n)$ und $F_n(X_n')$ dabei nicht in allen Bits unterscheiden müssen. Wäre dies immer der Fall, dann könnte ein Angreifer unsere Blockchiffre von einer echten Zufallsfunktion unterscheiden\footnote{Für eine echte Zufallsfunktion wird erwartet, dass sich bei einer 1-Bit Änderung der Eingabe nur die Hälfte der Ausgabe-Bits verändert.}. Wir erhalten also lediglich eine untere Schranke für die Anzahl der durchzuführenden Runden. Bei einer Eingabelänge von $n$ Bit sind mindestens $ \lceil \log n \rceil$ viele Runden nötig, damit sich eine Änderung der Eingabe auf alle Bits der Ausgabe auswirkt. Führen wir weniger Runden aus, enthält die neue Ausgabe unveränderte Bits und die Blockchiffre kann von einer echten Zufalls-Funktion unterschieden werden.

%TODO: Schlüsselabhängigkeit einfügen?

Das besondere Merkmal einer \textit{Feistel}-Struktur ist, dass obwohl ihre Komponenten (Rundenfunktionen, \textit{S-Boxen}) nicht invertierbar sein müssen, die \textit{Feistel}-Struktur selber es aber sehr wohl ist.

\subsubsection{DES - Data Encryption Standard}
\label{sssec:des}

%Der Data Encryption Standard ist eine Produktchiffre, die im Jahre 1977 als US-Ver\-schlüs\-sel\-ungs\-stan\-dard vom National Bureau of Standards (NBS) als
%Norm eingeführt wurde.

%Dieser Standard wurde in Zusammenarbeit mit IBM entworfen und entstand aus dem Lucifer-Algorithmus.
%
%Den Kern der Norm bildet ein Algorithmus, der Data Encryption Algorithmus, zur Ver- und Entschlüsselung binär dargestellter Nachrichten. Dieser Algorithmus
%besteht aus einer Folge von Permutationen und logischen Produktbildungen zwischen dem Klartext und einem individuellen Schlüssel.
%
%Der DES hat sich trotz zahlreicher Kritiken bezüglich seiner Sicherheit bei den meisten Anwendern durchsetzen können. Dies geschah nicht zuletzt deshalb, weil
%sehr früh integrierte Schaltungen auf dem "`freien"' Markt verfügbar waren.
%
%Der DES ist eine Block-Produkt-Chiffre aus Permutationen und Substitutionen, die in einer Iterationsschleife schlüsselgesteuert 16 mal durchlaufen werden.
%
%Dazu wird der Klartext in Blöcke der Länge 64 Bit zerlegt. Ein Schlüssel besteht ebenfalls aus 64 Bit, von denen 56 den eigentlichen Schlüssel darstellen,
%während die restlichen 8 Bit die Funktion von Paritätsbits haben, die die acht 7-Bit-Zeichen eines Nachrichtenblockes auf ungerade Parität ergänzen. Somit
%stehen $2^{56}\approx 7,2\cdot 10^{16}$ unterschiedliche Schlüssel zur Verfügung.

Im Jahr 1973 gab das \textit{National Bureau of Standards} (NBS) der USA, das heutige \textit{National Institute of Standards and Technology} (NIST), eine öffentliche Anfrage nach einem Verschlüsselungsalgorithmus ab, der sensitive Regierungsinformationen sicher verschlüsseln kann. 1974 entwarf IBM einen Kandidaten, der auf dem \textit{Lucifer}-Algorithmus basiert und ein \textit{Feistel}-Netzwerk verwendet. Das NBS kontaktierte daraufhin die \textit{National Security Agency} (NSA), um die Sicherheit dieses Algorithmus zu überprüfen. Nachdem die NSA einige Änderungen an diesem durchgeführt hatte, wurde der überarbeitete Algorithmus 1977 als \textit{Data Encryption Standard} (DES) \href{http://csrc.nist.gov/publications/fips/archive/fips46-3/fips46-3.pdf}{standardisiert} und für die öffentliche Verwendung freigegeben. Der DES ist ein symmetrischer Verschlüsselungsalgorithmus, der ein wie oben beschriebenes Feistel-Netzwerk verwendet, einen 64-Bit langen Schlüssel benutzt und Daten in je 64-Bit Blöcken verschlüsselt. Die öffentliche Standardisierung des DES durch eine US-Regierungsbehörde trug maßgeblich zur schnellen weltweiten Verbreitung des Algorithmus bei. Gleichzeitig führte die Beteiligung der NSA am Entwurf des DES dazu, dass die Sicherheit des DES kontrovers diskutiert wurde. Die durchgeführten Änderungen der NSA umfassten die Verkürzung des Schlüssels von ursprünglich 128 Bit auf 56 zufällig gewählte Bit, sowie eine unkommentierte Veränderung der verwendeten S-Boxen. In Anbetracht der zentralen Bedeutung der S-Boxen für die Sicherheit eines \textit{Feistel-}Netzwerkes wurde somit befürchtet, dass die NSA eine Hintertür in den DES eingebaut hatte. Daraufhin wurden 1994 die Design-Kriterien für die verwendeten S-Boxen von IBM \href{http://simson.net/ref/1994/coppersmith94.pdf}{veröffentlicht}. Die Veröffentlichung ergab, dass die S-Boxen besonders resistent gegenüber der erst kurz vorher (1990) öffentlich-bekannt gewordenen \hyperref[sssec:diffKryptoanalyse]{\textit{differentiellen Kryptoanalyse}} sind.\footnote{Untersuchungen haben ergeben, dass eine zufällige Wahl der S-Boxen zu einer deutlich höheren Anfälligkeit gegenüber der \textit{differentiellen Kryptoanalyse} geführt hätte. Dies impliziert, dass IBM und die NSA bereits Jahre vor der Öffentlichkeit über diese Angriffsmethode Bescheid wussten.}
\newpage
\bigskip
\begin{figure}[h]
	\begin{center}
		\unitlength=1mm
		\linethickness{0.4pt}
		\begin{picture}(145,175)
		\put(132,168){\makebox(0,0)[cc]{$\key$}}
		\put(128,158){\line(2,1){8}}
		\put(138,160){\makebox(0,0){64}}
		\put(132,165){\vector(0,-1){9}}
		\put(119,148){\framebox(26,8)[cc]{}}
		\put(132,154){\makebox(0,0)[cc]{Schlüsselaus-}}
		\put(132,150){\makebox(0,0)[cc]{wahlfunktion}}
		\put(0,169){\framebox(100,4)[cc]{$\plaint_i$}}
		\put(46,164){\line(2,1){8}}
		\put(56,165){\makebox(0,0){64}}
		\put(50,169){\vector(0,-1){7}}
		\put(0,158){\framebox(100,4)[cc]{$IP$}}
		\put(50,158){\vector(0,-1){5}}
		\put(51,152){\line(1,0){24}}
		\put(25,152){\line(1,0){24}}
		\put(50,152){\circle{2}}
		\put(21,147){\line(2,1){8}}
		\put(31,148){\makebox(0,0){32}}
		\put(25,152){\vector(0,-1){7}}
		\put(71,147){\line(2,1){8}}
		\put(81,148){\makebox(0,0){32}}
		\put(75,152){\vector(0,-1){7}}
		\put(55,140){\framebox(45,5)[cc]{$R_0$}}
		\put(0,140){\framebox(45,5)[cc]{$L_0$}}
		\put(110,140){\makebox(0,0)[cc]{$\key_{1}$}}
		\put(115,136.5){\line(2,1){8}}
		\put(119,135){\makebox(0,0)[cc]{48}}
		\put(125,138.5){\line(0,1){9.5}}
		\put(50,138.5){\line(1,0){75}}
		\put(50,138.5){\vector(0,-1){1.5}}
		\put(75,140){\line(0,-1){10}}
		\put(75,135){\vector(-1,0){23}}
		\put(50,135){\makebox(0,0)[cc]{$f$}}
		\put(50,135){\circle{4}}
		\put(48,135){\vector(-1,0){21}}
		\put(25,140){\vector(0,-1){3}}
		\put(25,133){\line(0,1){4}}
		\put(23,135){\line(1,0){4}}
		\put(25,135){\circle{4}}
		\put(25,133){\line(0,-1){3}}
		\put(25,130){\line(0,-1){3.5}}
		\put(75,130){\line(0,-1){3.5}}
		\put(75,118){\line(-6,1){50}}
		\put(25,118){\line(6,1){50}}
		\put(25,118){\vector(0,-1){3}}
		\put(75,118){\vector(0,-1){3}}
		\put(55,110){\framebox(45,5)[cc]{$R_1 = L_0 \oplus f(R_0,\key_1)$}}
		\put(0,110){\framebox(45,5)[cc]{$L_1 = R_0$}}
		\put(110,110){\makebox(0,0)[cc]{$\key_2$}}
		\put(115,106.5){\line(2,1){8}}
		\put(119,105){\makebox(0,0)[cc]{48}}
		\put(127.5,108.5){\line(0,1){39.5}}
		\put(50,108.5){\line(1,0){77.5}}
		\put(50,108.5){\vector(0,-1){1.5}}
		\put(75,110){\line(0,-1){10}}
		\put(75,105){\vector(-1,0){23}}
		\put(50,105){\makebox(0,0)[cc]{$f$}}
		\put(50,105){\circle{4}}
		\put(48,105){\vector(-1,0){21}}
		\put(25,110){\vector(0,-1){3}}
		\put(25,103){\line(0,1){4}}
		\put(23,105){\line(1,0){4}}
		\put(25,105){\circle{4}}
		\put(25,103){\line(0,-1){3}}
		\put(25,100){\line(0,-1){3.5}}
		\put(75,100){\line(0,-1){3.5}}
		\put(75,88){\line(-6,1){50}}
		\put(25,88){\line(6,1){50}}
		\put(25,88){\vector(0,-1){3}}
		\put(75,88){\vector(0,-1){3}}
		\put(55,80){\framebox(45,5)[cc]{$R_2 = L_1 \oplus f(R_1,\key_2)$}}
		\put(0,80){\framebox(45,5)[cc]{$L_2 = R_1$}}
		\put(110,80){\makebox(0,0)[cc]{$\key_3$}}
		\put(115,76.5){\line(2,1){8}}
		\put(119,75){\makebox(0,0)[cc]{48}}
		\put(130,78.5){\line(0,1){69.5}}
		\put(50,78.5){\line(1,0){80}}
		\put(50,78.5){\vector(0,-1){1.5}}
		\put(75,80){\vector(0,-1){10}}
		\put(75,75){\vector(-1,0){23}}
		\put(50,75){\makebox(0,0)[cc]{$f$}}
		\put(50,75){\circle{4}}
		\put(48,75){\vector(-1,0){21}}
		\put(25,80){\vector(0,-1){3}}
		\put(25,73){\line(0,1){4}}
		\put(23,75){\line(1,0){4}}
		\put(25,75){\circle{4}}
		\put(25,73){\vector(0,-1){3}}
		\put(74.5,65){\makebox(0,0){ \Huge $\vdots$}}
		\put(50,65){\makebox(0,0){Weitere 12 Runden}}
		\put(24.5,65){\makebox(0,0){ \Huge $\vdots$}}
		\put(25,60){\line(0,-1){3.5}}
		\put(75,60){\line(0,-1){3.5}}
		\put(75,48){\line(-6,1){50}}
		\put(25,48){\line(6,1){50}}
		\put(25,48){\vector(0,-1){3}}
		\put(75,48){\vector(0,-1){3}}
		\put(55,40){\framebox(45,5)[cc]{$R_{15} = L_{14} \oplus f(R_{14},\key_{15})$}}
		\put(0,40){\framebox(45,5)[cc]{$L_{15} = R_{14}$}}
		\put(110,40){\makebox(0,0)[cc]{$\key_{16}$}}
		\put(115,36.5){\line(2,1){8}}
		\put(119,35){\makebox(0,0)[cc]{48}}
		\put(140,38.5){\line(0,1){109.5}}
		\put(50,38.5){\line(1,0){90}}
		\put(50,38.5){\vector(0,-1){1.5}}
		\put(75,40){\vector(0,-1){10}}
		\put(75,35){\vector(-1,0){23}}
		\put(50,35){\makebox(0,0)[cc]{$f$}}
		\put(50,35){\circle{4}}
		\put(48,35){\vector(-1,0){21}}
		\put(25,40){\vector(0,-1){3}}
		\put(25,33){\line(0,1){4}}
		\put(23,35){\line(1,0){4}}
		\put(25,35){\circle{4}}
		\put(25,33){\vector(0,-1){3}}
		\put(55,25){\framebox(45,5)[cc]{$R_{16} = R_{15}$}}
		\put(0,25){\framebox(45,5)[cc]{$L_{16} = L_{15} \oplus f(R_{15},\key_{16})$}}
		\put(75,25){\line(0,-1){3}}
		\put(25,25){\line(0,-1){3}}
		\put(51,22){\line(1,0){24}}
		\put(25,22){\line(1,0){24}}
		\put(50,22){\circle{2}}
		\put(50,21){\vector(0,-1){6}}
		\put(0,11){\framebox(100,4)[cc]{$IP^{-1}$}}
		\put(46,6){\line(2,1){8}}
		\put(56,8){\makebox(0,0){64}}
		\put(50,11){\vector(0,-1){7}}
		\put(0,0){\framebox(100,4)[cc]{$\ciphert_i$}}
		\end{picture}
	\end{center}
	\caption{Struktur des DES}
	\label{fig:desprinciple}
\end{figure}
\bigskip

Betrachten wir nun den Aufbau von DES etwas genauer. Von den insgesamt 64 Bit des Schlüssels können nur 56 Bits frei gewählt werden. Die verbleibenden 8 Bit sind Paritätsbits und dienen der Fehlererkennung. Somit umfasst der Schlüsselraum insgesamt (nur) $2^{56}\approx 7,2\cdot 10^{16}$ mögliche Schlüsselkandidaten.

Bevor Verschlüsselt werden kann, wird die Nachricht in jeweils 64-Bit große Blöcke aufgeteilt. Jeder dieser Blöcke wird zunächst einer Initialpermutation $IP$ unterzogen, die die einzelnen Bits lediglich umordnet. Die Initialpermutation bietet daher keinerlei kryptographische Sicherheit, sondern dient der effizienten Nutzung der Hardware. Anschließend durchlaufen die Nachrichtenblöcke jeweils 16 Verschlüsselungsrunden, wobei jede Runde  einen unterschiedlichen, 48-Bit langen, Schlüssel verwendet, der sich aus den 56 Bit des Hauptschlüssels ergibt. Die Rundenfunktion $f$ bleibt hingegen gleich. Auf das Ergebnis der letzten Runde wird die zu $IP$ inverse Permutation $IP^{-1}$ angewandt.
%Der prinzipielle Aufbau des DES ist in Abb.~\ref{fig:desprinciple} dargestellt. Jeder Nachrichtenblock wird erst der Eingangspermutation $IP$ unterzogen und
%durchläuft anschließend 16 schlüsselabhängige, aber funktional identische Runden. Im Unterschied zur obigen allgemeineren Beschreibung einer \textit{Feistel}-Struktur, ist also die beim DES verwendete Rundenfunktion in jeder Runde die selbe. Jede der Runden benutzt unterschiedliche
%48 Bit als Schlüssel; diese werden als Teilschlüssel aus den 56 Bit ausgewählt. Auf das Ergebnis der letzten Iteration wird zum Schluss die zu $IP$
%inverse Permutation $IP^{-1}$ angewandt.
%
%Der Algorithmus ist so ausgelegt, dass sämtliche Schlüssel als gleichwahrscheinlich zu betrachten sind; au\ss{}erdem ist er geeignet, sowohl Klartext zu
%chiffrieren, als auch chiffrierten Text zu dechiffrieren.
DES ist dabei so konstruiert, dass die Ver- und Entschlüsselung bis auf die Reihenfolge der verwendeten Teilschlüssel identisch sind. Um das Chiffrat zu generieren, werden konsekutiv die Teilschlüssel $K_1,K_2,\ldots,K_{15},K_{16}$ verwendet, während der Entschlüsselungsvorgang die umgedrehte Reihenfolge $K_{16},K_{15},\ldots,K_2,K_1$ der Teilschlüssel nutzt.

Nachdem wir grob den Ablauf der gesamten Ver- und Entschlüsselung betrachtet haben, möchten wir nun die einzelnen Runden genauer beleuchten: Nach Anwenden der Initialpermutation wird der Eingabeblock in zwei Hälften geteilt. In jeder Runde $i$, $i \in \{1,\ldots,15\}$ wird die rechte Hälfte $R_{i-1}$ mit Hilfe des Teilschlüssels $K_i$ durch die Rundenfunktion $f$ verschlüsselt. Das Ergebnis von $f$ wird mit der linken Hälfte $L_{i-1}$ XOR-verknüpft und ergibt die rechte Hälfte $R_i$ der nächsten Runde. Die rechte Hälfte $R_{i-1}$ wird hingegen unverändert als linke Hälfte $L_{i}$ übernommen.

%Formeln einfügen %

%\begin{description}
%	\item[Initialpermutation] Der 64-Bit Eingabeblock wird zunächst einer Initialpermutation $IP$ unterworfen. Dabei werden die 64 Bit lediglich umgeordnet. Dieser Vorgang bietet keine nennenswerte kryptographische Sicherheit und dient lediglich einer effizienteren Nutzung der Hardware.
%	\item[Aufteilung] Der Eingabeblock wird nun in 2 Hälften geteilt: Eine linke und eine rechte Hälfte, die jeweils 32 Bit enthalten.
%	\item[Runden] Nun wird die rechte Hälfte mit einem Teilschlüssel durch die Rundenfunktion $f$ verschlüsselt und durch XOR mit der linken Hälfte verbunden.
%	Hieraus ergibt sich die neue rechte Hälfte. Die Iteration der linke Hälfte ist schlicht die rechte Hälfte der vorherigen Runde. Die Rundenfunktion $f$
%	arbeitet wie in Abbildung \ref{fig:desround} dargestellt: Der 32-Bit-Datenblock wird zunächst auf 48 Bit expandiert, indem einige Bits verdoppelt werden. Danach wird
%	der entstandene Block durch bitweises XOR mit dem Teilschlüssel verbunden. Dieser 48-Bit-Block wird nun auf verschiedene S-Boxen aufgeteilt, die jeweils aus einer 6-Bit-Eingabe eine 4-Bit-Ausgabe produzieren. In Tabelle \ref{ssec:des:tbl:s-box} ist auszugsweise eine verwendete \textit{S-Box} dargestellt. Konkateniert man nun die Ergebnisse der S-Boxen, erhält man einen 32-Bit-Block, der das Ergebnis der Rundenfunktion darstellt. Dieser Vorgang wird 16 mal wiederholt, wobei in der letzten Runde das Ergebnis der Berechnungen nun die linke Hälfte einnimmt und die rechte Hälfte die rechte Hälfte der zweitletzten Runde übernimmt.
%	\item[Zusammenführen] Nun werden die beiden Hälften durch einfache Konkatenation wieder zu einem 64-Bit-Block zusammengeführt.
%	\item[Ausgabepermutation] Schließlich werden die Bits erneut permutiert, um die ursprüngliche Ordnung wieder herzustellen.
%\end{description}

\begin{figure}[h]
	\begin{center}
		\unitlength=1mm
		\linethickness{0.4pt}
		\begin{picture}(96,85)
		\put(4,77){\framebox(32,5){$R_{i-1}$}}
		\put(16,70){\line(2,1){8}}
		\put(26,72){\makebox(0,0){32}}
		\put(20,67){\line(0,1){10}}
		\put(4,62){\framebox(32,5){Expansion $E$}}
		\put(16,55){\line(2,1){8}}
		\put(26,57){\makebox(0,0){48}}
		\put(20,52){\line(0,1){10}}
		\put(60,62){\framebox(32,5){$\key_i$}}
		\put(72,55){\line(2,1){8}}
		\put(82,57){\makebox(0,0){48}}
		\put(76,52){\line(0,1){10}}
		\put(76,52){\vector(-1,0){25}}
		\put(20,52){\vector(1,0){25}}
		\put(45,52){\line(1,0){6}}
		\put(48,52){\circle{6}}
		\put(48,55){\line(0,-1){13}}
		\put(1,42){\line(1,0){94}}
		\multiput(1,42)(2,0){48}{\vector(0,-1){5}}
		\put(85,32){\framebox(10,5)[cc]{$S8$}}
		\put(73,32){\framebox(10,5)[cc]{$S7$}}
		\put(61,32){\framebox(10,5)[cc]{$S6$}}
		\put(49,32){\framebox(10,5)[cc]{$S5$}}
		\put(37,32){\framebox(10,5)[cc]{$S4$}}
		\put(25,32){\framebox(10,5)[cc]{$S3$}}
		\put(13,32){\framebox(10,5)[cc]{$S2$}}
		\put(1,32){\framebox(10,5)[cc]{$S1$}}
		\multiput(1.5,27)(3,0){32}{\line(0,1){5}}
		\put(1.5,27){\line(1,0){93}}
		\put(44,20){\line(2,1){8}}
		\put(54,22){\makebox(0,0){32}}
		\put(48,27){\vector(0,-1){9}}
		\put(30,12){\framebox(36,6){Permutation $P$}}
		\put(48,12){\vector(0,-1){6}}
		\put(30,0){\framebox(36,6){$f(R_{i-1},\key_i)$}}
		\end{picture}
	\end{center}
	\caption{Die Rundenfunktion $f$ des DES}
	\label{fig:desround}
\end{figure}

\begin{table}[h]
	\hspace{-0.5cm}
	\begin{tabularx}{0.8\textwidth}{ r | >{\bfseries}Br | Cc | Cc | Cc | Cc | Cc | Cc | Cc |} 
		%\hline
		\multicolumn{2}{c}{}
		& \multicolumn{7}{c}{$\overbrace{\rule{8.6cm}{0pt}}^\text{\normalsize Input-Bits 0-3}$} \\
		\cline{2-9}
		& &  \textbf{ $\cdots$ } & \rowstyle{\bfseries}0100 & 0101 & 0110 & 0111 & 1000 & \textbf{$\cdots$} \\ 
		\cline{2-9}
		\ldelim\{{4}{2.7cm}[Input-Bits 4-5]
		& 00 & $\cdots$ & 0111 & 1010 & 1011 & 0110 & 1000 & $\cdots$ \\ 
		\cline{2-9}
		& 01 & $\cdots$ & 0100 & 0111 & 1101 & 0001 & 0101 & $\cdots$ \\ 
		\cline{2-9}
		& 10 & $\cdots$ & 1010 & 1101 & 0111 & 1000 & 1111 & $\cdots$ \\ 
		\cline{2-9}
		& 11 & $\cdots$ & 0001 & 1110 & 0010 & 1101 & 0110 & $\cdots$ \\
		\cline{2-9}
	\end{tabularx}
	\caption{Ein Auszug der 5. \textit{S-Box} des DES}
	\label{ssec:des:tbl:s-box}
\end{table}

Wie schon erwähnt, kann der DES-Algorithmus sowohl zum Chiffrieren als auch zum Dechiffrieren verwendet werden. Zum Dechiffrieren wird der chiffrierte
Textblock genau derselben Prozedur wie beim Chiffrieren unterzogen, wobei die $K_i$ in umgekehrter Reihenfolge angewendet werden. Der erste Dechiffrierschritt
ist dann $IP$, der den letzten Chiffrierschritt $IP^{-1}$ aufhebt. Danach dient $R_{16}L_{16}$ als permutierter Eingabeblock, und die einzelnen $R_i$ bzw.
$L_i$ erhält man durch:
\begin{eqnarray*}
	R_{i-1}	& = 	& L_i \\
	L_{i-1}	& = 	& R_i\oplus f(L_i,K_i)
\end{eqnarray*}
Wendet man auf $L_0R_0$ dann noch $IP^{-1}$ an, so hebt man den Chiffrierschritt $IP$ auf; der Klartextblock ist somit zurückgewonnen.

Der DES ist strukturell nahezu ungebrochen: Es gibt Angriffe durch \hyperref[sssec:linKryptoanalyse]{\textit{lineare Kryptoanalyse}}, die besser sind als die vollständige Suche des Schlüsselraums, allerdings sind diese nicht praktikabel. Problematisch ist allerdings der, für heutige Verhältnisse, mit 56 Bit kleine Schlüsselraum, der Brute-Force Attacken in akzeptabler Zeit zulässt. Schon in den 90er-Jahren gelang es, Maschinen zu konstruieren, die eine Brute-Force Attacke erfolgreich innerhalb eines Tages durchführten\footnote{In den letzten Jahren entwickelte Maschinen haben nicht nur die erforderliche Zeit für Brute-Force Angriffe weiter reduziert, sondern auch die Produktionskosten gesenkt. So wurde 2006 von den Universitäten Bonn und Kiel der Computer \textit{COPACOBANA} gebaut, der insgesamt nur noch knapp 9000 \euro{} in der Produktion kostete.}. Konsequenterweise zog die NIST den DES Standard daraufhin 2005 zurück und empfiehlt nur noch die Verwendung von \hyperref[sssec:3des]{3-DES} für die Verschlüsselung von sensitiven Informationen\footnote{Diese Empfehlung gilt aktuell nur bis zum Jahr 2030 und soll den Übergang zum \textit{AES} erleichtern, der der eigentliche Nachfolger des DES ist.}.

\subsubsection{2DES}
Die naive Lösung des Schlüsselproblems beim DES ist der 2DES. Hierbei wird der Klartextblock zwei mal mit verschiedenen Schlüsseln per DES verschlüsselt, wodurch
man sich die doppelte Schlüsselbitanzahl erhofft.

\begin{figure}[h]
	\begin{center}
		\unitlength=1mm
		\linethickness{0.4pt}
		\begin{picture}(120,20)
		\put(0,5){\vector(1,0){20}}
		\put(10,6){\makebox(0,0)[cb]{$\plaint_i$}}
		\put(35,15){\vector(0,-1){7.5}}
		\put(38,10){\makebox(0,0)[cb]{$\key_0$}}
		\put(20,2.5){\framebox(30,5){$\enc_{DES}$}}
		\put(50,5){\vector(1,0){20}}
		\put(60,6){\makebox(0,0)[cb]{$\ciphert_i$}}
		\put(85,15){\vector(0,-1){7.5}}
		\put(88,10){\makebox(0,0)[cb]{$\key_1$}}
		\put(70,2.5){\framebox(30,5){$\enc_{DES}$}}
		\put(100,5){\vector(1,0){20}}
		\put(110,6){\makebox(0,0)[cb]{$\widetilde \ciphert_i$}}
		\end{picture}
	\end{center}
	\caption{Prinzip des 2DES}
	\label{fig:2des}
\end{figure}

Leider ist der 2DES nicht so effektiv wie erwartet, da es sogenannte Meet-in-the-Middle-Angriffe gibt. Unter der Voraussetzung, dass man ein
Klartext-Chiffrat-Paar besitzt, ist die Vorgehensweise wie folgt:
\begin{enumerate}
	\item Erstelle eine Liste aller möglichen im ersten Schritt erzeugbaren Chiffrate $\ciphert_{\key_0} = \enc_{DES_{\key 0}}$, d.h. verwende alle möglichen
	$\key_0$.
	\item Sortiere diese Liste lexikographisch um binäre Suche zu ermöglichen.
	\item Berechne alle möglichen Chiffrate $\ciphert_{\key_1}= \dec_{DES_{\key 1}}$
	\item Falls nun ein Paar $\ciphert_{\key_0} = \ciphert_{\key_1}$ gibt, gebe jeweils den Schlüssel $\key_0$ und $\key_1$ aus.
\end{enumerate}
Ein solcher Angriff besitzt einen Platzbedarf von $64 \cdot 2^{56} + \epsilon$ und hat eine Laufzeit in $O(56 \cdot 2^{56}$). 2DES bietet also einen sehr geringen Vorteil gegenüber einfachem DES.

\subsubsection{3DES}
\label{sssec:3des}
Die direkte Erweiterung des 2DES ist der 3DES. Wie der Name bereits verrät werden hier 3 DES-Verschlüsselungen verwendet. Allerdings wird die mittlere DES-Verschlüsselung umgekehrt verwendet, d.h. bei der Verschlüsselung ist diese im Entschlüsselungsmodus und umgekehrt.

\begin{figure}[h]
	\begin{center}
		\unitlength=1mm
		\linethickness{0.4pt}
		\begin{picture}(140,20)
		\put(0,5){\vector(1,0){20}}
		\put(10,6){\makebox(0,0)[cb]{$\plaint_i$}}
		\put(30,15){\vector(0,-1){7.5}}
		\put(33,10){\makebox(0,0)[cb]{$\key_0$}}
		\put(20,2.5){\framebox(20,5){$\enc_{DES}$}}
		\put(40,5){\vector(1,0){20}}
		\put(50,6){\makebox(0,0)[cb]{$\ciphert_i$}}
		\put(70,15){\vector(0,-1){7.5}}
		\put(73,10){\makebox(0,0)[cb]{$\key_1$}}
		\put(60,2.5){\framebox(20,5){$\dec_{DES}$}}
		\put(80,5){\vector(1,0){20}}
		\put(90,6){\makebox(0,0)[cb]{$\widetilde \ciphert_i$}}
		\put(110,15){\vector(0,-1){7.5}}
		\put(113,10){\makebox(0,0)[cb]{$\key_2$}}
		\put(100,2.5){\framebox(20,5){$\enc_{DES}$}}
		\put(120,5){\vector(1,0){20}}
		\put(130,6){\makebox(0,0)[cb]{$\widehat \ciphert_i$}}
		\end{picture}
	\end{center}
	\caption{Prinzip des 3DES}
	\label{fig:3des}
\end{figure}

Ein Meet-in-the-Middle-Angriff ist hier zwar noch möglich, aber bereits weit weniger praktikabel: Die Laufzeit befindet sich in $O(2^{112})$.

\subsubsection{AES - Advanced Encryption Standard}
%Der AES ein viel verwendetes Verschlüsselungssystem, welches nicht auf die durch den DES geprägte Feistel-Struktur sondern auf ein Permutations- und
%Rotationsnetzwerk setzt. Ebenfalls anders als der DES besitzt der AES verschiedene Schlüssellängen: 128, 196 und 256 werden als Schlüssellängen unterstützt.

%Nach heutigem Kenntnisstand ist der AES noch sicher. Genauer gesagt gibt es Kryptoanalysen, welche schneller sind als vollständige Suche über den
%Schlüsselraum, allerdings noch immer sehr unpraktikabel sind. Bei einem 256-AES werden noch immer $2^{254}$ Schritte benötigt.

%TODO: Wollen wir wirklich alles italic machen?
% Wie sieht es mit confusion und diffusion aus?
%First Outline of AES%
Im Jahr 2000 stellte das \textit{National Institute of Standards and Technology} (NIST) den \textit{Advanced Encryption Standard} (AES) als Nachfolger des DES vor, nachdem drei Jahre zuvor ein offener Wettbewerb, um die alte Blockchiffre zu ersetzen, ausgeschrieben worden war.
Den eigentliche Sieger des Wettbewerbs, der \textit{Rijndael-Algorithmus}, hatte man dabei nur in einigen wenigen unwesentlichen Punkten angepasst.

Im Gegensatz zu DES verschlüsselt AES jeweils 128 Bit große Datenblöcke, wohingegen die Schlüssellänge aus 128 Bit, 192 Bit und 256 Bit gewählt werden kann. Dementsprechend bezeichnet AES-256 genau die Variante mit der größten Schlüssellänge. Der 128 Bit große Datenblock wird zu Beginn sequenziell in eine zweidimensionale 4x4-Tabelle, den sogenannten \textit{state}, geschrieben. Jede Zelle des \textit{states} repräsentiert dabei genau ein Byte des Klartextblocks. Ähnlich zu DES, läuft die Verschlüsselung bei AES rundenbasiert ab, wobei eine Runde jeweils aus den vier folgenden Schritten besteht:

\begin{enumerate}
	\item \textbf{AddRoundKey} \small{Der aus dem Hauptschlüssel abgeleitete, ebenfalls 128 Bit lange Rundenschlüssel wird byteweise mit dem \textit{state} XOR-verknüpft}
	\item \textbf{SubBytes} \small{Benutze die \textit{S-Box}, um jedes Byte der zweidimensionalen Tabelle durch ein anderes Byte zu ersetzen}
	\item \textbf{ShiftRows} \small{Rotiere die zweite Zeile zyklisch um ein Byte, die dritte Zeile zyklisch um zwei Byte und die vierte Zeile zyklisch um drei Byte nach links}
	\item \textbf{MixColumns} \small{Wende auf jede Spalte eine invertierbare lineare Transformation\footnote{Vereinfacht kann man sich unter der invertierbaren linearen Transformation eine Matrixmultiplikation auf einer speziellen Struktur vorstellen. Wichtig ist insbesondere die Invertierbarkeit. Genauere Details möchten wir an dieser Stelle jedoch nicht besprechen. Bei weiterführendem Interesse bietet der \href{http://csrc.nist.gov/publications/fips/fips197/fips-197.pdf}{Standard der NIST} einen formalen, aber aufschlussreichen Einblick.} an}
\end{enumerate}

Für die Verschlüsselung greift AES also auf Operationen zurück, die man in ihrer Ganzheit treffend als Substitutions- und Rotationsnetzwerk beschreiben kann. Wir sehen, dass S-Boxen nicht nur in der Feistel-Struktur Verwendung finden und sehr wohl auch bijektiv seien können.
Die Anzahl der Verschlüsselungsrunden ist von der gewählten Schlüssellänge abhängig: Bei 128 Bit werden 10, bei 192 Bit 12 und bei 256 Bit 14 Verschlüsselungsrunden durchlaufen. Um zu verhindern, dass ein Angreifer die letzten drei Schritte der Schlussrunde zurückrechnen kann, wird anstelle von MixColumns ein zusätzliches AddRoundKey ausgeführt.

Da die XOR-Operation, das byteweise Ersetzen mittels einer S-Box und das zyklische Rotieren jeweils invertierbare Funktionen sind und wir die Invertierbarkeit bei MixColumns explizit fordern, funktioniert das Entschlüsseln eines Chiffrats problemlos und effizient.

Nach heutigem Kenntnisstand garantiert AES ein hohes Maß an Sicherheit. Zwar gibt es theoretische Kryptoanalysen, die aber aufgrund ihrer hohen Laufzeit (für AES-256 werden $2^{254}$ Schritte benötigt) in der Praxis keine Relevanz haben. Nicht zuletzt deswegen ist AES ab der Schlüssellänge von 192 Bit in den USA noch immer zur Verschlüsselung staatlicher Dokumente der höchsten Geheimhaltungsstufe zugelassen.

\subsection{Angriffe auf Blockchiffren}

\subsubsection{Lineare Kryptoanalyse}
\label{sssec:linKryptoanalyse}
Die \textit{lineare Kryptoanalyse} stellt einen Angriff auf Blockchiffren dar, der meist besser als die vollständige Suche des Schlüsselraums ist. Das Ziel dieser Angriffsmethode ist eine lineare Abhängigkeit für die Verschlüsselung zu bestimmen:
\begin{equation}
	P \left[ i_1, i_2, \ldots, i_a \right] \oplus C \left[ j_1, j_2, \ldots, j_b \right]  = K \left[ k_1, k_2, \ldots, k_c \right], \label{approx}
\end{equation}
wobei  $i_1, i_2, \ldots, i_a, j_1, j_2, \ldots, j_b,k_1, k_2, \ldots, k_c$ feste Bitpositionen bezeichnen und die Gleichung mit einer Wahrscheinlichkeit von
$p\neq \frac{1}{2}$ für einen zufälligen Klartext $\plaint$ und den zugehörigen Chiffretext $\ciphert$ gilt. Die Größe $\left|{p-\frac{1}{2}}\right|$ gibt die
Effektivität der Gleichung~\ref{approx} an. Wenn eine effektive lineare Approximation bekannt ist, dann kann man mit der naiven Maximum-Likelihood-Methode ein
Schlüsselbit $\key \left[ \key_1, \key_2, \ldots, \key_c \right]$ erraten.

Bei Verschlüsselungssystemen, welche die \textit{Feistel}-Struktur verwenden, sehen entsprechende Angriffe wie folgt aus:
\begin{enumerate}
	\item Finde lineare Abhängigkeiten zwischen Ein- und Ausgabe
	\item Erweitere Abhängigkeiten auf die ersten n - 1 \textit{Feistel}-Runden
	\item Vollständige Suche über letzten Rundenschlüssel $\key_n$
	\item Gefundene Kandidaten durch die bekannten linearen Abhängigkeiten überprüfen
	\item Wenn $\key_n$, fahren mit $\key_{n-1}$ fort usw.
\end{enumerate}

Für den gewöhnlichen DES mit 16 Runden ist dieses Vorgehen allerdings schon wegen der immensen Anzahl an benötigten Klartext-Chiffrat-Paaren nicht praktikabel (es werden bis zu $2^{43}$ solcher Paare benötigt). Für andere Blockchiffren hingegen, die ebenfalls eine \textit{Feistel}-Struktur verwenden, beispielsweise FEAL, ist ein effizienter Angriff mit dieser Methode möglich.


\paragraph*{Beispiel: 3-Runden DES}
\begin{figure}[h]
	\begin{center}
		\unitlength=1mm
		\linethickness{0.4pt}
		\begin{picture}(145,135)
		\put(132,128){\makebox(0,0)[cc]{$\key$}}
		\put(128,118){\line(2,1){8}}
		\put(138,120){\makebox(0,0){64}}
		\put(132,125){\vector(0,-1){9}}
		\put(119,108){\framebox(26,8)[cc]{}}
		\put(132,114){\makebox(0,0)[cc]{Schlüsselaus-}}
		\put(132,110){\makebox(0,0)[cc]{wahlfunktion}}
		\put(0,129){\framebox(100,4)[cc]{$\plaint_i$}}
		\put(46,124){\line(2,1){8}}
		\put(56,125){\makebox(0,0){64}}
		\put(50,129){\vector(0,-1){7}}
		\put(0,118){\framebox(100,4)[cc]{$IP$}}
		\put(50,118){\vector(0,-1){5}}
		\put(51,112){\line(1,0){24}}
		\put(25,112){\line(1,0){24}}
		\put(50,112){\circle{2}}
		\put(21,107){\line(2,1){8}}
		\put(31,108){\makebox(0,0){32}}
		\put(25,112){\vector(0,-1){7}}
		\put(71,107){\line(2,1){8}}
		\put(81,108){\makebox(0,0){32}}
		\put(75,112){\vector(0,-1){7}}
		\put(55,100){\framebox(45,5)[cc]{$R_0$}}
		\put(0,100){\framebox(45,5)[cc]{$L_0$}}
		\put(110,100){\makebox(0,0)[cc]{$\key_{1}$}}
		\put(115,96.5){\line(2,1){8}}
		\put(119,95){\makebox(0,0)[cc]{48}}
		\put(125,98.5){\line(0,1){9.5}}
		\put(50,98.5){\line(1,0){75}}
		\put(50,98.5){\vector(0,-1){1.5}}
		\put(75,100){\line(0,-1){10}}
		\put(75,95){\vector(-1,0){23}}
		\put(50,95){\makebox(0,0)[cc]{$f$}}
		\put(50,95){\circle{4}}
		\put(48,95){\vector(-1,0){21}}
		\put(25,100){\vector(0,-1){3}}
		\put(25,93){\line(0,1){4}}
		\put(23,95){\line(1,0){4}}
		\put(25,95){\circle{4}}
		\put(25,93){\line(0,-1){3}}
		\put(25,90){\line(0,-1){3.5}}
		\put(75,90){\line(0,-1){3.5}}
		\put(75,78){\line(-6,1){50}}
		\put(25,78){\line(6,1){50}}
		\put(25,78){\vector(0,-1){3}}
		\put(75,78){\vector(0,-1){3}}
		\put(55,70){\framebox(45,5)[cc]{$R_1 = L_0 \oplus f(R_0,\key_1)$}}
		\put(0,70){\framebox(45,5)[cc]{$L_1 = R_0$}}
		\put(110,70){\makebox(0,0)[cc]{$\key_2$}}
		\put(115,66.5){\line(2,1){8}}
		\put(119,65){\makebox(0,0)[cc]{48}}
		\put(127.5,68.5){\line(0,1){39.5}}
		\put(50,68.5){\line(1,0){77.5}}
		\put(50,68.5){\vector(0,-1){1.5}}
		\put(75,70){\line(0,-1){10}}
		\put(75,65){\vector(-1,0){23}}
		\put(50,65){\makebox(0,0)[cc]{$f$}}
		\put(50,65){\circle{4}}
		\put(48,65){\vector(-1,0){21}}
		\put(25,70){\vector(0,-1){3}}
		\put(25,63){\line(0,1){4}}
		\put(23,65){\line(1,0){4}}
		\put(25,65){\circle{4}}
		\put(25,63){\line(0,-1){3}}
		\put(25,60){\line(0,-1){3.5}}
		\put(75,60){\line(0,-1){3.5}}
		\put(75,48){\line(-6,1){50}}
		\put(25,48){\line(6,1){50}}
		\put(25,48){\vector(0,-1){3}}
		\put(75,48){\vector(0,-1){3}}
		\put(55,40){\framebox(45,5)[cc]{$R_2 = L_1 \oplus f(R_1,\key_2)$}}
		\put(0,40){\framebox(45,5)[cc]{$L_2 = R_1$}}
		\put(110,40){\makebox(0,0)[cc]{$\key_3$}}
		\put(115,36.5){\line(2,1){8}}
		\put(119,35){\makebox(0,0)[cc]{48}}
		\put(130,38.5){\line(0,1){69.5}}
		\put(50,38.5){\line(1,0){80}}
		\put(50,38.5){\vector(0,-1){1.5}}
		\put(75,40){\vector(0,-1){10}}
		\put(75,35){\vector(-1,0){23}}
		\put(50,35){\makebox(0,0)[cc]{$f$}}
		\put(50,35){\circle{4}}
		\put(48,35){\vector(-1,0){21}}
		\put(25,40){\vector(0,-1){3}}
		\put(25,33){\line(0,1){4}}
		\put(23,35){\line(1,0){4}}
		\put(25,35){\circle{4}}
		\put(25,33){\vector(0,-1){3}}
		\put(55,25){\framebox(45,5)[cc]{$R_{3} = R_{2}$}}
		\put(0,25){\framebox(45,5)[cc]{$L_{3} = L_{2} \oplus f(R_{2},\key_{3})$}}
		\put(75,25){\line(0,-1){3}}
		\put(25,25){\line(0,-1){3}}
		\put(51,22){\line(1,0){24}}
		\put(25,22){\line(1,0){24}}
		\put(50,22){\circle{2}}
		\put(50,21){\vector(0,-1){6}}
		\put(0,11){\framebox(100,4)[cc]{$IP^{-1}$}}
		\put(46,6){\line(2,1){8}}
		\put(56,8){\makebox(0,0){64}}
		\put(50,11){\vector(0,-1){7}}
		\put(0,0){\framebox(100,4)[cc]{$\ciphert_i$}}
		\end{picture}
	\end{center}
	\caption{Darstellung eines DES mit 3 Runden}
	\label{fig:des3rounds}
\end{figure}

Bei einem wie in Abbildung~\ref{fig:des3rounds} dargestellten DES beginnt man zunächst damit die bekannten, aber nicht linearen S-Boxen linear zu approximieren, d.h. einen linearen Zusammenhang zwischen den Eingangs- und
Ausgangsbits einer S-Box zu finden. Dies wurde bereits in~\cite{Mats93} getan und übersteigt den Umfang dieser Vorlesung, jedoch erhält man hierdurch wichtige Zusammenhänge wie:

\emph{Bei S-Box $S_5$ gilt, dass das Eingabebit vier (von rechts mit null beginnend gezählt) in 12 der 64 Fälle (Eingaben) mit dem XOR der vier Ausgabebits
	überein stimmt.}

Falls es Gleichungen gibt, welche für ungleich 32 der 64 Fälle gelten, so gibt es eine Korrelation der Eingabe- und Ausgabebits der S-Box. Das obige Beispiel ist die größte bekannte Abweichung, da sie nur in 12 der 64 Fällen gilt. Berücksichtigt man nun die Permutation $P$ und die Expansion $E$ in der Rundenfunktion $f$, ergibt sich daraus die Gleichung:
\begin{equation}
	R_i\left[ 15\right] \oplus f(R_i,\key_i)\left[ 7,18,24,29\right] = \key_i\left[ 22\right]
	\label{s5relation}
\end{equation}

Die Gleichung~\ref{s5relation} gilt also in 12 der 64 Fälle und hat damit eine Wahrscheinlichkeit von $\approx 0,19$. Wenn man diese Gleichung~\ref{s5relation}
auf die erste Runde anwendet, dann sieht man, dass

\begin{equation}
	R_1\left[ 7,18,24,29\right] \oplus L_0\left[ 7,18,24,29\right] \oplus R_0\left[ 15 \right] = \key_1 \left[ 22 \right]
\end{equation}
mit der Wahrscheinlichkeit $12/64$ gilt. Dasselbe gilt für die letzte Runde, also
\begin{equation}
	R_1\left[ 7,18,24,29\right] \oplus L_3\left[ 7,18,24,29\right] \oplus R_3\left[ 15 \right] = \key_3 \left[ 22 \right].
\end{equation}

Aus den beiden vorhergehenden Gleichungen erhält man folgende lineare Approximation des 3-Runden-DES, indem man die beiden Gleichungen addiert und dabei die gemeinsamen Terme weglässt:
\begin{equation}
	L_0\left[ 7,18,24,29\right] \oplus L_3\left[ 7,18,24,29\right] \oplus R_0\left[ 15 \right] \oplus R_3\left[ 15 \right] = \key_1 \left[ 22 \right] \oplus \key_3\left[ 22 \right].
	\label{3roundsapprox}
\end{equation}
Die Wahrscheinlichkeit, dass die obige Gleichung für einen zufälligen Klartext $\plaint$ und den zugehörigen Chiffretext $\ciphert$ gilt, beträgt
$(12/64)^2+(1-12/64)^2 \approx 0,70$. Da die Gleichung~\ref{s5relation} die beste Approximation der $F$-Funktion ist, ist damit die
Gleichung~\ref{3roundsapprox} die beste Approximation für den 3-Runden-DES. Man kann jetzt statistisch (Maximum-Likelyhood-Methode)
Gleichung~\ref{3roundsapprox} lösen, um $K_1\left[ 22\right] \oplus K_3\left[ 22\right]$ zu erhalten.

Das bedeutet, dass man nun Schlüsselinformationen effizienter als mit vollständige Suche erhalten hat. Man kann mit diesen Informationen nun eine effizientere vollständige Suche für $\key_1$ und $\key_3$ durchführen, um Kandidaten zu finden und mit diesen $\key_2$ finden.


\subsubsection{Differentielle Kryptoanalyse}
\label{sssec:diffKryptoanalyse}
Anders als bei der \textit{linearen Kryptoanalyse} werden hier nicht direkte Zusammenhänge zwischen einzelnen Klartextblöcken und deren Chiffrate gesucht, sondern
indirekt durch Vergleiche zweier Blöcke miteinander: Es gilt die Auswirkungen der Differenz zweier Klartextblöcke $\plaint \oplus \plaint'$ auf die Differenz
ihrer Chiffrate $\dec_\key(\plaint) \oplus \dec_\key(\plaint')$ zu finden.

Für Feistel-Strukturen ist die Vorgehensweise ähnlich der \textit{linearen Kryptoanalyse}:
\begin{enumerate}
	\item Finde die wahrscheinlichsten Zusammenhänge zwischen Eingabe- und Ausgabedifferenzen der vorletzten Runde.
	\item Führe vollständige Suche für $\key_n$ durch.
	\item Überprüfe Kandidaten durch Testen der Konsistenz bezüglich den Ein- und Ausgabedifferenzen.
\end{enumerate}

Wie bereits bei der \textit{linearen Kryptoanalyse} ist DES selbst sehr resistent gegenüber der \textit{differentiellen Kryptoanalyse}, während andere auf \textit{Feistel}-Struktur basierende Systeme anfällig sind. Dies ist vor
allem darauf zurückzuführen, dass diese Resistenz ein Entwicklungsziel des DES war, obwohl dieser Angriff erst ein Jahrzehnt später veröffentlicht wurde.

\subsection{Betriebsmodi}
Da die Verschlüsselungsfunktion $\enc$ stets Blöcke einer festen Länge erwartet, die Nachrichten allerdings variierende Länge besitzen, gibt es zum Verschlüsseln der Nachrichten mehrere Möglichkeiten, sogenannte Betriebsmodi.

\subsubsection{ECB - Electronic Codebook}
\begin{figure}[h]
\centering
\subfigure[Verschlüsselung im ECB]{
\tikzstyle{vertex}=[draw,fill=black!15,circle,minimum size=20pt,inner sep=0pt]
\tikzstyle{encrypt}=[draw,fill=black!15,rectangle,minimum size=20pt,inner sep=0pt]
\begin{tikzpicture}
    \newcommand{\n}{3}
    \foreach \nr in {1, ..., \n}{
        \node (M\nr) at (0,{(\n-\nr)*2}) {$\plaint_\nr$};
        \node (C\nr) at (4,{(\n-\nr)*2}) {$\ciphert_\nr$};
        \node (E\nr)[encrypt] at (2,{(\n-\nr)*2}) {$\enc$};
        \node (K\nr) at (2,{(\n-\nr)*2+1}) {$\key$};

        \draw[->,very thick] (M\nr) -- (E\nr);
        \draw[->,very thick] (K\nr) -- (E\nr);
        \draw[->,very thick] (E\nr) -- (C\nr);
    }
\end{tikzpicture}
  }
\hfill
\subfigure[Entschlüsselung im ECB]{
\tikzstyle{vertex}=[draw,fill=black!15,circle,minimum size=20pt,inner sep=0pt]
\tikzstyle{encrypt}=[draw,fill=black!15,rectangle,minimum size=20pt,inner sep=0pt]
  \begin{tikzpicture}
    \newcommand{\n}{3}
    \foreach \nr in {1, ..., \n}{
        \node (M\nr) at (0,{(\n-\nr)*2}) {$\ciphert_\nr$};
        \node (C\nr) at (4,{(\n-\nr)*2}) {$\plaint_\nr$};
        \node (E\nr)[encrypt] at (2,{(\n-\nr)*2}) {$\dec$};
        \node (K\nr) at (2,{(\n-\nr)*2+1}) {$\key$};

        \draw[->,very thick] (M\nr) -- (E\nr);
        \draw[->,very thick] (K\nr) -- (E\nr);
        \draw[->,very thick] (E\nr) -- (C\nr);
    }
\end{tikzpicture}
}
\caption{Skizze der Verschlüsselung einer Nachricht $\plaint = M_1 M_2 M_3$ sowie der Entschlüsselung des zugehörigen Chiffrats $\ciphert = C_1 C_2 C_3$ im
Electronic Codebook Mode. Grafik erstellt von Martin Thoma \cite{martinthoma}.}
\end{figure}

Beim ECB wird der Klartext in Blöcke mit einer festen Länge von $n\,bits$ aufgeteilt und jeder Block unabhängig von den anderen einzeln verschlüsselt. In den
letzten Block wird dabei zusätzlich die Gesamtzahl von Blöcken codiert und falls notwendig, wird dieser Block, um die Blockgröße zu erhalten, z.B. mit Nullen
oder besser noch mit einer Zufallsfolge aufgefüllt. Identische Klartextblöcke liefern damit auch identische Chiffretextblöcke; daher wird diese Betriebsart in
Analogie zu einem Code"=Buch als Electronic Codebook Mode bezeichnet.

Diese Betriebsart hat im Hinblick auf die Sicherheit wenigstens zwei Nachteile:

\begin{itemize}
  \item[$-$] Da gleiche Klartextblöcke, verschlüsselt mit dem gleichen Schlüssel, zu gleichen Chiffretextblöcken führen, kann ein passiver Angreifer
  Information über den Klartext folgern, obwohl der Angreifer die Blöcke selbst nicht entschlüsseln kann.
  \item[$-$] Der zweite, schwerwiegendere Nachteil ist darin zu sehen, dass ein Angreifer den Chiffretext selbst ändern kann, ohne dass der Empfänger der
  Nachricht dies bemerkt. Chiffretextblöcke, die mit dem gleichen Schlüssel chiffriert und bei vorausgegangenen Übertragungen aufgezeichnet wurden, könnten
  z.B. eingefügt werden, um den Sinn einer Nachricht zu ändern.
\end{itemize}

Aufgrund dieser Nachteile ist der ECB-Modus ungeeignet, um lange Nachrichten zu verschlüsseln. Tritt ein Bitfehler bei der Übertragung in Block $\ciphert_i$ auf, so ist wegen der Unabhängigkeit der Chiffretextblöcke untereinander nur der Block
$\ciphert_i$ gestört, d.h. bei der Dechiffrierung erhält man i. allg. einen total gestörten Klartextblock. Alle folgenden Blöcke werden wieder korrekt
dechiffriert. Es gibt also keine Fehlerfortpflanzung.

\subsubsection{CBC - Cipher Block Chaining}
\label{cbc}

Im CBC-Modus wird eine Nachricht genau wie im ECB-Modus zuerst in Blöcke gleicher Länge zerlegt. Wie in Abbildung~\ref{pc:cbc} gezeigt, benutzt das CBC-Verfahren die Ausgabe eines jeden Chiffrierschrittes, um den folgenden Block "`vorzuchiffrieren``.
Für Anwendungen, wie die Festplattenverschlüsselung, ist es daher problematisch, auf CBC zu setzen: Zwar ist ein wahlfreier Lesezugriff - also das Entschlüsseln - auf den Chiffratblock $C_i$ mit Kenntnis von $C_{i-1}$ möglich, jedoch müssen für das Schreiben eines Blocks $M_i$ alle Klartextblöcke neuverschlüsselt werden müssen. In der Praxis gibt es dennoch Varianten der Festplattenverschlüsselung, die CBC nutzen.
Beispielsweise löst der \textit{Linux Unified Key Setup (LUKS)} das Problem, indem 512 Byte große Datenblöcke jeweils einzeln verschlüsselt werden.

\begin{figure}[h!]
\subfigure[Verschlüsselung im CBC]{
  \tikzstyle{encrypt}=[draw,fill=black!15,rectangle,minimum size=20pt,inner sep=0pt]
\begin{tikzpicture}
    \newcommand{\n}{3}
    \foreach \nr in {1, ..., \n}{
        \node (M\nr)            at (0,{(\n-\nr)*2}) {$\plaint_\nr$};
        \node (x\nr)[XOR]       at (2,{(\n-\nr)*2}) {};
        \node (E\nr)[encrypt]   at (4,{(\n-\nr)*2}) {$\enc$};
        \node (C\nr)            at (6,{(\n-\nr)*2}) {$\ciphert_\nr$};

        \node (K\nr)            at (4,{(\n-\nr)*2+1}) {$\key$};

        \draw[->,very thick] (M\nr) -- (x\nr);
        \draw[->,very thick] (x\nr) -- (E\nr);
        \draw[->,very thick] (E\nr) -- (C\nr);

        \draw[->,very thick] (K\nr) -- (E\nr);
    }

    \foreach \nr in {2, ..., \n}{
        \pgfmathtruncatemacro{\tmp}{\nr-1}
        \draw[->,very thick] (E\tmp) -- (4, {(\n-\tmp)*2-0.5}) -- (2, {(\n-\nr)*2+1.5}) -- (x\nr);
    }

    \node (IV) at (2,{\n*2-1}) {$IV$};
    \draw[->, very thick] (IV) -- (x1);
\end{tikzpicture}
  }
\subfigure[Entschlüsselung im CBC]{
  \tikzstyle{encrypt}=[draw,fill=black!15,rectangle,minimum size=20pt,inner sep=0pt]
\begin{tikzpicture}
    \newcommand{\n}{3}
    \foreach \nr in {1, ..., \n}{
        \node (C\nr)            at (0,{(\n-\nr)*2}) {$\ciphert_\nr$};
        \node (D\nr)[encrypt]   at (2,{(\n-\nr)*2}) {$\dec$};
        \node (x\nr)[XOR]       at (4,{(\n-\nr)*2}) {};
        \node (M\nr)            at (6,{(\n-\nr)*2}) {$\plaint_\nr$};

        \node (K\nr)            at (2,{(\n-\nr)*2+1}) {$\key$};

        \draw[->,very thick] (C\nr) -- (D\nr);
        \draw[->,very thick] (D\nr) -- (x\nr);
        \draw[->,very thick] (x\nr) -- (M\nr);

        \draw[->,very thick] (K\nr) -- (D\nr);
    }

    \foreach \nr in {2, ..., \n}{
        \pgfmathtruncatemacro{\tmp}{\nr-1}
        \draw[->,very thick] (1,{(\n-\tmp)*2}) -- (1,{(\n-\tmp)*2-0.5}) -- (4,{(\n-\tmp)*2-0.5}) -- (x\nr);
    }

    \node (IV) at (4,{\n*2-1}) {$IV$};
    \draw[->, very thick] (IV) -- (x1);
\end{tikzpicture}
}
\caption{Skizze der Verschlüsselung einer Nachricht $\plaint = M_1 M_2 M_3$ sowie der Entschlüsselung des zugehörigen Chiffrats $\ciphert = C_1 C_2 C_3$ im
Cipher Block Chaining Mode. Grafik erstellt von Martin Thoma \cite{martinthoma}.}
\label{pc:cbc}
\end{figure}

Man erhält im CBC Mode die Chiffretextblöcke $\ciphert_0 ,\ciphert_1 ,\ciphert_2 ,\ldots$ durch:
\begin{eqnarray*}
  \ciphert_0	& \colon=	& IV {\rm (Initialisierungsvektor)} \\
  \ciphert_i	& \colon=	& \enc_\key(\plaint_i \oplus \ciphert_{i-1}).
\end{eqnarray*}

Der erste Block $\plaint_1$ und ein Initialisierungsvektor $IV$ (Es gibt ja zu diesem Zeitpunkt noch keinen verschlüsselten Chiffretextblock) werden bitweise
modulo~2 addiert, das Ergebnis wird dann wie im ECB Mode verschlüsselt und ergibt den ersten Chiffretextblock $\ciphert_1$. Dieser und der zweite
Nachrichtenblock $\plaint_2$ werden bitweise modulo~2 addiert, das Ergebnis wird verschlüsselt, usw. Dieser Vorgang wiederholt sich bis zum Ende der Nachricht.

Die Entschlüsselung geschieht folgenderma\ss{}en:
\begin{eqnarray*}
  \ciphert_0	& \colon=	& IV \\
  \plaint_i	& \colon=	& \dec_\key(\ciphert_i)\oplus \ciphert_{i-1}
\end{eqnarray*}

Es gilt:
\begin{eqnarray*}
  \plaint_i	& =	& \dec_\key(\ciphert_i)\oplus \ciphert_{i-1} \\
	& =	& \dec_\key \bigl(\enc_\key(\plaint_i\oplus \ciphert_{i-1})\bigr) \oplus \ciphert_{i-1} \\
	& =	& \plaint_i\oplus \ciphert_{i-1}\oplus \ciphert_{i-1} .
\end{eqnarray*}

Es hängt also jeder Chiffretextblock $\ciphert_i$ von den vorausgegangenen Blöcken $\ciphert_j,\ 1\leq j<i$, und vom Initialisierungsvektor $IV$ ab. Damit
liefern gleiche Klartextblöcke $\plaint_i$ und $\plaint_j\ (i\neq j)$ i. allg. verschiedene Chiffretextblöcke $\ciphert_i$ und $\ciphert_j$.

Die Wahl des Initialisierungsvektors $IV$ ist wichtig für die Sicherheit dieser Verschlüsselung, denn durch Änderung einzelner Bits des $IV$ können gezielt
bestimmte Bits des ersten Blockes verändert werden, der dadurch anfällig für sinnvolle Veränderungen ist.

Aufgrund der Verkettung der Chiffretextblöcke im CBC Mode muss untersucht werden, welche Auswirkungen ein Fehler bei der Übertragung eines Chiffretextblockes
hat.
\begin{itemize}
  \item Tritt ein Bitfehler im Block $\ciphert_i$ auf, so zeigt sich der in Abbildung~\ref{pc:fehl.cbc} dargestellte Effekt. Die Ausgabe bei der Dechiffrierung
  des Blockes \,$\ciphert_i$ \,ist rein zufällig, da ein gefälschtes Bit in der Eingabe die Ausgabe völlig verändert. Der ganze Klartextblock $\plaint_i$ ist
  also zerstört. Durch die Verkettung der Blöcke wird auch noch Block $\plaint_{i+1}$ in Mitleidenschaft gezogen. Im Speicher steht jetzt der
  bitfehlerbehaftete Chiffretextblock $\ciphert_i$. Durch die Addition modulo~2  wird bewirkt, dass an der Stelle des Bitfehlers im Block $\ciphert_i$ im
  Klartextblock $\plaint_{i+1}$ nun auch ein Bitfehler entsteht. Nachfolgende Blöcke werden aber nicht mehr beeinflusst.
\end{itemize}

\begin{figure}[h!]
\begin{center}
\unitlength=1mm
\linethickness{0.4pt}
%\begin{picture}(119,77.5)(45,-7.5)
\begin{picture}(119,77)(35,-7.5)

%Block i

\put(67,50){\vector(1,0){11}}
\put(45,48){\framebox(20,4){~}}
\put(52,48){\framebox(1,4){~}}
\multiput(52,48)(0,0.8){5}{\line(1,0){1}}
\put(52.5,48){\line(0,-1){8}}
\put(45,39){\makebox(0,0)[lt]{1-bit Fehler}}
\put(55,53.5){\makebox(0,0)[b]{$\ciphert_i$}}
\put(79,40){\framebox(20,20)[cc]{\large $\dec$}}
\put(74,50){\line(0,-1){15}}
\put(74,35){\line(1,0){34.5}}
\put(108.5,35){\vector(0,-1){21.5}}
\put(108.5,56.5){\vector(0,-1){3.5}}
\put(108.5,57.5){\line(0,1){2}}
\put(108.5,60.5){\line(0,1){2}}
\put(108.5,50){\circle{5}}
\put(108.5,47.5){\line(0,1){5}}
\put(106,50){\line(1,0){5}}
\put(89,70){\vector(0,-1){9}}
\put(100,50){\vector(1,0){5}}
\put(112,50){\vector(1,0){10}}
\put(124,48){\framebox(20,4){~}}
\put(144,48){\line(0,1){4}}
\multiput(124,48)(0,0.8){5}{\line(1,0){20}}
\put(134,53.5){\makebox(0,0)[b]{$\plaint_i$}}
\put(88,65){\makebox(0,0)[rc]{$\key$}}
\put(134,46){\makebox(0,0)[t]{zerstörter}}
\put(134,41){\makebox(0,0)[t]{Block}}

%Block i + 1

\put(67,10){\vector(1,0){11}}
\put(45,8){\framebox(20,4){~}}
\put(55,13.5){\makebox(0,0)[b]{$\ciphert_{i+1}$}}
\put(79,0){\framebox(20,20)[cc]{\large $\dec$}}
\put(74,10){\line(0,-1){3}}
\put(74,6){\line(0,-1){3}}
\put(74,2){\line(0,-1){2}}
\put(108.5,10){\circle{5}}
\put(108.5,7.5){\line(0,1){5}}
\put(106,10){\line(1,0){5}}
\put(89,30){\vector(0,-1){9}}
\put(100,10){\vector(1,0){5}}
\put(112,10){\vector(1,0){10}}
\put(124,8){\framebox(20,4){~}}
\put(144,8){\line(0,1){4}}
\multiput(131,8)(0,0.8){5}{\line(1,0){1}}
\put(131,8){\framebox(1,4){~}}
\put(131.5,8){\line(0,-1){8}}
\put(88,25){\makebox(0,0)[rc]{$\key$}}
\put(134,13.5){\makebox(0,0)[b]{$\plaint_{i+1}$}}
\put(124,-1){\makebox(0,0)[lt]{1-bit Fehler}}
\end{picture}
\caption{Fehlererweiterung beim CBC Mode}
\label{pc:fehl.cbc}
\end{center}
\end{figure}

Der CBC Mode ist selbstkorrigierend, Übertragungsfehler innerhalb eines Blockes wirken sich bei der Entschlüsselung nur auf diesen und den direkt nachfolgenden
Block aus. Daraus folgt, dass der Initialisierungsvektor zum Start des Systems zwischen Sender und Empfänger nicht vereinbart sein muss. Wählen Sender und
Empfänger je einen zufälligen $IV,$ so kann nur Block $\plaint_1$ vom Empfänger nicht korrekt wiedergewonnen werden.

Die eben erläuterte Art der Fehlererweiterung des CBC Mode beinhaltet ein Sicherheitsrisiko. Durch das gezielte Verändern eines Bits im Chiffretext wird zwar
der zugehörige Klartextblock völlig zerstört, aber im nächsten Klartextblock wird genau dieses Bit negiert, was von entscheidender Bedeutung sein kann\footnote{Ein erfolgreich durchgeführter Angriff auf diese Schwachstelle ist  \href{http://www.heise.de/security/artikel/Erfolgreicher-Angriff-auf-Linux-Verschluesselung-2072199.html}{hier} beschrieben.}. Nach
Auftreten eines Fehlers sollte also die ganze Übertragungsfolge wiederholt und nur dann akzeptiert werden, wenn alle Blöcke korrekt übertragen wurden. Um dies
zu prüfen, können auf den Klartext bezogene Prüfnummern verwendet werden.

\subsubsection{CTR - Counter Mode}

Betrachten wir nun einen Betriebsmodus, der die Vorteile von CBC beinhaltet und gleichzeitig das Parallelisieren der Verschlüsselung und Entschlüsselung ermöglicht. Dieser Modus ist der \textit{Counter Mode} (CTR). Er funktioniert wie folgt:

\begin{figure}[h]
	\centering
	\subfigure[Verschlüsselung im CTR]{
		\tikzstyle{encrypt}=[draw,fill=black!15,rectangle,minimum size=20pt,inner sep=0pt]
		\begin{tikzpicture}
		\newcommand{\n}{3}
		\foreach \nr in {1, ..., \n}{
			\node (M\nr) at (0,{(\n-\nr)*3}) {$M_\nr$};
			\node (x\nr)[XOR] at (2,{(\n-\nr)*3}) {};
			\node (E\nr)[encrypt] at (2,{(\n-\nr)*3+1}) {$\enc$};
			\node (K\nr) at (2,{(\n-\nr)*3+2}) {$K$};
			\node (C\nr) at (4,{(\n-\nr)*3}) {$C_\nr$};
			\draw[->,very thick] (M\nr) -- (x\nr);
			\draw[->,very thick] (x\nr) -- (C\nr);
			\draw[->,very thick] (K\nr) -- (E\nr);
			\draw[->,very thick] (E\nr) -- (x\nr);
		}
		\foreach \nr in {1, ..., \n}{
			\node (IV\nr) at (0.5,{(\n - \nr)*3+1}) {$IV + \nr$};
			\draw[->, very thick] (IV\nr) -- (E\nr);
		}
		\end{tikzpicture}
	}
	\hfill
	\subfigure[Entschlüsselung im CTR]{
		\tikzstyle{encrypt}=[draw,fill=black!15,rectangle,minimum size=20pt,inner sep=0pt]
		\begin{tikzpicture}
		\newcommand{\n}{3}
		\foreach \nr in {1, ..., \n}{
			\node (C\nr) at (0,{(\n-\nr)*3}) {$C_\nr$};
			\node (x\nr)[XOR] at (2,{(\n-\nr)*3}) {};
			\node (E\nr)[encrypt] at (2,{(\n-\nr)*3+1}) {$\enc$};
			\node (K\nr) at (2,{(\n-\nr)*3+2}) {$K$};
			\node (M\nr) at (4,{(\n-\nr)*3}) {$M_\nr$};
			\draw[->,very thick] (C\nr) -- (x\nr);
			\draw[->,very thick] (x\nr) -- (M\nr);
			\draw[->,very thick] (K\nr) -- (E\nr);
			\draw[->,very thick] (E\nr) -- (x\nr);
		}
		\foreach \nr in {1, ..., \n}{
			\node (IV\nr) at (0.5,{(\n - \nr)*3+1}) {$IV + \nr$};
			\draw[->, very thick] (IV\nr) -- (E\nr);
		}
		\end{tikzpicture}
	}
\end{figure}
\bigskip

\noindent Zu einer gegebenen Nachricht \(M = M_1M_2 \dots M_n\) berechnet sich das dazugehörige Chiffrat \(C = C_1C_2 \dots Cn\) durch:
\begin{eqnarray*}
	\ciphert_i	& \colon=	& \enc_\key(IV + i) \oplus \plaint_i
\end{eqnarray*}

Analog zu CBC verwendet CTR einen Initialisierungsvektor \(IV\), der zufällig und gleichverteilt vor jedem Verschlüsselungsvorgang gewählt werden muss. Der Unterschied zu CBC liegt in der Verschlüsselung: Zum Verschlüsseln eines Klartext-Blocks wird kein vorher berechneter Chiffrat-Block benötigt. Stattdessen wird für jedes \(C_i\) der \(IV\) um 1 erhöht; für keine zwei Chiffrat-Blöcke \(C_i\), \(C_j\) (\(i \neq j\)) wird die gleiche Eingabe an die \(\enc\)-Funktion übergeben. Damit stellen wir sicher, dass gleiche Nachrichtenblöcke auf unterschiedliche Chiffrat-Blöcke abgebildet werden. Um einen im CTR-Modus verschlüsselten Text \(C = C_1C_2 \dots Cn\) zu entschlüsseln, gehen wir folgendermaßen blockweise vor:
\begin{eqnarray*}
	\plaint_i	& \colon=	& \enc_\key(IV + i) \oplus \ciphert_i
\end{eqnarray*}

Wir sehen, dass das Entschlüsseln, wie bei CBC, parallelisierbar ist und dass der \(IV\) bekannt sein muss. Ein wesentlicher Unterschied zu den vorangegangenen Modi ist, dass wir zum Ver- und Entschlüsseln dieselbe Funktion \(\enc\) benutzen. \(\enc\) muss folglich nicht invertierbar sein.

Betrachten wir die Fehlerfortplanzung des CTR-Modus, stellen wir fest, dass wir gezielt Bits im Block \(M_i\) manipulieren können, indem wir sie in dem entsprechenden Chiffratblock \(C_i\) verändern. Mittels CTR verschlüsselte Nachrichten sind somit homomorph veränderbar. Wird hingegen der gewählte \(IV\) verändert, erfolgt eine komplette Zerstörung der ursprünglichen Nachricht. 

\subsubsection{Galois/Counter Mode}

Der oben vorgestellte CTR-Modus besitzt bereits einige wünschenswerte Eigenschaften, die eine Blockchiffre haben sollte. Ein Nachteil von CTR ist jedoch, dass er homomorph veränderbar ist: Ein Angreifer kann gezielt Bits verändern und dabei unbemerkt bleiben. Wir sind nun auf der Suche nach einem Modus, bei dem keine unbemerkten Veränderungen an der Nachricht \(M\) durchgeführt werden können.

Im \textit{Galois/Counter Mode} (GCM) finden wir einen Modus, der die obige Eigenschaft mit den Vorteilen von CTR vereint. Zusätzlich zum Verschlüsseln einer Nachricht \(M\), welches analog zu CTR erfolgt, wird zeitgleich eine Signatur\footnote{Signaturen werden ausführlicher in \hyperref[cha6]{Kapitel 6} behandelt.} der Nachricht generiert. Das Verschlüsseln von \(M\) im GCM liefert uns ein Chiffrat-Signatur-Paar (\(C, \sigma\)), wobei sich die Signatur \(\sigma\) aus \(C\) ergibt\footnote{Für genauere Informationen verweisen wir auf die \href{http://csrc.nist.gov/groups/ST/toolkit/BCM/documents/proposedmodes/gcm/gcm-spec.pdf}{Veröffentlichung der Entwickler.}}. \(\sigma\) wird verwendet, um Manipulationen an \(C\) zu erkennen.
Vereinfacht dargestellt erhalten wir folgende Operationen:
\bigskip

%\begin{align*}
\begin{math}
	EncryptAndAuthenticate(M) := (Encrypt((M), Authenticate(C)) = (C, \sigma) \\
	\\
	\indent DecryptAndAuthenticate(C, \sigma) := \begin{cases}
		M           & Authenticate(C) == \sigma\\
		error       & \text{sonst}
	\end{cases}
\end{math}
%\end{align*}
\bigskip

Wir sehen, dass wir nur dann aus einem Chiffrat-Signatur-Paar (\(C, \sigma\)) die dazugehörige Nachricht \(M\) erhalten, wenn die Signatur von \(C\) identisch mit der übergebenen Signatur \(\sigma\) ist. Wurde also ein Chiffrat \(C\) mit Signatur \(\sigma\) zu einem Chiffrat \(C'\) mit Signatur \(\sigma'\) verändert, schlägt die Entschlüsselung \(DecryptAndAuthenticate(C', \sigma)\) fehl, da die übergebene Signatur nicht mit der Signatur des Chiffrats übereinstimmt.

\subsubsection{Zusammenfassung}
Betrachten wir abschließend eine Gegenüberstellung der hier vorgestellten Betriebsmodi.
\newpage

\captionsetup[table]{labelformat=empty}
\captionsetup[table]{singlelinecheck=false}
\captionsetup{font=footnotesize}

\begin{table}[h]
\centering
	\begin{tabularx}{\textwidth}{ | X | X | X | X | X |} 
		\hline
		  & ECB & CBC & CTR & GCM \\ 
		\hline
		Hauptsächliche Verwendung & Für Nachrichten, die kürzer als ein Block sind & Für Nachrichten, die länger als ein Block sind 
		& Für Nachrichten, die länger als ein Block sind & Für Nachrichten, die länger als ein Block sind und vor Manipulationen geschützt werden müssen \\ 
		\hline
		\parbox{3cm}{IND-CPA\(^{\ast}\) } sicher & Nein & Ja\(^{\ast\ast}\) & Ja\(^{\ast\ast}\) & Ja\(^{\ast\ast}\) \\
		\hline
		 Parallelisierbar & Ja & Nur Entschlüsselung  & Ja & Ja, das Signieren selbst aber nicht \\ 
		\hline
		Bit-Fehler im Block X & Block X zerstört & Block X zerstört und 1 Bit im Block (X - 1) geändert 
		& 1 Bit verändert & 1 Bit verändert und geänderte Signatur \\
		\hline
	\end{tabularx}
	\caption{\(^{\ast}\) IND-CPA ist ein Sicherheitsbegriff und wird in \hyperref[def:ind-cpa]{Abschnitt \ref{def:ind-cpa}} definiert \\ 
		\(^{\ast\ast}\) Hierfür muss der \(IV\) vor jeder Verschlüsselung zufällig gleichverteilt gewählt werden}
\end{table}

\captionsetup[table]{labelformat=default}
\captionsetup[table]{singlelinecheck=true}
\captionsetup{font=normal}
 
\bigskip

Die hier vorgestellten Modi sind nur ein Teil einer Vielzahl an Betriebsmodi. Zusammenfassend betrachtet hat jeder Modus mehr oder weniger wünschenswerte Eigenschaften, weshalb man sich gut überlegen sollte, ob der gewählte Modus im Kontext der jeweiligen Anwendung die nötige Sicherheit garantieren kann. In einem Szenario, in dem auch vor aktiven Angriffen Schutz geboten werden soll,
kann nur der \textit{Galois/Counter Mode} Sicherheit bieten. Von denen hier vorgestellten Modi ist er der einzige, der aufgrund des Chiffrat-Signatur-Paars neben Vertraulichkeit auch Datenintegrität sicherstellt. Jedoch gibt es noch immer Anwendungen, die den Modus
nicht unterstützen.