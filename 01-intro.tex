\chapter{Einleitung}
\label{cha:intro}

\section{Was ist Sicherheit?}
Sicherheit bedeutet, dass Schutz geboten wird. Was wird geschützt? Vor
wem? Wie wird es geschützt? Wer schützt? Es gibt zwei verschiedene
Ansätze, die beide unter den deutschen Begriff \emph{Sicherheit} fallen:
\emph{Betriebssicherheit} (engl. safety) und \emph{Angriffsicherheit}
(engl. security).
\begin{description}
\item[Betriebssicherheit:] Unter \emph{Betriebssicherheit}
  versteht man die Sicherheit einer Situation, die von einem System
  geschaffen wird: Ist der Betrieb eines Systems sicher? Das bedeutet vor
  allem, dass keine externen Akteure betrachten werden: Niemand
  manipuliert das System! Diese Art von Sicherheit wird mit Methoden
  erreicht, die wahrscheinliche Fehlerszenarien abdecken und
  verhindern. Beispiele für Systeme, die uns Betriebssicherheit gewähren,
  sind Arbeitsschutzkleidung zur Vermeidung von Arbeitsunfälle,
  Backup-Systeme zur Vorbeugung gegen den Ausfall von Komponenten oder
  elektrische Sicherungen, um uns vor gefährlichen Kurzschlussströmen zu
  schützen.
\item[Angriffssicherheit:] Unter \emph{Angriffssicherheit}
  versteht man die Sicherheit eines Systems in Bezug auf das externe
  Hinzufügen von Schäden: Ist es möglich, das System von Außen zu
  manipulieren? Anders als bei \emph{Betriebssicherheit} betrachten wir
  keine Schäden, die durch den aktuellen Zustand des Systems entstehen
  können. Wir betrachten Schäden die von einem externen Akteur, im
  folgenden \emph{Angreifer} genannt, ausgehen. Dabei gehen wir davon aus,
  dass ein Angreifer Schwachstellen des Systems gezielt sucht und
  verwendet. Aus diesem Grund genügt es nicht, wahrscheinliche
  Fehlerszenarien zu betrachten. Es ist vielmehr nötig, alle
  Angriffsmöglichkeiten zu unterbinden. Beispiele für diese Art von
  Sicherheit sind gepanzerte Fahrzeuge, Türschlösser gegen Einbrecher und
  Wasserzeichen, um das Fälschen von Banknoten zu erschweren.
\end{description} 
In dieser Vorlesung beschäftigen wir uns
ausschließlich mit dem Konzept der Angriffssicherheit. Darüber hinaus
beschäftigen wir uns nur mit dem Schutz informationstechnischer Systeme.

\section{Grundlagen}
Betrachten wir ein informationstechnisches System. Es existierten
zahlreiche Arten von Attacken, vor denen wir uns durch verschiedene
Techniken schützen müssen. Es ist selten hilfreich, das Gesamtsystem als
Einheit zu betrachten. Dafür ist es einfach zu komplex. Stattdessen
zerlegen wir es in kleinere "`Bausteine"', für deren Sicherheit wir
einzeln garantieren können. Diese Vorlesung stellt die wichtigsten
Bausteine vor. In diesem Abschnitt geben wir einen Überblick über die
wichtigsten Grundbegriffe. Im Anschluss werden wir stets auf die
weiterführenden Kapitel verweisen.

\subsection{Verschlüsselung}

Ziel der Verschlüsselung ist es, Informationen auf eine bestimmte
Personengruppe zu begrenzen. Stellen wir uns vor, ein Sender Bob möchte
eine Nachricht an eine Empfängerin Alice übermitteln. Die Nachricht ist
privat, doch Eve lauscht. Können Alice und Bob kommunizieren, ohne das
Eve sinnvolle Informationen erhält? Wie?

Ein Verschlüsselungsverfahren (\emph{Chiffre}) besteht aus einer oder
mehreren mathematischen Funktionen, die zur Ver- und Entschlüsselung
einer Nachricht eingesetzt werden. Bei der Verschlüsselung wird ein
Klartext (eine \emph{Nachricht}) in einen Geheimtext (ein
\emph{Chiffrat}) umgewandelt. Das Chiffrat soll einem unautorisierten
Dritten keine Informationen über die Nachricht offenbaren. Das Chiffrat
kann dann durch Entschlüsselung wieder in den Klartext umgewandelt
werden. Verschlüsselung wird auch als \emph{Chiffrierung},
Entschlüsselung als \emph{Dechiffrierung} bezeichnet.

In heutigen Algorithmen wird zur Chiffrierung und Dechiffrierung noch
eine weitere Information, der \emph{Schlüssel}, benutzt. Diese Situation
ist in Abbildung~\ref{fig:encryption:principle} dargestellt. Ist der
Schlüssel für Ver- und Entschlüsselung gleich, so spricht man von einem
\emph{symmetrischen} Verfahren\indexEncryptionSymm. Sind die Schlüssel
verschieden, handelt es sich um ein \emph{asymmetrisches}
Verfahren\indexEncryptionAsymm. Symmetrische Verfahren werden in
\autoref{cha:symencryption} vorgestellt, asymmetrische Verfahren in
\autoref{ch:asymmenc}. Klartext und Chiffrat können aus beliebigen
Zeichen bestehen. Im Kontext computergestützter Kryptographie sind beide
normalerweise binär kodiert.

Für den Fall, dass Bob seine Nachricht an Alice vor dem Senden
verschlüsselt, können die beiden ihre Kommunikation vor Eve
verbergen. Im Gegensatz zu Eve sollte Alice die Nachricht natürlich
entschlüsseln können. Ein Chiffrat muss jedoch nicht immer versendet
werden. Es kann auch zu Speicherung auf einem Datenträger vorgesehen
sein.
\begin{figure}[h]
  \begin{center}
    \unitlength=1mm
    \linethickness{0.4pt}
    \begin{picture}(120,20)
      \put(0,5){\vector(1,0){20}}
      \put(10,6){\makebox(0,0)[cb]{$\plaint$}}
      \put(35,15){\vector(0,-1){7.5}}
      \put(38,10){\makebox(0,0)[cb]{$\key$}}
      \put(20,2.5){\framebox(30,5){\enc}}
      \put(50,5){\vector(1,0){20}}
      \put(60,6){\makebox(0,0)[cb]{$\ciphert$}}
      \put(85,15){\vector(0,-1){7.5}}
      \put(88,10){\makebox(0,0)[cb]{$\widetilde{\key}$}}
      \put(70,2.5){\framebox(30,5){\dec}}
      \put(100,5){\vector(1,0){20}}
      \put(110,6){\makebox(0,0)[cb]{$\plaint$}}
    \end{picture}
  \end{center}
  \caption{$\mathrm{\dec_{\widetilde{\key}}(\enc_\key(\plaint))= \plaint}$. Falls $\key = \tilde{\key}$ handelt es sich um ein symmetrisches
    Verschlüsselungsverfahren, ist $\key \neq \tilde{\key}$, so ist es ein asymmetrisches Verfahren.}
  \label{fig:encryption:principle}
\end{figure}

\noindent Üblicherweise benutzen wir folgende Abkürzungen:

\begin{center}
	\begin{tabular}{ l l l }
		Nachricht: & \plaint\ & (engl. \emph{message})\\
		Chiffrat: & \ciphert\ & (engl. \emph{ciphertext})\\
		Schlüssel: & \key\ & (engl. \emph{key})\\
		Chiffrierung: & \enc\ & (engl. \emph{encryption})\\
		Dechiffrierung: & \dec\ & (engl. \emph{decryption})\\
	\end{tabular}
\end{center}

\subsubsection{Geheime Verfahren}

Zwar gibt es eine ganze Reihe von Verschlüsselungsverfahren ohne
Schlüssel, allerdings hängt deren Sicherheit allein davon ab, dass der
Algorithmus geheim bleibt. Im Kontext von Algorithmen, deren Sicherheit
auf der Geheimhaltung des Verfahren beruht, spricht man auch von
\emph{security by obscurity}. Solche Algorithmen sind unflexibel und aus
heutiger Sicht unsicher. Sie sind daher eher von historischem Interesse
und werden im Folgenden nicht näher betrachtet. Stattdessen hat sich
Kerckhoffs' Prinzip etabliert.

\subsubsection{Kerckhoffs' Prinzip}

Kerckhoffs' Prinzip ist ein Grundsatz moderner Kryptographie. Er wurde
im 19. Jahrhundert von Auguste Kerckhoffs formuliert
\cite{Kerckhoffs1883}. 
\begin{quote}
  \grqq{}The cipher method must not be required to be secret, and it
  must be able to fall into the hands of the enemy without
  inconvenience.\grqq
\end{quote}
Anders ausgedrückt darf die Sicherheit eines Verschlüsselungsverfahren
nur von der Geheimhaltung des Schlüssels und nicht von der Geheimhaltung
des Algorithmus abhängen.  Kerckhoffs' Prinzip findet in den meisten
heutigen Verschlüsselungsverfahren Anwendung. Gründe dafür sind:
\begin{itemize}
\item Es ist einfacher, einen Schlüssel als einen Algorithmus
  geheim zu halten.
\item Es ist einfacher, einen kompromittierten Schlüssel zu
  ersetzen, statt einen ganzen Algorithmus zu tauschen. Tatsächlich ist es
  gängige Sicherheitspraxis, den Schlüssel regelmäßig zu wechseln, selbst
  wenn dieser nicht bekannt geworden ist.
\item Bei vielen Teilnehmerpaaren (z.B. innerhalb einer Firma)
  ist es um einiges einfacher, unterschiedliche Schlüssel zu verwenden,
  statt unterschiedliche Algorithmen für jede Kombination zu entwerfen.
\item Veröffentlichte Verfahren können von vielen Fachleuten
  untersucht werden, wodurch eventuelle Fehler wahrscheinlicher auffind-
  und behebbar sind.
\item Da der Schlüssel keinen Teil des Algorithmus (bzw. seiner
  Implementierung) darstellt, ist er im Gegensatz zum Algorithmus nicht
  anfällig gegen Reverse-Engineering.
\item Öffentliche Entwürfe ermöglichen die Etablierung von
  Standards.
\end{itemize} 
Diese Gründe mögen einleuchtend sein. Trotzdem wurde Kerckhoffs' Prinzip
immer wieder zugunsten geheimer Verfahren ignoriert, was zu fatalen
Ergebnissen führte. Es sollten nur standardisierte und öffentlich
getestete Verfahren verwendet werden.